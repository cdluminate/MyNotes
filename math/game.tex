% Game Theory
% Pandoc LaTeX
\title{Game Theory}

\section{Introduction}

[\href{https://en.wikipedia.org/wiki/Game_theory}{Game Theory}]

Game theory is a bag of analytical tools designed to help us understand
the phenomena that we observe when decision-makers interact. The basic
assumptions that underlie the theory are that decision-makers pursue
well-defined exogenous objectives (they are rational ) and take into
account their knowledge or expectations of other decision-makers’ behavior
(they reason strategically). \footnote{page. 1}

The models we study assume that each dicision-maker is ``rational'' in
the sense that he is aware of his alternatives, forms expectations about
any unknowns, has clear preferences, and chooses his action deliberately
after some process of optimization. In the absence of undertainty the
following elements constitute a model of rational choice. \footnote{page. 4}
\begin{itemize}
	\item A set $A$ of actions from which the decision-maker makes a choice.
	\item A set $C$ of possible consequences of these actions.
	\item A consequence function $g: A\mapsto C$ that associates a consequence
		with each action.
	\item A preference relation (a complete transitive reflexive binary
		relation) $\succsim$ on the set $C$.
\end{itemize}

Sometimes the decision-maker's preferences are specified by giving a utility
function: $U: C\mapsto \Re$, which defined a preference relation $\succsim$
by the condition $x\succsim y$ if and only if $U(x)\geqslant U(y)$.

Strategic game, a model of strategic interaction, specifies for each player
a set of possible actions and a preference ordering over the set of possible
action profiles. Nash equilibrium is the most widely used solution concept
for strategic games. \footnote{page. 9}

\section{Nash Equilibrium}

\subsection{Strategic Games}

A strategic game consists of
\begin{itemize}
	\item a finite set $N$ (the set of players)
	\item for each player $i\in N$ a nonempty set $A_i$ (the set of actions
		available to player $i$)
	\item for each palyer $i\in N$ a preference relation $\succsim_i$ on
		$A = \times_{j\in N} A_j$ (the preference relation of player $i$)
\end{itemize}
Different from a dicision problem, in a strategic game each player may care
not only about his own action but also about the actions taken by the other
players. \foonote{page. 11}

A common interpretation of a strategic game is that it is a model of an event
that occurs only once; each player knows the details of the game and the
fact that all the players are ``rational'', and the players choose their
actions simutaneously and independently.

\subsection{Nash Equilibrium}

[\href{https://en.wikipedia.org/wiki/Nash_equilibrium}{Nash Equilibrium}]

A Nash equilibrium of a strategic game $\langle N, (A_i), (\succsim_i)\rangle$
is a profile $a^* \in A$ of actions with the property that for every palyer
$i\in N$ we have
$$(a^*_{-i},a^*_i) \succsim_i (a^*_{-i},a_i) ~\text{for all}~ a_i \in A_i$$

For any $a_{-i} \in A_{-i}$ define $B_i(a_{-i})$ to be the set of player $i$'s
best actions given $a_{-i}$:
$$ B_i(a_{-i}) = \{ a_i\in A_i:
	(a_{-i},a_i) \succsim_i (a_{-i},a'_i) ~\text{for all}~ a'_i \in A_i\} $$
The set-valued function $B_i$ is called the best-response function of player
$i$. A Nash equilibrium is a profile $a^*$ of actions for which
$$ a^*_i \in B_i(a_{-i}^*) ~\text{for all}~ i \in N $$
This alternative formulation of the definition points us to a (not necessarily
efficient) method of finding Nash equilibria: first calculate the best response
function of each player, then find a profile $a^*$ of actions which satisfies
the above condition.

\subsection{Examples}

[\url{https://en.wikipedia.org/wiki/Prisoner\%27s_dilemma}]

\subsection{Existence of a Nash Equilibrium}

The existence of an equilibrium shows that the game is consistent with a
steady state solution.

LEMMA: Kakutani's fixed point theorem: 
\begin{quote}
Let $X$ be a compact convex subset
of $\Re^n$ and let $f:X\mapsto X$ be a set-valued function for which
	\begin{itemize}
		\item for all $x\in X$ the set $f(x)$ is nonempty and convex
		\item the graph of $f$ is closed (i.e. for all sequences $\{x_n\}$
			and $\{y_n\}$ such that $y_n\in f(x_n)$ for all $n$,
			$x_n\mapsto x$, and $y_n \mapsto y$, we have $y\in f(x)$
	\end{itemize}
	Then there exists $x^*\in X$ such that $x^* \in f(x^*)$.
\end{quote}

Define a preference relation $\succsim_i$ over $A$ to be quasi-concave on
$A_i$ if for every $a^*\in A$ the set
$\{a_i\in A_i:(a^*_{-i},a_i)\succsim_i a^*\}$ is convex.
\begin{quote}
	The strategic game $\langle N, (A_i), (\succsim_i)\rangle$ has a Nash
	equillibrium if for all $i \in N$, the set $A_i$ of actions of player
	$i$ is a nonempty conpact convex subset of a Euclidian space, and the
	preference relation $\succsim_i$ is continuous, quasi-concave on $A_i$.
\end{quote}

\subsection{Strictly Competitive Games}

todo: p. 21

%    \subsubsubsection{Stragetic Games}
%
%  \subsubsection{Mixed Equilibrium}
%
%\subsection{Extensive games with perfect information}
%
%  \subsubsection{Extensive, perfect info}
%
%\subsection{Extensive games with imperfect information}
%
%  \subsubsection{Extensive, imperfect info}
%
%  \subsubsection{Sequential equilibrium}
%
%\subsection{Coalitional games}
%
%  \subsubsection{Core of Coalitional games}


\section{Reference}

1. Martin J. Osborne, Ariel Rubinstein, {\it A Course in Game Theory}, MIT Press.
