
\section{Linear Space and Linear Transformation}

\url{https://en.wikipedia.org/wiki/Vector_space#Definition}

\subsection{Base of a Linear Space}

Given a set of vectors ${x_1, x_2, \ldots}$ from a linear
space $V$, if there is no solution to $\sum_i c_i x_i = 0$ s.t. $\sum|c_i|=0$,
the vectors $x_*$ are linearly independent to each other. The maxium size of
a set of linearly independent vectors from linear space $V$ is the
dimentionality of the space, denoted as $\text{dim}V=n$. Such a group of
vectors is called a base of space $V$.

Any vector from the space $V$ can be represented as a linear combination
of the base vectors $x_1,\ldots,x_n$, i.e.
$\forall x\in V^n, x = \sum_i \xi_i x_i$. The coefficients for this linear
combination is called coordinate, which can be denoted as column vector
$(\xi_1,\ldots,\xi_n)^T$. In this way, the vector can be denoted as
$$x = [x_1, \ldots, x_n][\xi_1,\ldots,\xi_n]^T$$

Assume that we have another set of base vectors $(y_1,\ldots,y_n)$, where 
each base vector can still be represented as a linear combination of the
original base vectors, i.e. $(y_1,y_2,\ldots,y_n) = (x_1,x_2,\ldots,x_n)C$.
Here matrix $C$ is called the base shift matrix, and $y_*$ is the new base
while $x_*$ is the original base. Given $\forall x\in V^n$, and
$x = \sum_i \xi_i x_i = \sum_i \eta_i y_i$, we have
\begin{align}
(x_1,\ldots,x_n)(\xi_1,\ldots,\xi_n)^T = (y_1,\ldots,y_n)
	(\eta_1,\ldots,\eta_n)^T \\= (x_1,\ldots,x_n)C(\eta_1,\ldots,\eta_n)^T
\end{align}
That is to say
$$(\xi_1,\ldots,\xi_n)^T = C(\eta_1,\ldots,\eta_n)^T$$
$$(\eta_1,\ldots,\eta_n)^T = C^{-1}(\xi_1,\ldots,\xi_n)^T$$

\subsection{Subspace}

$V_s \subset V$ and $V_s \neq \Phi$, s.t.  $\forall x, y \in V_s, x+y\in V_s$,
and $\forall x\in V_s, kx \in V_s$. Note, the dim of this subspace is less or
equal to that of the original linear space, i.e.
$\text{dim} V_s \leq \text{dim} V$.

{\bf Column space} $R(A) = \text{lincomb}(a_1,\ldots,a_n) = \{Ax|x\in R^n\}$
where $A\in R^{m\times n}$ and $a_* \in R^m$ is the columns of matrix $A$.
Similarly, the row space can be defined as
$R(A^T) = \{A^Tx|x\in R^m\} \subset R^n$.
Note that $\text{rank} A = \text{dim} R(A) = \text{dim} R(A^T)$.
Or equivalently, the column space of a matrix $A$ is the span of its column
vectors, i.e. $R(A) = \text{Span}\{a_1,a_2,\ldots,a_n\}$.

{\bf Null space} $N(A) = \{x|Ax=0\}$ where $A\in R^{m\times n}$. Note,
$\text{rank} A + n(A) = n$ where $n(A) = \text{dim} N(A)$.

\subsection{Linear Transformation and Matrix}

A linear transformation $T$ must satisfy the following property:
$$ T(kx + ly) = k(Tx) + l(Ty)$$
where $k, l$ are constants, vector $x, y$ are from vector space $V$.

Given a vector $a \in V$, and $a = [x_1,\ldots,x_n][\xi_1,\ldots,\xi_n]^T$,
a linear transformation $T$ will update it to a new vector, i.e.
$$ Ta = T([x_1,\ldots][\xi_1,\ldots]^T) = T([x_1,\ldots])[\xi_1,\ldots]^T
   = [x_1,\ldots]A[\xi_1,\ldots]^T $$
which means the new coordinate of this vector after transformation is
$[\eta_1,\ldots]^T = A[\xi_1,\ldots]^T$. At the same time, matrix $A$ is
called the matrix of transformation $T$ with respect to the base $[x_1,\ldots]$.

\subsection{Similar Matrix}

Assume that the matrix of transformation $T$ with respect to two bases
$[x_1,\ldots]$ and $[y_1,\ldots]$ are $A$ and $B$, respectively. Namely
$$T[x_1,\ldots] = [x_1,\ldots]A, T[y_1,\ldots] = [y_1,\ldots]B,
                                  [y_1,\ldots] = [x_1,\ldots]C$$
$$T[y_1,\ldots] = T([x_1,\ldots]C) = T([x_1,\ldots])C =
   [x_1,\ldots]AC = [y_1,\ldots]C^{-1}AC $$
We call that $B$ is similar to $A$.

\subsection{Euclid Space and Unitary Space}

A linear space with a distance function defined as $f: V\times V \mapsto R$.
