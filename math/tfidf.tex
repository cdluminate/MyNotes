\section{TF-IDF}

{\bf TF-IDF} stands for term frequency-inverse document frequency, and the
tf-idf weight is a weight often used in information retrieval and text mining.
This weight is a statistical measure used to evaluate how important a word is
to a document in a collection or corpuses. Tf-idf can be successfully used for
stop-words filtering in various subject fields including text summarization and
classification.

One of the simplest ranking function is computed by summing the tf-idf for each
query term; many more sophisticated ranking functions are variants of this
simple model.

\subsection{HowTo}

Typically, the tf-idf weight is composed by two terms: the first computes the
normalized Term Frequency (TF), aka. the number of times a word appears in a
document, divided by the total number of words in that document; the second
term is the Inverse Document Frequency (IDF), computed as the logarithm of the
number of the documents in the corpus divided by the number of documents where
the specific term appears. 

\begin{quote}
TF: Term Frequency, which measures how frequently a term occurs in a
document. Since every document is different in length, it is possible that a
term would appear much more times in long documents than shorter ones. Thus,
the term frequency is often divided by the document length (aka. the total
number of terms in the document) as a way of normalization:

TF(t) = (Number of times term t appears in a document) / (Total number of terms
in the document).
\end{quote}

\begin{quote}
IDF: Inverse Document Frequency, which measures how important a term is. While
computing TF, all terms are considered equally important. However it is known
that certain terms, such as "is", "of", and "that", may appear a lot of times
but have little importance. Thus we need to weigh down the frequent terms while
scale up the rare ones, by computing the following:

IDF(t) = log\_e(Total number of documents / Number of documents with term t in it).
\end{quote}

\subsection{Example}

Consider a document containing 100 words wherein the word cat appears 3 times.
The term frequency (i.e., tf) for cat is then (3 / 100) = 0.03. Now, assume we
have 10 million documents and the word cat appears in one thousand of these.
Then, the inverse document frequency (i.e., idf) is calculated as
log(10,000,000 / 1,000) = 4. Thus, the Tf-idf weight is the product of these
quantities: 0.03 * 4 = 0.12.

\subsection{Another Example}

\begin{verbatim}
'''
http://blog.christianperone.com/2011/10/machine-learning-text-feature-extraction-tf-idf-part-ii/
'''
from sklearn.feature_extraction.text import CountVectorizer
from sklearn.feature_extraction.text import TfidfTransformer
from sklearn.metrics.pairwise import cosine_similarity

documents = (
    "The sky is blue",
    "The sun is bright",
    "The sun in the sky is bright",
    "We can see the shining sun, the bright sun"
    )

vectorizer = CountVectorizer(stop_words=['the', 'is', 'in', 'we', 'can'])
print(vectorizer)

vectorizer.fit_transform(documents)
print(vectorizer.vocabulary_)

tf = vectorizer.transform(documents).todense()
print('TF:', tf)

tfidf = TfidfTransformer(norm="l2")
tfidf.fit(tf)
print('IDF:', tfidf.idf_)

tfidf_mat = tfidf.transform(tf).todense()
print('TFIDF:', tfidf_mat.shape, tfidf_mat)

print(cosine_similarity(tfidf_mat[0], tfidf_mat))
\end{verbatim}


\subsection{SeeAlso}

\url{http://www.tfidf.com/}
\url{https://en.wikipedia.org/wiki/Tf-idf}
\url{http://blog.christianperone.com/2011/09/machine-learning-text-feature-extraction-tf-idf-part-i/}
