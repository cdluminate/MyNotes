\section{Correlation}

There are three types of Correlation coefficients.

\begin{itemize}
\item{\bf Pearson} Pearson produce-moment correlation coefficient, also known
as $r$, $R$, or pearson's $r$, is a measure of the strength and direction of
the linear relationship between two variables that is defined as the covariance
of the variables divided by the product of their standard deviations. This is
the best known and most commonly used type of correlation coefficient; when the
term "correlation coefficient" is used without further qualification, it
usually refers to the pearson product-moment correlation coefficient.
\item{\bf Intra-Class} Intraclass correlation (ICC) is a descriptive statistic
that can be used when quantitative measurements are made on units that are
organized into groups; it describes how strongly units in the same group
resemble each other.
\item{\bf Rank} Rank correlation is a measure of the relationship between the
rankings of two variables or two rankings of the same variable.
\item{\bf Distance} Distance correlation was introduced to address a deficiency
of Pearson's correlation that it's mainly sensitive to a linear relationship
between two variables.
\end{itemize}

\subsection{PPMCC}

$$\rho_{X,Y} = \frac{\text{cov}(X,Y)}{\sigma_X \sigma_Y} $$
where
$$ \text{cov}(X,Y) = E[(X-E[X])(Y-E[Y])] $$

\subsection{Rank Correlation}

\subsubsection{Spearman's Rank Correlation}

The Spearman correlation coefficient is defined as the Pearson correlation
coefficient between the ranked variables. For a sample of size $n$, the $n$ raw
scores $X_i$, $Y_i$ are converted to ranks $rg_{X_i}$, $rg_{Y_i}$, and $r_s$ is
computed from $$ r_s = \rho_{rg_X,rg_Y} = \frac{cov(rg_X, rg_Y)}{\sigma_{rg_X}
\sigma_{rg_Y}} $$ where $\rho$ denotes the usual Pearson correlation
coefficient, but applied to the rank variables.

Example Julia code
\begin{verbatim}
using Statistics
xs, ys = sortperm(x), sortperm(y)
sprman = cor(xs, ys)
\end{verbatim}

\subsubsection{Kendal Rank Correlation Coefficient} 

Any pair of observations $(x_i,y_i)$ and $(x_j,y_j)$ where $i<j$, are said to
be concordant if the ranks for both elements agree: that is, if both $x_i>x_j$
and $y_i>y_j$; or if both $x_i<x_j$ and $y_i<y_j$. They are said to be
discordant, if $x_i>x_j$ and $y_i<y_j$; or if $x_i<x_j$ and $y_i>y_j$. If
$x_i=x_j$ or $y_i=y_j$, the pair is neither concordant nor discordant, but
tied. The Kendal $\tau$ coefficient is defined as:
$$ \tau_a = \frac{n_c-n_d}{n_0} $$
where the coefficient itself must be in the range $-1\le \tau \le 1$.  $n_0$ is
the total number of pair combinations, i.e. $n(n-1)/2$. $n_c$ denotes the
number of concordant pairs, while $n_d$ denotes the number of discordant pairs.

When tied pairs arise in the data, the coefficient may be modified in a number
of ways to keep it in the range $[-1,1]$. See $\tau_b$ and $\tau_c$.

Reference Implementation in Julia
\begin{verbatim}
# https://stackoverflow.com/questions/5284646/rank-items-in-an-array-using-python-numpy
# https://codereview.stackexchange.com/questions/65031/creating-a-list-containing-the-rank-of-the-elements-in-the-original-list
function Rank(X::Vector)::Vector
	sortperm(sortperm(X))
end

function kendalltau(R1::Vector, R2::Vector)::Number
	concordant, discordant, tied = 0, 0, 0
	if length(R1) != length(R2)
		error("Unequal length of ranking vector")
	end
	# Traversal in the upper triangular matrix
	n = length(R1)
	for i in 1:n
		for j in i+1:n
			if (R1[i]-R1[j]) * (R2[i]-R2[j]) > 0
				concordant += 1
			elseif (R1[i]-R1[j]) * (R2[i]-R2[j]) < 0
				discordant += 1
			elseif (R1[i]==R2[i]) || (R1[j]==R2[j])
				tied += 1
			else
				error("unexpected value")
			end
		end
	end
	ta = (concordant - discordant)/(n*(n-1)/2)
	dt = discordant/(n*(n-1)/2)
	println("C", concordant, " D", discordant, " T", tied, " ta", ta, " dt", dt)
	ta # dt
end

x1 = vec([12 2 1 12 2])
x2 = vec([1 4 7 1 0])
s1 = sortperm(x1)
s2 = sortperm(x2)
println(kendalltau(sortperm(s1), sortperm(s2)))

r1 = vec([1 2 3 4 5])
r2 = vec([3 4 1 2 5])
println(kendalltau(r1, r2))
\end{verbatim}

Reference implementation in Python
\begin{verbatim}
import numpy as np

def Rank(X: np.ndarray) -> np.ndarray:
    return X.argsort().argsort()

def kendalltau(r1: np.ndarray, r2: np.ndarray) -> float:
    r1 = r1.ravel()
    r2 = r2.ravel()
    concordant, discordant, tied = 0, 0, 0
    if len(r1) != len(r2):
        raise Exception("length unequal")
    n = len(r1)
    for i in range(n):
        for j in range(i+1,n):
            if (r1[i]-r1[j])*(r2[i]-r2[j]) > 0:
                concordant += 1
            elif (r1[i]-r1[j])*(r2[i]-r2[j]) < 0:
                discordant += 1
            elif (r1[i]==r2[i]) or (r1[j]==r2[j]):
                tied += 1
            else:
                raise Exception("unexpected condition")
    ta = (concordant - discordant)/(n*(n-1)/2)
    dt = discordant/(n*(n-1)/2)
    print(f"C{concordant} D{discordant} T{tied}, ta {ta}, dt {dt}")
    return ta

if __name__ == '__main__':
    x1 = np.array([12,2,1,12,2])
    x2 = np.array([1,4,7,1,0])
    r1, r2 = x1.argsort().argsort(), x2.argsort().argsort()
    print(kendalltau(r1, r2))
    r1 = np.array([1,2,3,4,5])
    r2 = np.array([3,4,1,2,5])
    print(kendalltau(r1, r2))
\end{verbatim}

\subsubsection{Goodman and Kruskal's Gamma}

Similar to the Kendal's $\tau$ coefficient, Goodman's $G$ depends on two
quantities: the number of concordant pairs $n_c$, and the number of discordant
pairs $n_d$. Note that "ties" are dropped.  $$ G = \frac{n_c-n_d}{n_c+n_d} $$
This statistic can be regarded as the maximum likelihood estimator for the
theoretical quantity $\gamma$.

\subsubsection{Somers' D}

$$ D_{YX} = \tau(X,Y)/\tau(X,X) $$

Note that kendall's $\tau$ is symmetric in $X$ and $Y$, whereas Somers' D is
asymmetric in $X$ and $Y$.

\subsection{See Also}

\url{https://en.wikipedia.org/wiki/Correlation_and_dependence}
\url{https://en.wikipedia.org/wiki/Covariance}
\url{https://en.wikipedia.org/wiki/Rank_correlation}
\url{https://en.wikipedia.org/wiki/Correlation_coefficient}
\url{https://en.wikipedia.org/wiki/Kendall_tau_distance}
\url{https://en.wikipedia.org/wiki/Kendall_rank_correlation_coefficient}
