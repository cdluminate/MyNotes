\documentclass[onecolumn,times]{article}
\usepackage{CJK}
\usepackage{amsmath}
\usepackage[margin=1in]{geometry}
\usepackage{indentfirst}
\usepackage{fancyvrb}

\begin{document}
\begin{CJK}{UTF8}{gkai}

\section{原问题: INLP}

\begin{align}
	\max\qquad & \sum_{i,j} c_{ij} w_{ij} \\
	\text{s.t.}\qquad & x_i \sum_j (c_{ij}\prod_{k~(k\neq j)} \overline{c}_{ik} )
	+ \overline{x}_i \prod_j \overline{c}_{ij} = 1,~ j,k\in \Phi_i,~ \forall i\\
	& c_{ij} \cdot \overline{\text{bool}(\sum_{m\in q_{ij}} x_m)} 
	+ \overline{c}_{ij} = 1
	,~ \forall i,j
\end{align}

其中 $x_i$, $c_{ij}$ 为待求的0-1变量。
其余变量如 $q_{ij}$, $\Phi_i$, $w_{ij}$
均已事先给定。函数$\text{bool}(\cdot)$ 为整数转化为布尔型的辅助函数。
我们需要这个函数是因为一个整数(亦即一系列布尔值的和)不存在取反的操作。

可见,这个问题中所有待求变量均为0-1变量,并且约束中存在显然的变量间非线性
关系(乘法),因此这个问题是 INLP (Integer Non-Linear Programming) 问题。
在实践中,这类问题因为其非线性会带来很大的求解难度,所以往往需要尝试将
非线性部分转化为线性,使其成为一个 ILP 问题。

当我们得到一个 ILP 问题之后,这个问题就可以放到常见的 MILP Solver中求解,
	比如 CPLEX, Matlab 或者 Julia的JuMP。 MILP (Mixed-Integer Linear Programming)
,亦即混合整数规划,是 ILP 的更加泛化形式。MILP问题中需要求解的变量为
非整数和整数的混合。

所以,解决本题的关键在于将 INLP 问题转化为 ILP 问题。

\subsection{需要用到的数学技巧}

\textbf{技巧1。} 将 $\text{bool}(\cdot)$ 函数线性化。
$$y=\text{bool}(x_1 + x_2 + \ldots + x_k)$$
	$$y\geq x_i,~ i=1,2,\ldots,k$$
$$y\in\{0,1\}$$
	将$y$代回原问题即消去非线性。

\textbf{技巧2.} 将0-1变量乘积线性化。
$$y=\prod_{k=1}^{K} x_k$$
	$$y\leq x_i,~ i=1,2,\ldots,K$$
$$y\geq (\sum_{k=1}^{K} x_k) - (K - 1)$$
$$y\in\{0,1\}$$
代回原问题。

\subsection{线性化条件 (3)}

	通过技巧1令 $a_{ij}=\text{bool}(\sum x_m)$, 原条件即可化为
\begin{align}
	c_{ij} \cdot \overline{a}_{ij}
	+ \overline{c}_{ij} &= 1 ,~ \forall i,j\\
	a_{ij} &\geq x_m,~ m\in q_{ij},~ \forall i,j\\
	a_{ij} &\in \{0,1\},~ \forall i,j
\end{align}

现在条件中还是存在非线性,我们继续使用技巧2消除乘积。令
	$b_{ij} = c_{ij}\cdot \overline{a}_{ij}$, 
	条件(4)变
\begin{align}
	b_{ij} + \overline{c}_{ij} &= 1,~ \forall i,j\\
	b_{ij} &\leq c_{ij},~ \forall i,j\\
	b_{ij} &\leq \overline{a}_{ij},~ \forall i,j\\
	b_{ij} &\geq c_{ij} + \overline{a}_{ij} - 1,~ \forall i,j
\end{align}
我们注意到$b_{ij} \in \{0,1\}$。

汇总,条件(3)线性化结果
\begin{align}
	a_{ij} &\geq x_m,~ m\in q_{ij},~ \forall i,j\\
	b_{ij} + \overline{c}_{ij} &= 1,~ \forall i,j\\
	b_{ij} &\leq c_{ij},~ \forall i,j\\
	b_{ij} &\leq \overline{a}_{ij},~ \forall i,j\\
	b_{ij} &\geq c_{ij} + \overline{a}_{ij} - 1,~ \forall i,j
\end{align}

\subsection{线性(2)}

条件(2)成分比较复杂,我们分为左右两部分分别线性化。

右半部分。我们令 $d_i = \overline{x}_i \prod_j \overline{c}_{ij}$
且 $d_i$ 为0-1变量,于是我们将原(2)变换为以下问题:
\begin{align}
	\text{左} + d_i &= 1,~ \forall i\\
	d_i &\leq \overline{x}_i,~\forall i\\
	d_i &\leq \overline{c}_{ij},~\forall j\in \Phi_i,~\forall i\\
	d_i &\geq \overline{x}_i + \sum_j(\overline{c}_{ij}) - (\sum_j 1)
\end{align}

左半部分。我们为了方便先将 $x_i$ 搬到括号内。我们令
$e_{ij} = x_i c_{ij} \prod_{k\neq j} \overline{c}_{ik}$, 于是(2)被我们
转化成
\begin{align}
	\sum_j e_{ij} + d_i &= 1,~ \forall i\\
	d_i &\leq \overline{x}_i,~\forall i\\
	d_i &\leq \overline{c}_{ij},~\forall j\in \Phi_i,~\forall i\\
	d_i &\geq \overline{x}_i + \sum_j(\overline{c}_{ij}) - (\sum_j 1)\\
	e_{ij} &\leq x_i\\
	e_{ij} &\leq c_{ij}\\
	e_{ij} &\leq \overline{c}_{ik}~(k\neq j)\\
	e_{ij} &\geq x_i + c_{ij} + \sum \overline{c}_{ik} - (\sum_j 1)
\end{align}

\section{使用Matlab的MILP solver求解}

我们随机给定一些初始变量。令问题规模为$3$ ($i\in \{1,2,3\}$),
设 $w_{ij}=1$, $q_{ij}$ 为一个随机的0-1矩阵。

\subsection{Matlab MILP 编程模型}

Matlab 的 MILP 求解器使用固定的 MILP 问题模型:
\begin{align}
	\min\qquad & f^T x\\
	\text{s.t.}\qquad & x~\text{(intcon) 为整数}\\
	A\cdot x &\leq b\\
	Aeq\cdot x &= beq\\
	lb &\leq x \leq ub
\end{align}
其中的 \verb|intcon| 是一个矢量,指定哪几个变量 $x$ 是整数(因为这是
一个MILP求解器),其中的值为整数变量的index。$A$ 和 $Aeq$ 两个是矩阵,
$b$ 与 $beq$ 两个都是矢量,这四个变量用于集中所有的等式约束和不等式约束。
$lb$ 和 $ub$ 都是矢量,用于规定 $x$ 的 lower bound and upper bound.
目标函数中的$f$是一个矢量,包含预定义的任意值。

于是,使用Matlab的编程模型需要先将我们的ILP问题整理成上述统一形式,然后
使用
\begin{center}
\verb|intlinprog(f,intcon,A,b,Aeq,beq,lb,ub)|
\end{center}
这个MILP接口求解。详细文档matlab文档已经写的非常清楚。

\subsection{编程}

规模为3的问题有42个0-1变量(包括中间变量)。
我们严格规定在在这42个位置中,
$x_i$ occupies $1:3$,
$c_{ij}$ ... $4:12$,
$a_{ij}$ ... $13:21$,
$b_{ij}$ ... $22:30$,
$d_{i}$ ... $31:33$,
$e_{ij}$ ... $34:42$.
这个位置关系我们在后续的代码中使用
\begin{center}
\verb@x 1:3 | c 4:12 | a 13:21 | b 22:30 | d 31:33 | e 34:42@
\end{center}
这个记号来表示。

\begin{itemize}
%@objective(m, Max, sum(c .* w))
\item 准备$f$。因为我们目标函数完全不关心中间变量,所以
		只需要$w_{ij}$填入对应于$c_{ij}$位置的部分即可。注意,我们取负号因为编程模型最小化,我们要最大化。\\
\verb|f = -[0 0 0  1 1 1  1 1 1  1 1 1  zeros(1,42-12) ]'|
\item 将所有变量指定为整数变量\\
\verb|intcon = [1:42]|
\item 随便给定的 $q_{ij}$, 张量 $q$ 尺寸是$3\times 3\times 3$,因为
	$q_{ij}$ 是一个集合。$q_{ij}$ 的三个布尔值决定 $m$ 取1,2,3中的哪个。
\begin{verbatim}
q = reshape([1 0 0 0 1 0 0 0 1
	0 1 1 1 1 0 1 0 1
	0 0 1 0 0 0 1 0 0 ], 3, 3, 3)
\end{verbatim}
\item 进一步限制整数变量为0-1变量\\
\verb|lb = zeros(42, 1)|\\
\verb|ub = ones(42, 1)|

\item 初始化空矩阵和空向量用来存放海量约束\\
\begin{verbatim}
A = []
b = []
Aeq = []
beq = []
\end{verbatim}

\item 根据 $q_{ij}$ 的值来完成公式(11),
%# aij >= xm
%for i in 1:II
%	for j in 1:J
%		for k in 1:II
%			if p[i,j,k] > 0
%				@constraint(m, a[i,j] >= x[k])
%			end
%		end
%	end
%end
\begin{Verbatim}
% x 1:3 | c 4:12 | a 13:21 | b 22:30 | d 31:33 | e 34:42
% a[i,j] >= x[k]
for i = 1:3
	for j = 1:3
		for k = 1:3
			if q(i,j,k) ~= 0
				tmp = zeros(42, 1)
				tmp(13+(i-1)*3+(j-1)) = -1
				tmp(k) = 1
				A = [A tmp]
				b = [b; 0]
			end
		end
	end
end
\end{Verbatim}

\item 公式(13)
%# bij <= cij
%for i in 1:II
%	for j in 1:J
%		@constraint(m, b[i,j] <= c[i,j])
%	end
%end
\begin{Verbatim}
% x 1:3 | c 4:12 | a 13:21 | b 22:30 | d 31:33 | e 34:42
% b[i,j] <= c[i,j]
for i = 1:3
	for j = 1:3
		tmp = zeros(42, 1)
		tmp(22+(i-1)*3+(j-1)) = 1
		tmp(4+(i-1)*3+(j-1)) = -1
		A = [A tmp]
		b = [b; 0]
	end
end
\end{Verbatim}

\item 公式(14)
%# bij <= \bar{aij}
%for i in 1:II
%	for j in 1:J
%		@constraint(m, b[i,j] <= 1 - a[i,j])
%	end
%end
\begin{Verbatim}
% x 1:3 | c 4:12 | a 13:21 | b 22:30 | d 31:33 | e 34:42
%b[i,j] <= 1 - a[i,j]
for i = 1:3
	for j = 1:3
		tmp = zeros(42, 1)
		tmp(22+(i-1)*3+(j-1)) = 1
		tmp(13+(i-1)*3+(j-1)) = 1
		A = [A tmp]
		b = [b; 1]
	end
end
\end{Verbatim}

\item 公式(15)
%# bij >= cij + \bar{aij} - 1
%for i in 1:II
%	for j in 1:J
%		@constraint(m, b[i,j] >= c[i,j] + (1 - a[i,j]) - 1)
%		# 0 >= c[i,j] - a[i,j] - b[i,j]
%	end
%end
\begin{Verbatim}
% x 1:3 | c 4:12 | a 13:21 | b 22:30 | d 31:33 | e 34:42
% b[i,j] >= c[i,j] + (1 - a[i,j]) - 1
% --> 0 >= c[i,j] - a[i,j] - b[i,j]
for i = 1:3
	for j = 1:3
		tmp = zeros(42, 1)
		tmp(4+(i-1)*3+(j-1)) = 1
		tmp(13+(i-1)*3+(j-1)) = -1
		tmp(22+(i-1)*3+(j-1)) = -1
		A = [A tmp]
		b = [b; 0]
	end
end
\end{Verbatim}

\item 公式(12). 注意这是等式约束,要并入 $Aeq$.
%# bij + \bar{cij} = 1
%for i in 1:II
%	for j in 1:J
%		@constraint(m, b[i,j] + (1 - c[i,j]) == 1)
%       # b[i,j] - c[i,j] == 0
%	end
%end
\begin{Verbatim}
% x 1:3 | c 4:12 | a 13:21 | b 22:30 | d 31:33 | e 34:42
% b[i,j] + (1 - c[i,j]) == 1
% --> b[i,j] - c[i,j] == 0
for i = 1:3
	for j = 1:3
		tmp = zeros(42, 1)
		tmp(22+(i-1)*3+(j-1)) = 1
		tmp(4+(i-1)*3+(j-1)) = -1
		Aeq = [Aeq tmp]
		beq = [beq 0]
	end
end
\end{Verbatim}

\item 公式(21)
%# di <= \bar{xi}
%for i in 1:II
%	@constraint(m, d[i] <= 1 - x[i])
%end
\begin{Verbatim}
% x 1:3 | c 4:12 | a 13:21 | b 22:30 | d 31:33 | e 34:42
% d[i] <= 1 - x[i]
for i = 1:3
	tmp = zeros(42, 1)
	tmp(31+(i-1)) = 1
	tmp(1+(i-1)) = 1
	A = [A tmp]
	b = [b; 1]
end
\end{Verbatim}

\item 公式(22)
%# di <= \bar{cij}
%for i in 1:II
%	for j in 1:J
%		@constraint(m, d[i] <= 1 - c[i,j])
%	end
%end
\begin{Verbatim}
% x 1:3 | c 4:12 | a 13:21 | b 22:30 | d 31:33 | e 34:42
% d[i] <= 1 - c[i,j]
for i = 1:3
	for j = 1:3
		tmp = zeros(42, 1)
		tmp(31+(i-1)) = 1
		tmp(4+(i-1)*3+(j-1)) = 1
		A = [A tmp]
		b = [b; 1]
	end
end
\end{Verbatim}

\item 公式(23)
%# di >= \bar{xi} + sum_j(\bar{cij}) - J
%for i in 1:II
%	@constraint(m, d[i] >= (1 - x[i]) + sum(1 .- c[i,:]) - J)
%                  0 >= -d[i] + 1 - x[i] + I - c[i,:] - J
%                  -1 >= -d[i] -x[i] -c[i,:]
%end
\begin{Verbatim}
% x 1:3 | c 4:12 | a 13:21 | b 22:30 | d 31:33 | e 34:42
% d[i] >= (1 - x[i]) + sum(1 .- c[i,:]) - J
% --> 0 >= -d[i] + 1 - x[i] + J - c[i,:] - J
% --> -1 >= -d[i] -x[i] -c[i,:]
for i = 1:3
	tmp = zeros(42, 1)
	tmp(31+(i-1)) = -1
	tmp(1+(i-1)) = -1
	tmp(4+(i-1)*3:4+(i-1)*3+2) = -1
	A = [A tmp]
	b = [b; -1]
end
\end{Verbatim}

\item 公式(24)
%# eij <= xi
%for i in 1:II
%	for j in 1:J
%		@constraint(m, e[i,j] <= x[i])
%	end
%end
\begin{Verbatim}
% x 1:3 | c 4:12 | a 13:21 | b 22:30 | d 31:33 | e 34:42
% e[i,j] <= x[i]
for i = 1:3
	for j = 1:3
		tmp = zeros(42, 1)
		tmp(1+(i-1)) = -1
		tmp(34+(i-1)*3+(j-1)) = 1
		A = [A tmp]
		b = [b; 0]
	end
end
\end{Verbatim}

\item 公式(25)
%# eij <= cij
%for i in 1:II
%	for j in 1:J
%		@constraint(m, e[i,j] <= c[i,j])
%	end
%end
\begin{Verbatim}
% x 1:3 | c 4:12 | a 13:21 | b 22:30 | d 31:33 | e 34:42
% e[i,j] <= c[i,j]
for i = 1:3
	for j = 1:3
		tmp = zeros(42, 1)
		tmp(34+(i-1)*3+(j-1)) = 1
		tmp(4+(i-1)*3+(j-1)) = -1
		A = [A tmp]
		b = [b; 0]
	end
end
\end{Verbatim}

\item 公式(26)
%# eij <= cik (k neq j)
%for i in 1:II
%	for j in 1:J
%		for k in 1:J
%			if j != k
%				@constraint(m, e[i,j] <= 1 - c[i,k])
%			end
%		end
%	end
%end
\begin{Verbatim}
% x 1:3 | c 4:12 | a 13:21 | b 22:30 | d 31:33 | e 34:42
% e[i,j] <= 1 - c[i,k]
for i = 1:3
	for j = 1:3
		for k = 1:3
			tmp = zeros(42, 1)
			tmp(34+(i-1)*3+(j-1)) = 1
			tmp(4+(i-1)*3+(j-1)) = 1
			A = [A tmp]
			b = [b; 1]
		end
	end
end
\end{Verbatim}

\item 公式(27)
%# eij >= xi + cij + sum(\bar{cik})
%tmp = 1 .- I(3)
%for i in 1:II
%	for j in 1:J
%		@constraint(m, e[i,j] >= x[i] + c[i,j] + sum(tmp[i,:]' * (1 .- c[i,:])))
% 0 >= -eij +cij + sum(tmp[i,:]' * (1-ci:))
% 2 >= -eij +cij - sum(ci: excl i==j)
%	end
%end
\begin{Verbatim}
% x 1:3 | c 4:12 | a 13:21 | b 22:30 | d 31:33 | e 34:42
% e[i,j] >= x[i] + c[i,j] + sum(ones' * (1-cik)) - J
% --> 0 >= -eij +xi +cij + sum(ones' * (1-cik)) - J
% --> -(J-1) + J >= -eij +xi +cij - sum(cik)
% --> 1 >= -eij +xi +cij - sum(cik)
for i = 1:3
	for j = 1:3
		tmp = zeros(42, 1)
		tmp(34+(i-1)*3+(j-1)) = -1
		tmp(4+(i-1)*3:4+(i-1)*3+2) = -1   % cik
		tmp(4+(i-1)*3+(j-1)) = 1  % cij
		tmp(1+(i-1)) = 1
		A = [A tmp]
		b = [b; 1]
	end
end
\end{Verbatim}

\item 公式(20)
%# sum_j(eij) + d_i = 1
%for i in 1:II
%	@constraint(m, sum(e[i,:]) + d[i] == 1)
%end
\begin{Verbatim}
% x 1:3 | c 4:12 | a 13:21 | b 22:30 | d 31:33 | e 34:42
% sum(e[i,:]) + d[i] == 1
for i = 1:3
	tmp = zeros(42, 1)
	tmp(31+(i-1)) = 1
	tmp(34+(i-1):34+(i-1)+2) = 1
	Aeq = [Aeq tmp]
	beq = [beq 1]
end
\end{Verbatim}

\item 调用MILP Solver,矩阵需要翻转一下才能正确点积。
\begin{Verbatim}
intlinprog(f, intcon, A', b, Aeq', beq, lb, ub)
\end{Verbatim}

\end{itemize}

\subsection{完整代码}

\begin{Verbatim}
f = -[0 0 0  1 1 1  1 1 1  1 1 1  zeros(1,42-12) ]'

intcon = [1:42]

q = reshape([1 0 0 0 1 0 0 0 1
	0 1 1 1 1 0 1 0 1
	0 0 1 0 0 0 1 0 0 ], 3, 3, 3)

lb = zeros(42, 1)
ub = ones(42, 1)

A = []
b = []
Aeq = []
beq = []

% x 1:3 | c 4:12 | a 13:21 | b 22:30 | d 31:33 | e 34:42
% a[i,j] >= x[k]
for i = 1:3
	for j = 1:3
		for k = 1:3
			if q(i,j,k) != 0
				tmp = zeros(42, 1)
				tmp(13+(i-1)*3+(j-1)) = -1
				tmp(k) = 1
				A = [A tmp]
				b = [b; 0]
			end
		end
	end
end

% x 1:3 | c 4:12 | a 13:21 | b 22:30 | d 31:33 | e 34:42
% b[i,j] <= c[i,j]
for i = 1:3
	for j = 1:3
		tmp = zeros(42, 1)
		tmp(22+(i-1)*3+(j-1)) = 1
		tmp(4+(i-1)*3+(j-1)) = -1
		A = [A tmp]
		b = [b; 0]
	end
end

% x 1:3 | c 4:12 | a 13:21 | b 22:30 | d 31:33 | e 34:42
%b[i,j] <= 1 - a[i,j]
for i = 1:3
	for j = 1:3
		tmp = zeros(42, 1)
		tmp(22+(i-1)*3+(j-1)) = 1
		tmp(13+(i-1)*3+(j-1)) = 1
		A = [A tmp]
		b = [b; 1]
	end
end

% x 1:3 | c 4:12 | a 13:21 | b 22:30 | d 31:33 | e 34:42
% b[i,j] >= c[i,j] + (1 - a[i,j]) - 1
% --> 0 >= c[i,j] - a[i,j] - b[i,j]
for i = 1:3
	for j = 1:3
		tmp = zeros(42, 1)
		tmp(4+(i-1)*3+(j-1)) = 1
		tmp(13+(i-1)*3+(j-1)) = -1
		tmp(22+(i-1)*3+(j-1)) = -1
		A = [A tmp]
		b = [b; 0]
	end
end

% x 1:3 | c 4:12 | a 13:21 | b 22:30 | d 31:33 | e 34:42
% b[i,j] + (1 - c[i,j]) == 1
% --> b[i,j] - c[i,j] == 0
for i = 1:3
	for j = 1:3
		tmp = zeros(42, 1)
		tmp(22+(i-1)*3+(j-1)) = 1
		tmp(4+(i-1)*3+(j-1)) = -1
		Aeq = [Aeq tmp]
		beq = [beq 0]
	end
end

% x 1:3 | c 4:12 | a 13:21 | b 22:30 | d 31:33 | e 34:42
% d[i] <= 1 - x[i]
for i = 1:3
	tmp = zeros(42, 1)
	tmp(31+(i-1)) = 1
	tmp(1+(i-1)) = 1
	A = [A tmp]
	b = [b; 1]
end

% x 1:3 | c 4:12 | a 13:21 | b 22:30 | d 31:33 | e 34:42
% d[i] <= 1 - c[i,j]
for i = 1:3
	for j = 1:3
		tmp = zeros(42, 1)
		tmp(31+(i-1)) = 1
		tmp(4+(i-1)*3+(j-1)) = 1
		A = [A tmp]
		b = [b; 1]
	end
end

% x 1:3 | c 4:12 | a 13:21 | b 22:30 | d 31:33 | e 34:42
% d[i] >= (1 - x[i]) + sum(1 .- c[i,:]) - J
% --> 0 >= -d[i] + 1 - x[i] + J - c[i,:] - J
% --> -1 >= -d[i] -x[i] -c[i,:]
for i = 1:3
	tmp = zeros(42, 1)
	tmp(31+(i-1)) = -1
	tmp(1+(i-1)) = -1
	tmp(4+(i-1)*3:4+(i-1)*3+2) = -1
	A = [A tmp]
	b = [b; -1]
end

% x 1:3 | c 4:12 | a 13:21 | b 22:30 | d 31:33 | e 34:42
% e[i,j] <= x[i]
for i = 1:3
	for j = 1:3
		tmp = zeros(42, 1)
		tmp(1+(i-1)) = -1
		tmp(34+(i-1)*3+(j-1)) = 1
		A = [A tmp]
		b = [b; 0]
	end
end

% x 1:3 | c 4:12 | a 13:21 | b 22:30 | d 31:33 | e 34:42
% e[i,j] <= c[i,j]
for i = 1:3
	for j = 1:3
		tmp = zeros(42, 1)
		tmp(34+(i-1)*3+(j-1)) = 1
		tmp(4+(i-1)*3+(j-1)) = -1
		A = [A tmp]
		b = [b; 0]
	end
end

% x 1:3 | c 4:12 | a 13:21 | b 22:30 | d 31:33 | e 34:42
% e[i,j] <= 1 - c[i,k]
for i = 1:3
	for j = 1:3
		for k = 1:3
			tmp = zeros(42, 1)
			tmp(34+(i-1)*3+(j-1)) = 1
			tmp(4+(i-1)*3+(j-1)) = 1
			A = [A tmp]
			b = [b; 1]
		end
	end
end

% x 1:3 | c 4:12 | a 13:21 | b 22:30 | d 31:33 | e 34:42
% e[i,j] >= x[i] + c[i,j] + sum(ones' * (1-cik)) - J
% --> 0 >= -eij +xi +cij + sum(ones' * (1-cik)) - J
% --> -(J-1) + J >= -eij +xi +cij - sum(cik)
% --> 1 >= -eij +xi +cij - sum(cik)
for i = 1:3
	for j = 1:3
		tmp = zeros(42, 1)
		tmp(34+(i-1)*3+(j-1)) = -1
		tmp(4+(i-1)*3:4+(i-1)*3+2) = -1   % cik
		tmp(4+(i-1)*3+(j-1)) = 1  % cij
		tmp(1+(i-1)) = 1
		A = [A tmp]
		b = [b; 1]
	end
end

% x 1:3 | c 4:12 | a 13:21 | b 22:30 | d 31:33 | e 34:42
% sum(e[i,:]) + d[i] == 1
for i = 1:3
	tmp = zeros(42, 1)
	tmp(31+(i-1)) = 1
	tmp(34+(i-1):34+(i-1)+2) = 1
	Aeq = [Aeq tmp]
	beq = [beq 1]
end
intlinprog(f, intcon, A', b, Aeq', beq, lb, ub)

\end{Verbatim}

\end{CJK}
\end{document}
