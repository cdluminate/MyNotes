% app-crypt
\section{Applied Cryptography}

Encryption can be described in math as follows
$$ E(M) = C $$
and similarly Decryption can also be written as
$$ D(C) = M $$
And this equation must hold for any $E(\cdot)$ and $D(\cdot)$:
$$ D(E(M)) = M $$
Apartfrom encryption and decryption, Cryptography is also used for
authentication, integrity and nonrepudiation.

Modern cryptography introduced key $K$ to enhance cipher (cryptographic algorithm) security,
{\it i.e.}
\begin{align*}
E_K(M) &= C \\
D_K(C) &= M \\
D_K(E_K(M)) &= M
\end{align*}
which is also called symmetric algorithm.
Some algorithms uses different keys for encryption and decryption,
{\it i.e.}
\begin{align*}
E_{K_1}(M) &= C \\
D_{K_2}(C) &= M \\
D_{K_2}(E_{K_1}(M)) &= M
\end{align*}
which is called public-key algorithm.

The most common algorithm is the following three,
(1) data encryption standard, DES, symmetric; (2) RSA, public-key; (3) digital signature algorithm, DSA, sign only.

Unix program \verb|rot13| is a simple example of substitution cipher, but it's not a key point here.

%\subsection{Protocols}
%Advanced protocol: zero-knowledge proof.

\subsection{Mathematical Background}

\subsubsection{Information Theory}

\begin{description}
\item[Entropy and Uncertainty]
 The amount of information of a message $M$ can be represented by its entropy $H(M)$.
 Generally the entropy of a message $M$ is $\log_2 n$ where $n$ is the number of all
 possible values of $M$. Here we suppose the possibility distribution is uniform
 across all possible cases. The entropy of a message also reveals its uncertainty.
\item[Rate of language]
 The rate of a given language is $r = H(M)/N$ where $N$ is the length of message.
 The absolute rate is $R = \log_2 L$ where $L$ is the number of available alphabets
 in the given language, and the redundancy $D = R - r$.
\item[Security of Cipher System]
 The entropy of a given cipher system can be represented by the key space $H(K)=\log_2 K$.
 The larger the entropy is, the more difficult the cipher system be cracked.
\end{description}

\subsubsection{Pseudo-random Sequence Generator}
Linear congruential generator
$$ X_n = (aX_{n-1}+b)mod m $$
Here are some good numbers for the generator use $(a,b,m) \in \{(106,1283,6075), (84589,45989,217728)\}$.

\subsection{Technique}

\subsubsection{Birthday Attack against Single-direction Hash Function}
\begin{description}
\item[Classic Brute Force Method] Select a message $M$, and then find another message $M'$ so that
  $H(M') = H(M)$ holds.
\item[Birthday Attack] Randomly select a pair of messages $M$ and $M'$ so that $H(M') = H(M)$ holds.
  As we know, you need to add $253$ people into the room to find a person whose birthday is the same
  as yours, however only $23$ people are needed if you only want to find a pair in the same birthday in the room.
\end{description}
Suppose that the output length of hash function $H$ is $m$-bit, then
method 1 requires $2^m$ times of computation, however $2^{\frac{m}{2}}$ times for birthday attack.

Birthday Paradox\footnote{\tt{https://en.wikipedia.org/wiki/Birthday\_problem}}
$$ P(\overline A) = \frac{365}{365} \times \frac{364}{365} \times \frac{363}{365} \times \ldots \frac{343}{365} \approx 0.49 $$
therefore
$$ P(A) = 1 - P(\overline A) \approx 0.51 $$

\subsubsection{Ciphertext in Ciphertext}

$$ P\oplus K = C$$
$$ C\oplus K = P$$
$$ K' = C \oplus P' $$
$$ C\oplus K' = P' $$
where $C$ is ciphertext, $P, P'$ are real and fake original plaintexts respectively, $K, K'$ are real and fake keys.

\subsection{RSA}

Select two very big prime numbers $p$ and $q$, then calculate their product
$$ n = pq$$
Next, randomly select a ecryption key $e$ which is prime to $(p-1)(q-1)$. Finally
calculate decryption key $d$ so this equation holds
$$ed \equiv 1 \text{ mod }(p-1)(q-1)$$
{\it i.e.}
$$d = e^{-1} 1 \text{ mod }(p-1)(q-1)$$
Here $e$ and $n$ is public key, and $d$ is private key. The two prime numbers $p$ and $q$ should
be discard but are still secret. Before encryption, the message $m$ should be diveded into
many groups where the group size is less than $n$. The resulting ciphertext $c$ is formed
of $c_i$,
$$ c_i = m_i^e (\text{ mod }n) $$
On decryption, this operation is performed
$$ m_j = c_i^d (\text{ mod }n)  $$

For example, select $p=47$ and $q=71$, then $n=pq=3337$. Encryption key $e$ must be prime to
$(p-1)(q-1)=3220$, so we randomly select $e=79$, then $d=79^{-1} \text{ mod }3220 = 1019 $. To encrypt
message $m=688$, we obtain $688^{79} \text{ mod }3337 = 1570 = c$. To decrypt, we get
$1570^{1019} (\text{ mod } 3337) = 688 = m$.

\subsection{Reference}

1. Bruce Schneier, {\it Applied Cryptography, Second Edition}, John Wiley \& Sons.
