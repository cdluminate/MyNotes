\section{Probablistic Graph Model}

Probablistic graph models fall into two categories: Bayesian network
(using directed acyclic graph) and Markov network (using undirected
graph).

\subsection{Hidden Markov Model (HMM)}

HMM is the simplest dynamic Bayesian network. There are two groups of variables
in the HMM model: (1) hidden variable (i.e. state variable) $y_i\in Y$ (state
space); (2) observation variable $x_i \in X$ (observation space, often
discrete).

\begin{verbatim}
y_1 -> y_2 -> ... -> y_i -> ... -> y_n
 |      |             |             |
 v      v             v             v
x_1    x_2           x_i           x_n
\end{verbatim}

The joint probability distribution of all variables is
$$ P(x_1,y_1,\ldots,x_n,y_n) = P(y_1)P(x_1|y_1)
\prod_{i=2}^n P(y_i|y_{i-1})P(x_i|y_i) $$

An HMM is determined by state space $Y$, observation space $X$ and three
parameters: (1) state transfer probability matrix $A_{ij} =
P(y_{t+1}=s_j|y_t=s_i)$; (2) observation output probability matrix $B_{ij} =
P(x_t=o_j|y_t=s_i)$; (3) initial state probability $\pi = (\pi_1, \pi2, \ldots,
\pi_N)$.

In real life applications of HMM, people often focus on three fundamental
problems: (1) how well does the model match with the observed sequence?
(2) how to infer the hidden states according to observed sequences?
(3) how to train the model such the model fits better with data?

\subsection{Markov Random Field (MRF)}

MRF, a typical Markov network, is a hind of undirected graph model.

\subsection{Conditional Random Field (CRF)}

CRF, is a kind of discriminant undirected graph model. Note that HMM and MRF
are both generative models, which models the joint probability distribution.
However CRF models the conditional distribution.

\subsection{Reference}

1. Zhihua Zhou, Machine Learning.
