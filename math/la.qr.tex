\verb|Sat Jun 23 14:51:28 UTC 2018|

\section{QR Decomposition}
\url{https://en.wikipedia.org/wiki/QR_decomposition}

QR factorization is to decomposite an invetible matrix $A$ into
the product of an orthogonal matrix $Q$ and an invertible upper
triangular matrix $R$, i.e. $A=QR$.

There are not only one way to conduct QR factorization on a given
matrix. The methods for conducting QR factorization include
Schmidt orthogonalization, Givens transformation, and Householder
transformation.

\subsection{QR Decomp by Schmidt Orthogonalization}

Let $a_\ast$ be the columns of the given matrix $A$. We take the
first column $a_1$ as the first new column vector $b_1$ of the
orthogonal matrix. For the second column of the orthogonal matrix,
we want to remove the $b_1$ subcomponent from $a_2$, such that
$b_2$ is orthogonal to $b_1$, i.e.
$$b_2 = a_2 - k_{21}b_1, b_1\cdot b_2=0$$
By solving the above equation we get $k_{21} = \frac{b_1a_1}{b_1b_1}$.
Similarly for the third column in the orthogonal matrix we want
to remove the $b_1$ and $b_2$ subcomponent from $a_3$, i.e.
$$b_3 = a_3 - k_{32}b_2 - k_{31} b_1, b_3\cdot b_2 = 0, b_3 \cdot b_1 = 0$$
where $k_{ij} = \frac{b_j\cdot a_i}{b_j\cdot b_j}$.

Finally we'll get a system like this:
$$
\begin{aligned}
	b_1 &= a_1 \\
	b_2 &= a_2 - k_{21}b_1 \\
	\ldots \\
	b_n &= a_n - k_{n,n-1}b_{n-1} - \ldots - k_{n1}b_1
\end{aligned}
\Rightarrow
\begin{aligned}
	a_1 &= b_1 \\
	a_2 &= b_2 + k_{21}b_1 \\
	\ldots \\
	a_n &= b_n + k_{n,n-1}b_{n-1} + \ldots + k_{n1}b_1
\end{aligned}
$$
Or equivalently in matrix form:
$$
\begin{bmatrix}
	a_1 & \cdots & a_n
\end{bmatrix}
=
\begin{bmatrix}
	b_1 & \cdots & b_n
\end{bmatrix}
\begin{bmatrix}
	1 & k21 & \cdots & kn1 \\
	  & 1   & \ddots  & \vdots \\
	  &     &  1 & k_{n,n-1} \\
	  & & & 1
\end{bmatrix}
$$

Next, $Q$ can be obtained by unitizing/normalizing the columns in
matrix $B$, i.e. $q_i = \frac{b_i}{||b_i||}$. And
$$
[a_1 \ldots a_n] = [q_1 \ldots q_n] \text{Diag}(||b_1||, ||b_2||, \ldots,
  ||b_n||)
  \begin{bmatrix}
	  1 & \cdots & k_{\ast\ast}\\
	    & \ddots & \vdots \\
		& & 1
  \end{bmatrix}
$$
which is exactly the decomposition result $A=QR$.

\subsection{QR by Givens Transformations}

Recall that a rotating a 2-D vector looks like
$$
y = 
\begin{bmatrix}
	\cos \theta & \sin \theta \\
	-\sin \theta & \cos \theta
\end{bmatrix}
x
= Tx
$$
where $\det T = 1$.

Givens transformation is a generalized form of such rotation where
two dimensions are involved. In brief the Givens matrix looks like
$$
\begin{bmatrix}
	I& & & & \\
	 &c& &s& \\
	 & &I& & \\
	 &-s& &c& \\
	 & & & &I
\end{bmatrix}
\begin{matrix}
	\\
	(i)\\
	\\
	(j)\\
	\\
\end{matrix}
$$
where $c^2 + s^2 = 1$. This represents a rotation transformation in
the plane expanded by $\hat{e}_i$ and $\hat{e}_j$. Determinant of any 
Givens matrix is 1, and the Givens matrix itself is orthogonal.

A given vector can be transformed so that it has the same direction with
a base vector. We first find a rotation matrix, so that the first column of $A$
can be rotated to the same direction as that of $\hat{e}_1$.
$$ T_1 a_1 = |a_1| \hat{e}_1$$
Then we finish the first step by applying transformation $T_1$ to A.
$$ T_1 A[1:n,1:n] = A_1 $$
Next we find a rotation for the submatrix of $A_1$ starting from the second
row and the second column, such that
$$ T_2 A_1[2:n,2] = |A_1[2:n,2]| \hat{e}_2$$
$$T_2 A_1[2:n,2:n] = A_2$$
Until the last step which looks like
$$ T_{n-1} A_{n-1}[n-1:n,n] = |\ldots|\hat{e}_n$$
$$ T_{n-1} A_{n-2}[n-1:n,n-1:n] = A_{n-1} $$
which indicates that
$$
TA = A_{n-1} = \begin{bmatrix} I_{n-2} & \\ & T_{n-1} \end{bmatrix}
	\cdots \begin{bmatrix} I_2 & \\ & T_3 \end{bmatrix}
		\begin{bmatrix} I_1 & \\ & T_2 \end{bmatrix}
			T_1 = R
$$
where $I_\ast$ means eye matrix of size $\ast$, R is a upper triangular.
We can find out the QR factorization if we examine the above result further.
$$ TA = \tilde{T}_{n-1} \tilde{T}_{n-2} \ldots \tilde{T}_1 A = R$$
$$ A = T^{-1}R = T^T R = QR $$
where $Q = T^{-1} = T^T$ because $T$ is a product of a series of orthogonal
matrices.

\subsection{QR by Householder Transformations}

Householder transformation mirrors the vector along one dimension. For instance
$$
y = [x_1, x_2]^T = [1, 0; 0 -1][x_1, x_2]^T = (I-2\hat{e}_2\hat{e}_2^T)x = Hx
$$
Assume that $u\in R^n$ is a unit column vector, $H=I-2uu^T$ is a householder matrix.
Householder matrices have several properties: (1) $H^T=H$, (2) $H^TH=I$,
(3) $H^2=I$, (4) $H^{-1}=H$, (5) $\det H = -1$.

Any non-zero vector $x$ can be transformed into the same direction as that of
a base vector, i.e. $Hx = |x|e_\ast$. When $x=|x|e_\ast$, we select a unit column
vector so that $u^Tx=0$. When $x\ne |x|e_\ast$, we use
$$ u = \frac{x-|x|e_\ast}{|x-|x|e_\ast|} $$

To conduct QR factorization with householder transformations, we similarly
mirror the first column vector of the submatrices so that they point to the
same direction as that of the corresponding base vector. This procedure is similar
to QR factorization with Givens transformation.
