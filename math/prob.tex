% Pandoc LaTeX
% todo: parameter estimation
\title{Probability and Statistics}

\section{Basic Concepts about Probability}

[\href{https://en.wikipedia.org/wiki/Probability_theory}{Probability Theory}]

Let $A$, $B$, $\ldots$ denote
[\href{https://en.wikipedia.org/wiki/Event_(probability_theory)}{events}],
then the following laws are often used when conducting operations with events.

\begin{itemize}
	\item Commutative law\\
$$ A\cup B=B\cup A, A\cap B=B\cap A $$
	\item Associative law\\
$$ A\cup(B\cup C)=(A\cup B)\cup C, A\cap(B\cap C)=(A\cap B)\cap C $$
	\item Distributive law\\
$$ A\cup(B\cap C)=(A\cup B)\cap(A\cup C), A\cap(B\cup C)=(A\cap B)\cup(A\cap C) $$
	\item DE Morgan law\\
$$ \overline{A\cup B}=\overline{A}\cap\overline{B}, \overline{A\cap B}=\overline{A}\cup\overline{B} $$
\end{itemize}

[\href{https://en.wikipedia.org/wiki/Probability_axioms}{Probability Axioms}]

The possibility of an impossible event is zero, {\it i.e.}
$ P(\Phi) = 0 $.

The possibility of events is finitely summable, {\it i.e.}
$$ P(A_1 \cup A_2 \cup \ldots \cup A_n) = 
   P(A_1) + P(A_2) + \ldots + P(A_n) $$

if $A \subset B$ then $P(B-A) = P(B)-P(A); P(B)\geq P(A)$.

$\forall A, P(A) \leqslant 1$

Possibility of a reciprocal event $\forall A, P(\overline{A}) = 1-P(A)$

$P(A\cup B) = P(A) + P(B) - P(AB)$

\subsection{3 Important Equations}

[\href{https://en.wikipedia.org/wiki/Conditional_probability}{Conditional Probability}]:
$A$ and $B$ are events and $P(A)>0$, then conditional probability can be
written as follows,
$$ P(B|A) = \frac{P(AB)}{P(A)} $$
This is the generalized form of the conditional probability formula
\begin{align}
	& P(A_1A_2\ldots A_n)\\
	=& P(A_n|A_1A_2\ldots A_{n-1})P(A_{n-1}|A_1A_2\ldots A_{n-2})\ldots P(A_2|A_1)P(A_1)
\end{align}

[\href{https://en.wikipedia.org/wiki/Law_of_total_probability}{Law of Total Probability}]
$$ P(A) = P(A|B_1)P(B_1) + P(A|B_2)P(B_2) + \ldots + P(A|B_n)P(B_n) $$

[\href{https://en.wikipedia.org/wiki/Bayes\%27_theorem}{Bayes' Theorem}]
Experiment $E$ is in a sample space $S$, $B_i, i\in\{1,2,\ldots,n\}$ is
a split of $S$, and $P(A)>0$, $P(B_i)>0$, then
$$ P(B_i|A) = \frac{P(A|B_i)P(B_i)}{\sum\limits_{j=1}^n P(A|B_j)P(B_j)} , i \in \{1,2,\ldots, n\}$$
This equation can be derived from the conditional probability and total
probability.

\subsection{Independence}

[\href{https://en.wikipedia.org/wiki/Independence_(probability_theory)}{Independence}]

Independent events: if $P(AB) = P(A)P(B)$ holds.

\section{Random Variable and its Distribution}

\subsection{Discrete Random Variable and Common Distributions}

Assume the sample space of a random experiment is $S = \{e\}$, then
$X=X(e)$ is real-and-single-valued function in sample space $S$.
And $X=X(e)$ is called random variable.

For a discrete random variable $X$ whose possible values are
$x_k (k=1,2,\ldots)$, the possibility that the variable $X$ takes
a specific values is $ P\{X=x_k\} = p_k, ~~ k = 1, 2, \ldots$.

There are three important kinds of discrete random variable:
\begin{itemize}
\item $(0-1)$ Distribution\newline
  $$P\{X=k\} = p^k(1-p)^{1-k} , k \in \{ 0, 1 \}, (0<p<1) $$
\item Binomial Distribution $X\sim b(n,p)$\newline
  $P(A) = p, P(\overline A) = 1-p$, $X$ is the counter of event $A$,
		$$P\{X=k\} = \big(_k^n\big) p^k (1-p)^{n-k} , k = 0,1,\ldots, n$$
\item Posion Distribution $X\sim \pi(\lambda)$\newline
  $$ P\{X=k\} = \frac{\lambda^k e^{-\lambda}}{k!}, k = 0,1,2,\ldots. $$
  $$ \sum_{k=0}^\infty P\{X=k\} = \sum_{k=0}^\infty \frac{\lambda^k e^{-\lambda}}{k!} 
     = e^{-\lambda} \sum_{k=0}^\infty \frac{\lambda^k}{k!} = e^{-\lambda} \cdot e^\lambda = 1$$
\end{itemize}

[\href{https://en.wikipedia.org/wiki/Poisson_limit_theorem}{Poisson Limit Theorem}]

\subsection{Continuous Random Variable and Common Distributions}

[\href{https://en.wikipedia.org/wiki/Probability_distribution}{Probability Distribution}]

Distribution function of a (non-discrete) random variable
$$ F(x) = P\{X\leq x\}, -\infty < x < \infty$$
\begin{align}
	 & P\{x_1<X\leq x_2\} = P\{X\leq x_2\} - P\{X\leq x_1\}\\
	=& F(x_2)-F(x_1) = \int_{x_1}^{x_2} f(x) dx
\end{align}

[\href{https://en.wikipedia.org/wiki/Probability_density_function}{Probability
Density Function (PDF)}] is for continuous random variable.
$$ F(x) = \int_{-\infty}^x f(t)dt $$
where $f(x)$ is called possibility density of $X$.

There are three important types of continuous random variables:
\begin{itemize}
\item Uniform Distribution $X\sim U(a,b)$\newline
 $$ f(x) = \{a<x<b\} \frac{1}{b-a} $$
\item Exponential Distribution\newline
 $$ f(x) = \{x>0\} \frac{1}{\theta} e^{-x/\theta} $$
Exponential distribution has an interesting property,
$$P\{X>s+t|X>s\} = P\{X>t\}, ~~\forall s, t > 0$$
\item Gauss/Normal Distribution $X\sim N(\mu,\sigma^2)$\newline
 $$ f(x) = \frac{1}{\sqrt{2\pi}\sigma} e^{-\frac{(x-\mu)^2}{2\sigma^2}}
		 , -\infty<x<\infty $$
 Particularly when $\mu = 0, \sigma = 1$ it turns into a Standard Gauss Distribution.
 $$ X\sim N(\mu,\sigma^2) \Rightarrow Z = \frac{X-\mu}{\sigma}\sim N(0,1) $$
% Multi-Dimentional Random Variable and its Distribution, see Wikipedia
\end{itemize}

Given that random variable $X\sim N(\mu, \sigma^2)$, try to prove that a
linear function of $X$, $Y=aX+b, ~(a\neq 0)$, obeys a Gaussian distribution too.

\subsection{Distribution of Random-Variable-Argument Function}

\url{https://en.wikipedia.org/wiki/Probability_density_function#Dependent_variables_and_change_of_variables}

The PDF of random variable $X$ is $f_X(x), -\infty<x<\infty$. Function
$g(x)$ is derivable everywhere and $g'(x) >0$ (or $g'(x)<0$) always holds.
Then $Y = g(X)$ is continuous random variable, whose PDF is
$$ f_Y(y) = \{\alpha<y<\beta\} f_X(g^{-1}(y))|h'(y)| $$
where $a = \min(g(-\infty),g(\infty))$, $b= \max(g(-\infty),g(\infty))$.
Because when $\alpha < y < \beta$,
\begin{align}
	& F_Y = P\{Y\leq y\} = P\{g(X)\leq y\}\\
	=& P\{X\leq g^{-1}(y)\} = F_X[g^{-1}(y)]
\end{align}
Now $f_Y$ can be obtained by deriving $F_Y$.

\section{Multi-dimensional Random Variable and its Distributions}

Assume that $(X,Y)$ is a $2$-dimensional random variable. The distribution
function of $(X,Y)$, namely the joint distribution function of $X$ and $Y$ is
$$F(x,y) = P\{(X\leq x)\cap (Y\leq y)\} = P\{X\leq x, Y\leq y\}$$
For discrete $2$-D random variables, $P\{X=x_i,Y=y_i\}=p_{ij},~i,j=1,2,\ldots$
is the distribution law of $(X,Y)$, or the joint distribution law of $X$
and $Y$.

The derivative of the joint distribution function is the joint probability
density function,
$$ \frac{\partial^2 F(x,y)}{\partial x\partial y} = f(x,y) $$
as long as $f(x,y)$ is continuous at point $(x,y)$. It is also noted that
$$\int_{-\infty}^\infty \int_{-\infty}^\infty f(x,y)dxdy=F(\infty, \infty)=1$$

When extended to the case of $n$-dimensional random variable, the joint
distribution function is as follows
$$ F(x_1,x_2,\ldots,x_n) = P\{X_1\leq x_1, X_2\leq x_2, \ldots X_n\leq x_n\} $$

\subsection{Marginal Distribution}

[\href{https://en.wikipedia.org/wiki/Marginal_distribution}{Marginal distribution}]
function could be determined by the distribution function $F(x,y)$ of variable
$(X,Y)$ as follows,
$$ F_X(x) = P\{X\leq x\} = P\{X\leq x, Y\leq \infty\} = F(x,\infty)$$
For discrete random variable, the marginal distribution law is
$$ F_X(x) = F(x,\infty) = \sum_{x_i\leq x} \sum_{j=1}^\infty p_{ij}$$

Hence the marginal probability density function is
$$ f_X(x) = \int_{-\infty}^\infty f(x,y) dy $$

\subsection{Conditional Distribution}

$$ P\{X=x_i|Y=y_j\} = \frac{P\{X=x_i,Y=y_j\}}{P\{Y=y_j\}} = \frac{p_{ij}}{p_{*j}}$$
is called the conditional distribution law of variable $X$ under the condition
where $Y=y_j$.

Accordingly, the conditional probability density function could be defined as
$$f_{X|Y}(x|y) = \frac{f(x,y)}{f_Y(y)}$$

\subsection{Independent Random Variable}

Random variable $X$ and $Y$ are independent to each other as long as the
following condition holds,
$$ F(x,y) = P\{X\leq x, Y\leq y\} = P\{X\leq x\}P\{Y\leq y\} = F_X(x)F_Y(y)$$
which is equivalent to the following equation,
$$ f(x,y) = f_X(x) f_Y(y) $$

If $(X_1,X_2,\ldots,X_n)$ and $(Y_1,Y_2,\ldots,Y_n)$ are independent to each
other, i.e. $X_i$ and $Y_i$ are independent, then $h(X_1,X_2,\ldots,X_n)$ and
$g(Y_1,Y_2,\ldots,Y_n)$ are independent each other, where $h,g$ are continuous
functions.

\subsection{Distribution of Two-variable Function}

Assume that random variable $X$ and $Y$ are independent to each other.

1. $Z=X+Y$
\begin{quote}
	$$ f_{X+Y}(z) = \int_{-\infty}^\infty f(z-y, y)dy $$
	When $X$ and $Y$ are independent to each other, the above equation
	can be transformed into
	$$ f_{X+Y}(z) = \int_{-\infty}^\infty f_X(z-y)f_Y(y)dy$$
	which is called the convolution formula, symbolized as $f_X*f_Y$.
	$$ f_X*f_Y = \int_{-\infty}^\infty f_X(z-y)f_Y(y)dy
	           = \int_{-\infty}^\infty f_X(x)f_Y(z-x)dx$$
\end{quote}
2. $Z=Y/X$
\begin{quote}
	$$ f_{Y/X}(z) = \int_{-\infty}^\infty |x|f(x,xz)dx $$
\end{quote}
3. $Z=XY$
\begin{quote}
	$$ f_{Y/X}(z) = \int_{-\infty}^\infty \frac{1}{|x|}f(x,\frac{z}{x})dx $$
\end{quote}
4. $M=\max(X,Y)$
\begin{quote}
	$$ F_{\max}(z) = P(M\leq z) = P(X\leq z, Y\leq z) = P(X\leq z)P(Y\leq z)$$
	$$ F_{\max}(z) = F_{X_1}(z)F_{X_2}(z)\cdots F_{X_n}(z) $$
\end{quote}
5. $N=\min(X,Y)$
\begin{quote}
	$$ F_{\min}(z) = P(N\leq z) = 1-P(N>z) = 1-P(X>z,Y>z) = 1-P(X>z)P(Y>z)$$
	$$ F_{\min}(z) = 1-[1-F_{X_1}(z)][1-F_{X_2}(z)]\cdots [1-F_{X_n}(z)] $$
\end{quote}

%%%%%%%%%%%%%%%%%%%%%%%%%%%%%%%%%%%%%%%%%%%%%%%%%%%%%%%%%%%%%%%%%

\section{Numerical Characteristics}

\subsection{Mathematical Expectation}

Assume that a discrete random variable $X$ has such a distribution
$P\{X=x_k\}=p_k,~~k=1,2,\ldots$. Then the following series, as long
as it converges, is called the mathematical expectation of variable $X$.
$$ E(X) = \sum\limits_{k=1}^\infty x_k p_k $$

Similarly, for continuous random variable $X$ whose probability density
is $f(x)$, its mathematical expectation is as follows as long as it
converges.
$$ E(X) = \int^\infty_{-\infty} xf(x)dx $$

Here are some properties of mathematical expectation: $E(C)=C$,
$E(CX)=CE(X)$, $E(X+Y)=E(X)+E(Y)$, $E(XY)=E(X)E(Y) ~~ (X,Y independent)$.

$$E[g(X)] = \sum_{k=1}^\infty g(x_k)p_k$$
$$E[g(X)] = \int_{-\infty}^\infty g(x)f(x)dx$$
$$E[g(X,Y)] = \sum_{i=1}^\infty \sum_{j=1}^\infty g(x_i,y_j)p_{ij}$$
$$E[g(X,Y)] = \int_{-\infty}^\infty\int_{-\infty}^\infty g(x,y)f(x,y)dxdy$$

\subsection{Variance}
The variance of discrete random variable $X$ is defined as follows,
as long as it exists.
$$ D(X)=Var(X)=E\{ [X-E(X)]^2 \} = E(X^2)-[E(X)]^2 $$
Similarly, for continuous random variable the variance is
$$ D(X) = \int^\infty_{-\infty} [X-E(X)]^2 f(x)dx $$
Accordingly, the standard deviation is $\delta(X)=\sqrt{D(X)}$.

Some properties of variance: $D(C)=0$, $D(CX)=C^2 D(X)$,
$D(X+C)=D(X)$, $D(X+Y)=D(X)+D(Y)+Cov(X,Y)$.

\subsection{Coviriance and Correlation Coefficient}
 Coviriance
 $$ Cov(X,Y) = E\{ [X-E(X)][Y-E(Y)] \} = E(XY) - E(X)E(Y)$$
 correlation coefficient
 $$ \rho_{XY} = \frac{Cov(X,Y)}{ \sqrt{D(X)} \sqrt{D(Y)} } $$

 $Cov(aX,bY) = ab~ Cov(X,Y)$

 $Cov(X+Z, Y) = Cov(X,Y) + Cov(Z,Y)$

\subsection{Moment and Covariance Matrix}

 $k$-th (origin) moment $E(X^k)$. $k$-th central moment $E\{[X-E(X)]^k\}$.
 $k+l$-th mixed moment $E(X^kY^l)$.
 $k+l$-th central mixed moment $E\{[X-E(X)]^k[Y-E(Y)]^l\}$.

 Obviously, mathematical expectation $E(X)$ is $X$'s first-order origin
 moment. Variance $D(X)$ is the second-order central moment of $X$.
 $Cov(X,Y)$ is the second-order central mixed moment of $X$ and $Y$.

 Covariance matrix $C_{ij} = Cov(X_i,X_j)$ is a symmetric matrix because
 $c_{ij} = c_{ji}$.

 Question: probability density function of $n$-D Gaussian random vector.

\section{Law of Large Numbers and Central Limit Theorem}

 \subsection{Chebyshev Inequility}
 var $X$, $E(X)=\mu$, $D(X)=\delta^2$, $\forall \varepsilon >0$,

 $$ P \big\{ \big| X - \mu \big| \geqslant \varepsilon \big\} 
   \leqslant \frac{\delta^2}{\varepsilon^2} $$

 \begin{align*}
	 P \big\{ \big| X - \mu \big| \geqslant \varepsilon \big\} &=
		\int_{|X-\mu|\geq \varepsilon} f(x)dx \\
		&\leq \int_{|X-\mu|\geq \varepsilon} \frac{|x-\mu|^2}{\varepsilon^2}
			f(x)dx \\
		&\leq \frac{1}{\varepsilon^2} \int^\infty_{-\infty} (x-\mu)^2 f(x)dx\\
		&= \frac{\delta^2}{\varepsilon^2}
 \end{align*}

 \subsection{Weak Law of Large Numbers}

[\href{https://en.wikipedia.org/wiki/Law_of_large_numbers}{Law of Large Numbers}]

 Assume that random variables $X_1, X_2, \ldots$ are independent to each
 other, and have the same distribution. $E(X_k)=\mu,~k=1,2,\ldots$. Then
 $\forall \varepsilon>0$, the following equation holds,
 $$ \lim\limits_{n\rightarrow\infty} \big\{ \big|
		 \frac{1}{n} \sum\limits_{k=1}^n X_k - \mu
     \big| < \varepsilon \big\} = 1 $$

 When $D(X_k)=\delta^2,~k=1,2,\ldots$ exists,
 $$ E(\frac{1}{n} \sum\limits_{k=1}^n X_k) =
	\frac{1}{n} \sum\limits_{k=1}^n E(X_k) = \frac{n\mu}{n} = \mu$$
 $$ D(\frac{1}{n} \sum\limits_{k=1}^n X_k) =
	\frac{1}{n^2} \sum\limits_{k=1}^n D(X_k) = \frac{n\delta^2}{n^2}
	= \frac{\delta^2}{n} $$
 We plug them in the Chebyshev inequility, and obtain the following equation,
 $$ 1 \geqslant P \big\{ \big| \frac{1}{n} \sum\limits_{k=1}^n X_k - \mu
    \big| < \varepsilon \big\} \geqslant 1 - \frac{\delta^2/n}{\varepsilon^2} $$
 where $ 1 - \frac{\delta^2/n}{\varepsilon^2} $ tends to $1$ when
 $n \rightarrow \infty$.

 See also Bernoulli Law of Large Numbers.

\subsection{Central Limit Theorem}
See also Lyapunouv Theorem, De Moivre-Laplace Theorem.

\section{Sample and Sampling Distribution}

Statistics, based on the probability theory.

[\href{https://en.wikipedia.org/wiki/Random_sample}{Random Sample}]

[\href{https://en.wikipedia.org/wiki/Simple_random_sample}{Simple random sample}]

Wiki(Histogram,BoxPlot)

\subsection{Sampling Distribution}

Sample Mean
$$ \bar{X} = \frac{1}{n} \sum_{i=1}^n X_i $$
Sample Variance
$$ S^2 = \frac{1}{n-1} \sum_{i=1}^n(X_i - \bar{X})^2 $$
etc.

\section{Parameter Estimation}

\subsection{Point Estimation}

Estimate an unkown parameter with a sample from the sample population.

\subsubsection{Method of Moments}

[\href{https://en.wikipedia.org/wiki/Method_of_moments_(statistics)}{Method of Moments}]

Assume that $\theta_1,\theta_2,\ldots,\theta_n$ are the parameters to be
estimated. Given a continuous random variable $X$ with probability
density function $f(x;\theta_1,\theta_2,\ldots,\theta_n)$, or a discrete
random variable $X$ with distribution law
$P(X=x)=p(x;\theta_1,\theta_2,\ldots,\theta_n)$
where $X_1,X_2,\ldots,X_n$ are samples from $X$. Assume that the first
$k$-th moment of $X$ exists:
$$ \mu_l = E(X^l) = \int_{-\infty}^\infty
   x^l f(x;\theta_1,\theta_2,\ldots,\theta_n) dx, ~~l=1,2,\ldots,k $$
or
$$ \mu_l = E(X^l) = \sum_{x\in R_X}
   x^l p(x;\theta_1,\theta_2,\ldots,\theta_n), ~~l=1,2,\ldots,k $$
As we know the sample moment $A_l = \frac{1}{n} \sum_{i=1}^n X_i^l$
converges to the overall moment $\mu_l$ at a certain probability.
\begin{align}
	\mu_1 &= \mu_1(\theta_1,\theta_2,\ldots,\theta_k)\\
	\mu_2 &= \mu_2(\theta_1,\theta_2,\ldots,\theta_k)\\
	& \ldots\\
	\mu_k &= \mu_k(\theta_1,\theta_2,\ldots,\theta_k)
\end{align}
\begin{align}
	\theta_1 &= \theta_1(\mu_1,\mu_2,\ldots,\mu_k)\\
	\theta_2 &= \theta_2(\mu_1,\mu_2,\ldots,\mu_k)\\
	& \ldots\\
	\theta_k &= \theta_k(\mu_1,\mu_2,\ldots,\mu_k)\\
\end{align}
$$ \hat{\theta}_i = \theta_i(A_1, A_2, \ldots, A_k),~~i=1,2,\ldots,k $$

Example 1. $X\sim U(a,b)$ where a and b are unknowns.
$\mu_1 = E(X) = (a+b)/2$, $\mu_2 = E(X^2) = D(X) + [E(X)]^2$.
By solving the system we can obtain the solution of a and b.
$a = \mu_1 - \sqrt{3(\mu_2-\mu_1^2)}$,
$b = \mu_1 + \sqrt{3(\mu_2-\mu_1^2)}$.
If we replace $\mu_1,\mu_2$ with $A_1,A_2$, we get the moment estimators
$\hat{a}$ and $\hat{b}$.
\begin{quote}
	\begin{verbatim}
# Julia 0.6.2
x = rand(100000) # ~U(0,1)
sm1, sm2 = mean(x.^1), mean(x.^2)
hata = sm1 - sqrt(3(sm2-sm1^2)) # approx 0
hatb = sm1 + sqrt(3(sm2-sm1^2)) # approx 1
	\end{verbatim}
\end{quote}

Example 2. Mean $\mu$ and variance $\sigma^2$ unknown.
$\mu_1 = \mu$, $\mu_2 = \sigma^2 + \mu^2$.
$\hat{\mu}, \hat{\sigma^2} = ?$

\subsubsection{Maximum Likelihood Estimation}

[\href{https://en.wikipedia.org/wiki/Maximum_likelihood_estimation}{Maximum Likelihood Estimation}]

Likelihood function for discrete random variable $X$ with parameter $\theta$:
$$ L(\theta) = L(x_1,x_2,\ldots,x_n;\theta) = \prod_{i=1}^n P(X_i=x_i;\theta), ~\theta\in\Theta$$
And for continuous random variable $X$ with parameter $\theta$:
$$ L(\theta) = L(x_1,x_2,\ldots,x_n;\theta) = \prod_{i=1}^n f(x_i;\theta), ~\theta\in\Theta$$
$$ L(x_1,x_2,\ldots,x_n;\hat{\theta}) = \max_{\theta\in\Theta} L(x_1,x_2,\ldots,x_n;\theta) $$
$$ \hat{\theta} = \text{arg}\max_{\theta\in\Theta} L(x_1,x_2,\ldots,x_n;\theta) $$
$$ \frac{d}{d\theta} L(\theta) = 0 $$
$$ \frac{d}{d\theta} \ln L(\theta) = 0 $$

Example 1. $X\sim b(1,p)$, $x_i\in {0,1}$. $L(p) = \prod_{i=1}^n p^{x_i} (1-p)^{1-x_i}$,
$\ln L(p) = \big(\sum_{i=1}^n x_i\big) \ln p + \big(n-\sum_{i=1}^n x_i\big) \ln (1-p)$.
$\frac{d}{dp} \ln L(p) = 0 \Rightarrow \hat{p} = \frac{1}{n} \sum_{i=1}^n x_i$.

Example 2. $X\sim U(a,b)$. $L(a,b) = \{a\leq x_*\leq b\}\frac{1}{(b-a)^n}$.
$L(a,b) = \{a\leq x_{\min}, x_{\max} \leq b\}\frac{1}{(b-a)^n}$.
$L(a,b) = \frac{1}{(b-a)^n} \leqslant \frac{1}{(x_{\max} - x_{\min})^n}$.
$\hat{a},\hat{b} = \text{argmax}~L(a,b) = x_{\min}, x_{\max}$.

Example 3. Event A happened $n_1$ times while B happened $n_2$ times.
$L(\theta)=\theta^{n_1} (1-\theta)^{n_2}$. $\frac{\partial \ln L(\theta)}{\partial \theta}
= \frac{n_1}{\theta} + \frac{n_2}{\theta-1}=0 \Rightarrow \theta = \frac{n_1}{n_1+n_2}$.

When analytical solution is not available, numerical methods, such as the
Newton-Raphson method, could be leveraged to obtain an approximate solution. 

\subsection{Estimator Assessment}

\subsubsection{Unbiased Estimator}

We call $\hat{\theta}$ an unbiased estimator of $\theta$ when
$E(\hat{\theta}) = \theta$. Actually, a parameter may have various estimators.

Example 1. $A_k = \frac{1}{n} \sum_{i=1}^n X_i^k$ is an unbiased extimator
of $k$-th population moment $\mu_k$.
$E(A_k) = \frac{1}{n} \sum_{i=1}^n E(X_i^k) = \mu_k$.

\subsubsection{Validity}

$\hat{\theta}_1$ is more effective than $\hat{\theta}_2$ when
$D(\hat{\theta}_1) \leqslant D(\hat{\theta}_2)$.

\subsubsection{Congruence}

...

\section{Hypothesis Testing}

\section{Stochastic Process}

\section{Markov Chain}

In a stochastic process, given that the ``current'' state is known,
its ``future'' state doesn't depend on the ``past'' state. This is
Markov property. Moreover, when the time and status of a stochastic
process are both discrete, and the process satisfies Markov property,
then it's called Markov chain. Namely,
\begin{align*}
     & P\{ X_{m+n}=a_j | X_{t_1}=a_{i_1}, X_{t_2}=a_{i_2}, \ldots, X_{m}=a_{i} \} \\
	 &= P\{ X_{m+n}=a_j | X_{m}=a_{i} \}\\
	 &= P_{ij}(m,m+n)
\end{align*}
where $P_{ij}(m,m+n)$ denotes the transition probability from state
$a_i$ at time $m$ to state $a_j$ at time $m+n$. Obviously it must
transit to an existing status, so the summary of the probability of
all possibilities should equal one, {\it i.e.}
$$ \sum\limits_{j=1}^{+\infty} P_{ij}(m,m+n) = 1 $$

See \verb|stochproc.tex| for detailed discussion.

\section{Reference}

% 1. 概率论与数理统计,浙江大学,第四版

2. \url{http://cs229.stanford.edu/section/cs229-prob.pdf}
