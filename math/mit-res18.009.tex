\title{MIT RES18.009: Learn Differential Equations, Gilbert Strang}

\href{https://ocw.mit.edu/resources/res-18-009-learn-differential-equations-up-close-with-gilbert-strang-and-cleve-moler-fall-2015/}{Link: MIT-RES-18.009}

\section{Introduction}

\subsection{Overview of differential equation}

First order equations
$$\frac{dy}{dt} = ay + q(t)$$
$$\frac{dy}{dt} = f(y)$$

Second order equations
$$\frac{d^2y}{dt^2} = -ky$$
$$my''+by'+ky = f(x)$$

It is good if these equations are linear and have constant coefficients.
However the wide part of the subject ends up as numerical solutions.

Systems of n equations
$$\frac{d\vec{y}}{dt} = A\vec{y}$$
$$\frac{d^2\vec{y}}{dt^2} = -S\vec{y}$$
We can leverage eigenvalues and eigenvectors, to convert these coupled
equations into n uncoupled equations, so that we can solve them individually.
Besides, we can also obtain the numerical solutions by using MATLAB ode45.

Partial Differential Equation (PDE). For example the heat equation,
$$\frac{\partial u}{\partial t} = \frac{\partial^2 u}{\partial x^2}$$
the wave equation,
$$\frac{\partial^2 u}{\partial t^2} = \frac{\partial^2 u}{\partial x^2}$$
and the Laplace equation
$$\frac{\partial^2 u}{\partial x^2} + \frac{\partial^2 u}{\partial y^2} = 0$$

\subsection{The calculous you need}

Derivatives of $x^n$, $\sin x$, $\cos x$, $e^x$, $\ln x$.

Rules for $f+g$, $f(x)g(x)$, $f(x)/g(x)$, $f(g(x))$.

Fundamental Theorem
$$\frac{d}{dx} \int_0^x y(t)dt = y(x)$$

For example,
$$y(t) = \int_0^t e^{t-s} q(s) ds$$
solves
$$\frac{dy}{dt} = y + q(t)$$

Tangent line to the graph,
$$f(t+\Delta t) \approx f(t) + \Delta t \frac{df(t)}{dt}$$
This equation can be extended into Taylor Series.

\section{First Order Equations}

\subsection{Response to Exponential Input}
