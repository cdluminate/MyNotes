% math-fund.tex

\subsection{Functions}

 A functions $f$ is a mappings between two non-empty sets:
 $$f: A \mapsto B$$
 When $f(-x) = f(x)$, the function is even; When $f(-x) = -f(x)$,
 the function is odd. See
 \href{https://en.wikipedia.org/wiki/Function_(mathematics)}
 {Wikipedia:Function} for more detail about function.
 Here is a list of some elementary functions and their properties:

 \begin{itemize}

 \item {$y = kx + b$}
 \item {$y = \frac{k}{x},~~x\neq 0$}
 \item {$ y = ax^2+bx+c $}
 \item {$ y = a^x $}\\
  $$a^{\frac{m}{n}} = \sqrt[n]{a^m}$$
  $$a^{-1} = \frac{1}{a}$$
 \item {$ y = \log_a x $}
  $$a^x = N ~~ \Rightarrow ~~ x = \log_a N$$
  $$\log_a MN = \log_a M + \log_a N$$
  $$\log_a \frac{M}{N} = \log_a M - \log_a N$$
  $$\log_a M^n = n \log_a M, n \in \mathcal{R}$$
  \[ \log_a b = \frac{\log_c b}{\log_c a}\]
 \item
  $$ \text{sh} x = \frac{e^x - e^{-x}}{2} $$
  $$ \text{ch} x = \frac{e^x + e^{-x}}{2} $$
  $$ \text{tanh} x = \frac{\text{sh} x}{\text{ch} x} =
     \frac{e^x - e^{-x}}{e^x + e^{-x}} $$
 \end{itemize}

\subsection{Trigonometry}

\begin{figure}[!h]
 \centering
 \setlength{\unitlength}{0.8cm}
 \begin{picture}(3,3)
 \thicklines
 %\put(0,0){\vector(1,0){5}}
 %\put(0,0){\vector(0,1){5}}
 \put(1,0.5){\line(2,1){3}}
 \put(4,2){\line(-2,1){2}}
 \put(2,3){\line(-2,-5){1}}
 \put(0.7,0.3){$A$}
 \put(4.05,1.9){$B$}
 \put(1.7,2.95){$C$}
 \put(3.1,2.5){$a$}
 \put(1.3,1.7){$b$}
 \put(2.5,1.05){$c$}
 \end{picture}
 \label{pic:triangle}
 \caption{Triangle}
\end{figure}

\begin{table}[!h]
 \begin{center}
 \begin{tabular}{|c|c|}
 \hline
 Symbol & Meaning \\
 \hline\hline
 $a,b,c$ & Lengths of triangle's sides\\
 $A,B,C$ & Angles opposite to sides $a,b,c$ respectively\\
 \hline
 \end{tabular}
 \end{center}
\end{table}

 See \href{https://en.wikipedia.org/wiki/Trigonometry}
 {Wikipedia:Trogonometry} for more detail.

\begin{itemize}
\item{Triangle}\\
 For any triangle, the following equations hold:
 \[ a + b > c \]
 \[ A + B + C = \pi \]
\item{Law of Sines}\\
  \begin{equation}
  \frac{a}{\sin A} = \frac{b}{\sin B} = \frac{c}{\sin C} 
  \end{equation}
\item{Law of Cosines}\\
  \begin{equation}
  a^2 = b^2 + c^2 - 2 b c \cos A
  \end{equation}
\item{Area of Triangle}\\
  \begin{equation}
  A_\triangle = \frac{1}{2} bc \sin A
  \end{equation}
  \begin{equation}
  A_\triangle = \sqrt{p(p-a)(p-b)(p-c)}
  \end{equation}
  where
  \[ p = \frac{a+b+c}{2} \]
\item{Archway length and Sectorial Area}\\
  \begin{equation}
  l = \theta r
  \end{equation}
  \begin{equation}
  A = \frac{1}{2} l r
  \end{equation}
\item{Period}\\
  \[ f(x) = A \sin (\omega x + \varphi) \]
  \[ T = \frac{2\pi}{\omega} \]
\item{Trigonometrical Function Convertion}\\
  \[ \tan \theta = \frac{\sin \theta}{\cos \theta}\]
  \[ \sin^2 \theta + \cos^2 \theta = 1 \]
  \[ \sin (\frac{\pi}{2} - \theta) = \cos \theta \]
  \[ \cos (a+b) = \cos a \cos b - \sin a \sin b \]
  \[ \sin (a+b) = \sin a \cos b + \cos a \sin b \]
  \[ \tan (a+b) = \frac{\tan a + \tan b}{1 - \tan a \tan b} \]

 These tricks are sometimes useful for such convertions:
  \[ a  = (a+b) - b \]
  \[ a = (a+b)/2 + (a-b)/2 \]

 The following equations can be derived from the equations above:
  \[ \sin (a\pm b) = \sin a \cos b \pm \cos a \sin b \]
  \[ \cos (a\pm b) = \cos a \cos b \mp \sin a \sin b \]
  \[ \sin 2\theta = 2 \sin a \cos a \]
  \[ \cos 2\theta = \cos^2 \theta + \sin^2 \theta = 2\cos^2 \theta -1 \]
  \[ \sin a + \sin b = 2 \sin \frac{a+b}{2} \cos \frac{a-b}{2} \]
  \[ \cos a + \cos b = 2 \cos \frac{a+b}{2} \cos \frac{a-b}{2} \]
  \[ \sin a \cos b = \frac{1}{2} [ \sin (a+b) + \sin (a-b) ] \]
  \[ \cos a \cos b = \frac{1}{2} [ \cos (a+b) - \cos (a-b) ] \]
  \[ \sin \theta =
	\frac{2 \tan \frac{\theta}{2}}{1+\tan^2 \frac{\theta}{2}} \]
  \[ \cos \theta =
	\frac{1-\tan^2 \frac{\theta}{2}}{1+\tan^2 \frac{\theta}{2}} \]
  \[ \tan \theta =
	\frac{2 \tan \frac{\theta}{2}}{1-\tan^2 \frac{\theta}{2}} \]

  And the following one has migic power \ldots
  \[ a \sin \alpha + b \cos \alpha =
	  \sqrt{a^2+b^2} \sin (\alpha + \varphi) \]
  where $0 \leqslant \varphi < 2\pi$, and 
  \[ \sin \varphi = \frac{b}{\sqrt{a^2+b^2}} \]
  \[ \cos \varphi = \frac{a}{\sqrt{a^2+b^2}} \]

  However this one is even more powerful \ldots
  \[ A \cos (\omega t + \varphi_1) + B \cos(\omega t + \varphi_2)
	  = C \cos (\omega t + \varphi_3) \]
  where
  \[ C = \sqrt{A^2 + B^2 - 2AB \cos(\varphi_2 - \varphi_1)} \]
  and \[ \varphi_3 =
	  \arctan \Big[ \frac{A\sin\varphi_1+B\sin\varphi_2}
	  {A\cos\varphi_1+B\cos\varphi_2} \Big]\]

\item{Frequently used Values}\\
\begin{table}[!h]
  \begin{center}
  \begin{tabular}{|c|ccccc|}
  \hline
  Angle $\theta$ & $0$ & $\pi/6$ & $\pi/4$ & $\pi/3$ & $\pi/2$\\
  \hline\hline
  $\sin \theta$  & 0 & 1/2 & $\sqrt{2}$/2 & $\sqrt{3}$/2 & 1 \\
  $\cos \theta$  & 1 & $\sqrt{3}$/2 & $\sqrt{2}$/2 & 1/2 & 0 \\
  $\tan \theta$  & 0 & $\sqrt{3}$/3 & 1 & $\sqrt{3}$   & NaN \\
  \hline
  \end{tabular}
  \caption{Frequently Used Trigonometric Function Values}
  \end{center}
\end{table}

\end{itemize}

\subsection{Geometry}

 \begin{itemize}
 \item{Line}\\
  \[ y = kx + b\]
  \[ y - y_0 = k(x - x_0) \]
  \[ \frac{y-y_1}{y_2-y_1} = \frac{x-x_1}{x_2-x_1}\]
  \[ \frac{x}{a} + \frac{y}{b} = 1 \]
  \[ Ax + By + C = 0\]
 
  Distance from point $P(x_0, y_0)$ to line $ax + by +c = 0$:
  \[ \text{Dist} = \frac{|ax_0 + by_0 +c|}{\sqrt{a^2+b^2}} \]
 \item{Circle}\\
  \[ (x-a)^2 + (y-b)^2 = r^2\]
 \item{Ellipse}\\
  \[ \frac{x^2}{a^2} + \frac{y^2}{b^2} = 1\]
  $M(x, y)$ on ellipse, $F_1, F_2$ are its focuses, $|F_1M|+|MF_2|= 2a$
 \item{Hyperbola}\\
  \[ \frac{x^2}{a^2} - \frac{y^2}{b^2} = 1\]
 \item{Parabola}\\
  \[ y^2 = 2px\]
 \item perimeter circle $c = 2 \pi r$ 
 \item area circle $A = \pi r^2$
 \item surface sphere $S = 4 \pi r^2$
 \item volume circular cone $V = \frac{1}{3} \pi r^2 h$
 \item volume sphere $V = \frac{4}{3} \pi r^3$
 \item cosine distance
  \[ \langle\vec{a},\vec{b}\rangle =
     \frac{\vec{a}\cdot\vec{b}}{|\vec{a}||\vec{b}|} \]
 \end{itemize}

\subsection{Sequence of Numbers}

 Symbol ${a_n}$ is used to denote a sequence of numbers,
 i.e. $a_1, a_2, \ldots, a_n$. The partial fraction expansion
 trick may sometimes help you to get the summary of a sequence.

 \begin{itemize}
 \item This is useful when only the summary of a sequence is given.
  \[ a_n = \sum_{i=1}^n a_i - \sum_{i=1}^{n-1} a_i \]

 \item{Arithmatic Progression}

  \[ a_n = a_1 + (n-1) d \]
  \begin{equation}
  \sum_{i=1}^n a_i = \frac{n(a_1+a_n)}{2}
  \label{eq:sumofseq_samediff}
  \end{equation}

  One of its properties:
  \[ a_{n} + a_{n-2} = 2 a_{n-1} \]

  Proof for Eq.~\ref{eq:sumofseq_samediff}:
  \begin{eqnarray}
                S_n & = & a_1 + a_2 + \ldots + a_n \\
                S_n & = & a_n + a_{n-1} + \ldots + a_1 \\
  \Rightarrow 2 S_n & = & (a_1 + a_n) + \ldots + (a_n + a_1)
  \end{eqnarray}
  Note, this may not work for infinite series, since
  the commutative law and the associative law may not
  work for infinite series.
 
 \item{Geometric Progression}

 \[ a_n = a_1 c^{n-1} \]
 where $c$ is the ratio, a constant not equal to $0$.
 \begin{equation}
 \sum_{i=1}^n a_n = \frac{a_1 (1-c^n)}{1-c} ~~~~(c\neq1)
 \label{eq:sumofseq_samerat}
 \end{equation}

 Proof for Eq.~\ref{eq:sumofseq_samerat}:
 \begin{eqnarray}
                        S_n & = & a_1 + a_2 + \ldots + a_n \\
                c \cdot S_n & = & a_2 + a_3 + \ldots + a_{n+1} \\
 \Rightarrow (c-1)\cdot S_n & = & a_{n+1} - a_1
 \end{eqnarray}

 \item{Special Sequences}

 \[ \sum^n_{i=1} i = \frac{1}{2} n(n+1) \]
 \[ \sum^n_{i=1} i^2 = \frac{1}{6} n(n+1)(2n+1) \]
 \[ \sum^{n-1}_{i=1} i^2 = \frac{1}{6} n(n-1)(2n-1) \]
 \[ \sum^n_{i=1} i^3 = \frac{1}{4} n^2(n+1)^2 \]

 This is the general form for finding the summary of $i^p$:
 \[ \sum_{k=1}^n k^p = \sum_{k=1}^p \Big( \sum_{j=0}^{k-1} (-1)^j C_k^j (k-j)^{p+1} \Big) C_{n+1}^{k+1} \]

 \end{itemize}

\subsection{Inequality}

 \begin{itemize}

 \item{Fundamental Inequibility}

 \[ \frac{a+b}{2} \geqslant \sqrt{ab} \]
 the $=$ holds only when $a=b$. Additionally,
 \[ \frac{a+b+c}{3} \geqslant \sqrt[3]{abc} \]
 the $=$ holds only when $a=b=c$.

 \[ \frac{2}{\frac{1}{a}+\frac{1}{b}} \leqslant \sqrt{ab} \leqslant
    \frac{a+b}{2} \leqslant \sqrt{\frac{a^2+b^2}{2}} ~~,
		 a>0, ~b>0 \]

 \item{Trigonometrical Ineq.}

 \[ |a+b| \leqslant |a| + |b| \]
 The "$=$" holds only when $ab\geq0$.
 \[ |a-c| \leqslant |a-b| + |b-c| \]
 The "$=$" holds only when $(a-b)(b-c)\geq0$.

 Proof: using vectors.

 \item{Cauthy Inequality}

 \[ (a^2+b^2)(c^2+d^2) \geqslant (ac + bd)^2 \]
 The "$=$" holds only when $ad=bc$.

 Proof:
 \[ |\vec{a}\cdot\vec{b}| \leqslant |\vec{a}|\cdot|\vec{b}| \]

 General form of Cauthy Ineq. :
 \[ (a_1^2+a_2^2+\ldots+a_n^2)(b_1^2+b_2^2+\ldots+b_n^2) \geqslant (a_1b_1 + a_2b_2 +\ldots+ a_nb_n)^2 \]
 the "$=$" sign holds only when $b_i=0$ or $\exists~ k \in \mathcal{R}, a_i = kb_i$
 See \href{https://en.wikipedia.org/wiki/Cauchy-Schwarz_inequality}
 {Wikipedia:Cauthy Inequality} for detail.

 \item{Sequence Inequality}

 Given that $a_i \in \mathcal{R}, b_i \in \mathcal{R}$,
 and $a_1\leq a_2\leq \cdots a_n$ and $b_1\leq b_2\leq \cdots b_n$,
 and $c_1,c_2,\ldots,c_n$ is one of the permutations of $b_1,b_2,\ldots,b_n$, then
 \[ a_1b_n+a_2b_{n-1}+\ldots+a_nb_1 \leq a_1c_1 + a_2c_2 + \ldots + a_nc_n \]
 \[ a_1c_1 + a_2c_2 + \ldots + a_nc_n \leq a_1b_1 + a_2b_2 + \ldots + a_nb_n \]

 \end{itemize}

\subsection{Permutation \& Combination}

 \[ a^3 + b^3 = (a+b)(a^2-ab+b^2) \]
 \[ a^3 - b^3 = (a-b)(a^2+ab+b^2) \]
 
\begin{lstlisting}[language=Mathematica]
factor(a^3+b^3); factor(a^3-b^3);
\end{lstlisting}

 \[ A_n^m = \frac{n!}{(n-m)!}\]

 \[ C_n^m = \frac{n!}{m!(n-m)!}\]

 Binomial theorem, where $n$ is integer,
 \[ (a+b)^n = \sum_{k=0}^n C_n^k a^{n-k}b^k \]
 See \href{https://en.wikipedia.org/wiki/Binomial_theorem}
 {Wikipedia:Binomial Theorem} for generalized binomial theorem.

\subsection{Complex Number}
 A complex number $z\in C$ could be written as $z = a+ib$ where $i$ is the
 imaginary unit, $a,b\in\Re$. If we inspect this number regarding it as
 a vector in the geometry aspect, then its length is $|z| = r = \sqrt{a^2+b^2}$,
 and its angle to the $x$-axis is $\tan\theta=\frac{b}{a}$. Since $a=r\cos\theta$
 and $b=r\sin\theta$, we could write the number in another form,
 $z = r(\cos\theta + i\sin\theta) = re^{i\theta}$.

\subsection{Fundamental Physics Bits}

\subsubsection{Linear Movement}

$$ \frac{dv}{dt} = a $$
$$ \frac{dx}{dt} = v $$

where $v$ denotes velocity, $a$ denotes acceleration (e.g. $g$),
$t$ for time. The following equations can be derived from the
above equations

$$ v = v_0 + at $$
$$ x = v_0 t + \frac{1}{2}at^2 + x_0 $$

\subsubsection{Newton's laws of motion}

$$ F = ma $$

\subsubsection{Curvilinear motion}

$$ v = \frac{2\pi r}{T} $$
$$ a_n = \frac{v^2}{r} = \omega^2 \cdot r = \omega \cdot v $$
$$ F_n = m \cdot a_n $$

\subsubsection{The law of attraction}

$$ F = G\frac{m_1 m_2}{r^2} $$
where $G$ is the gravitation constant.

\subsubsection{Mechanical Energy}

$$ W=Fl $$
$$ P=Fv $$
$$ E_{kinetic} = \frac{1}{2}mv^2 $$
$$ E_{potential1} + E_{kinetic1} = E_{potential2}+E_{kinetic2} $$

\subsubsection{Static electric field}

$$ F = k\frac{q_1 q_2}{r^2} $$
$$ E = F/q $$

\subsubsection{Constant current}

$$ I = q/t $$
$$ I = U/R $$
$$ R = \rho l/S $$
$$ P = UI $$
$$ W = Pt $$

\subsubsection{Magnetic Field}

$$ F=BIL $$ (B is vertical to I)
$$ F=qvB $$

\subsubsection{Electromagnetic Induction}

$$ \Phi = BS $$
$$ E = n\frac{\Delta \Phi}{\Delta t} $$

\subsubsection{Thermology}

$$ pV = C $$
where $C$ is a constant and $V$ is the volume of gas,
$p$ denotes the pressure.

\subsubsection{Mechanical Vibration}

$$ x = A\sin(\omega t+\varphi) $$

For electromagnetic waves
\[ c = \lambda f \]
where $c$ denotes the speed of light, $\lambda$ denotes wave length, and $f$ denotes the frequency. ($v=\lambda f$)

\subsubsection{Optics}

$$ n = \frac{\sin\theta_1}{\sin\theta_2} $$

\subsubsection{Conservation of momentum}

$$ p = mv $$

Impulse $I=Ft$, ($dp/dt = F$).

$$ p_1 + p_2 = p'_1 + p'_2 $$
