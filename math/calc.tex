% math-calc

Booknote for book~\cite{bib:calc1}.

\subsection{Function and Limit}

 \href{https://simple.wikipedia.org/wiki/Function_(mathematics)}
 {Wikipedia:Function}.

 \href{https://en.wikipedia.org/wiki/Limit_(mathematics)}
 {Wikipedia:Limit}.

 \href{https://en.wikipedia.org/wiki/Limit_of_a_function}
 {Wikipedia:Limit of a function}.

 \href{https://en.wikipedia.org/wiki/Limit_of_a_sequence}
 {Wikipedia:Limit of a sequence}.

 $$ \lim\limits_{x\rightarrow 0} \frac{\sin x}{x} = 1 $$
 $$ \lim\limits_{n\rightarrow \infty} (1 + \frac{1}{n})^n = e $$

 \href{https://en.wikipedia.org/wiki/Continuous_function}
 {Wikipedia:Continuous Function}.

 \href{https://en.wikipedia.org/wiki/Fundamental_theorem_of_calculus}
 {Wikipedia:Fundamental Theorem of Calculus}.

\subsection{Derivative}

 \href{https://en.wikipedia.org/wiki/Derivative}
 {Wikipedia:Derivative/Differentiation}.

$$ f'(x) = \lim_{h\rightarrow 0} \frac{f(x+h)-f(x)}{h} $$
$$ y-y_0 = f'(x_0) (x-x_0) $$

 Chain Rule:
 $$ \frac{dy}{dx} = \frac{dy}{du}\cdot \frac{du}{dx} $$

 Some other basic Rules:
 $$ [u(x)\pm v(x)]' = u'(x) \pm v'(x) $$
 $$ [u(x)v(x)]' = u'(x)v(x) + u(x)v'(x) $$
 $$ [\frac{u(x)}{v(x)}]' = \frac{u'(x)v(x) - u(x)v'(x)}{v^2(x)} $$
 Note, actually $(\frac{u}{v})' = (u\cdot v^{-1})'$.

 Some basic derivatives:
 $$(x^a)' = ax^{a-1}$$
 $$(\sin x)' = \cos x$$
 $$(\cos x)' = -\sin x$$
 $$(\tan x)' = \sec^2 x$$
 $$(a^x)' = a^x \ln a$$
 $$(x^x)' = e^x \ln x$$
 $$(\log_a x)' = \frac{1}{x\ln a}$$
 $$(\ln x)' = \frac{1}{x} $$
 $$(\arcsin x)' = \frac{1}{\sqrt{1-x^2}}  $$
 $$(\arccos x)' = - \frac{1}{\sqrt{1-x^2}}  $$
 $$(\arctan x)' = - \frac{1}{1+x^2}  $$

\subsection{Mean Value Theorem}

 \href{https://en.wikipedia.org/wiki/Mean_value_theorem}
 {Wikipedia:Mean value theorem}.

 \href{https://en.wikipedia.org/wiki/Rolle's_theorem}
 {Wikipedia:Rolle's Theorem}.

\subsection{L'Hospital's Rule}

 \href{https://en.wikipedia.org/wiki/L%27H%C3%B4pital%27s_rule}
 {Wikipedia:L'Hospital's Rule}.

 Suppose that
 \begin{enumerate}
 \item when $x\rightarrow a$, both function $f(x)$ and $F(x)$ tend to zero.
 \item over the range $\mathring{U}(x_0)$, both $f'(x)$ and $F'(x)$ exist and $F'(x)\neq 0$.
 \item $\lim_{x\rightarrow a} \frac{f'(x)}{F'(x)}$ exists or is infinitely great.
 \end{enumerate}
 then
 $$ \lim\limits_{x\rightarrow a} \frac{f(x)}{F(x)} = \lim\limits_{x\rightarrow a} \frac{f'(x)}{F'(x)} $$

\subsection{Antiderivative/Indefinite Integral}

 \href{https://en.wikipedia.org/wiki/Antiderivative}
 {Wikipedia:Antiderivative}.

$$ \int f(x) dx = F(x) + C $$

Frequently used indefinite integral:
$$ \int k dx = kx + C$$
$$ \int x^a dx = \frac{x^{a+1}}{a+1}  + C, (a\neq -1)$$
$$ \int \frac{dx}{x} = \ln|x| + C$$
$$ \int \frac{dx}{1+x^2} = \arctan x + C$$
$$ \int \frac{dx}{\sqrt{1-x^2}} = \arcsin x + C$$
$$ \int \cos x dx = \sin x + C$$
$$ \int \sin x dx = -\cos x + C$$
$$ \int \frac{dx}{\cos^2 x} = \tan x + C$$
$$ \int \frac{dx}{\sin^2 x} = -\cot x + C$$
$$ \int \sec x \tan x dx = \sec x + C$$
$$ \int \csc x \cot x dx = -\csc x + C$$
$$ \int e^x dx = e^x + C$$
$$ \int a^x dx = \frac{a^x}{\ln a} + C$$
$$ \int sh x dx = ch x + C$$
$$ \int ch x dx = sh x + C$$
$$ d \sin^2 x = 2\sin x \cos x dx = \sin 2x dx $$
$$ d \cos^2 x = -2\sin x \cos x dx $$
$$ \frac{1}{\sin x \cos x} = \frac{1}{\tan x \cos^2 x} = \frac{\sec^2 x}{\tan x} $$

Property:
$$ \int [f(x) + g(x)] dx = \int f(x) dx + \int g(x) dx $$
$$ \int kf(x) dx = k\int f(x) dx $$

  \subsubsection{Integration by Substitusion}

The first type:
$$ \int f[\varphi(x)]\varphi'(x) dx = F[\varphi(x)] + C = [\int f(u)du]_{u=\varphi(x)} $$

The second type:
$$ \int f(x) dx = [\int f(\psi(t)) \psi'(t) dt ]_{t = \psi^{-1}(x)} $$

Additional Frequently used:
$$ \int \tan x dx = -\ln |\cos x| + C$$
$$ \int \cot x dx = \ln |\sin x| + C$$

$$ \int \sin^p x \cdot \cos^q x dx $$
If there is a odd number between $p$ and $q$, then make up a differential.
Or find a way to descrease its exponential.

  \subsubsection{Integration by Parts}

$ (uv)' = u'v + uv' $, $ uv' = (uv)' - u'v $, then
$$ \int udv = uv - \int vdu $$

\subsection{Definite Integral}

 \href{https://en.wikipedia.org/wiki/Integral}
 {Wikipedia:Integral}.

Riemann integral
$$ \int_a^b f(x) dx = I = \lim_{\lambda\rightarrow 0} \sum_{i=1}^n f(\xi_i)\Delta x_i $$

$$ \int_a^b f(x) dx = - \int_b^a f(x) dx $$

If $f(x)\leq g(x), x\in [a, b]$, then $\int_a^b f(x)dx \leq \int_a^b g(x)dx$.

Let $m = min(f(x)), M = max(f(x)), x \in [a, b], a<b$, then $m(b-a) \leq \int_a^b f(x)dx \leq M(b-a)$.

If $f(x)$ is continuous on range $[a,b]$, then $\exists \xi \in [a, b]$, $\int_a^b f(x)dx = f(\xi)(b-a)$.

(Newton-Leibniz) If function $F(x)$ is a original function of $f(x)$'s, where $f(x)$
is continuous on range $[a,b]$, then $\int_a^b f(x)dx = F(b)-F(a)$.

 \subsubsection{Ex1: Finding the area enclosed by several curves}

 \begin{quote}
 Find the area of the ellipse $\frac{x^2}{a^2}+\frac{y^2}{b^2}=1$.
 $$ A = 4A_1 = 4 \int_0^a ydx $$
 We can write the equation of ellipse as $x = a\cos t$ and $y=b\sin t$ where
 $0\leq t\leq \frac{\pi}{2}$, then
 $$ A = 4\int_{\pi/2}^0 b\sin t \cdot -a\sin t dt = 4ab \int_0^{\pi/2} \sin^2 tdt$$
 $$ \sin^2 t = \frac{1-\cos 2t}{2} $$
 $$ \Rightarrow A = \pi a b $$
 \end{quote}

 \subsubsection{Ex2: Finding the area of a given sector}

 $$ A = \frac{1}{2}R^2 \theta $$
 $$ dA = \frac{1}{2} \varphi^2(\theta) d\theta $$

 \subsubsection{Ex3: Finding the volume of a given solid of rotation}

 $$ dV = \pi [f(x)]^2 dx $$
 Here the volume to be found are split into a pile of disks.

 \subsubsection{Ex4: Finding the arc length}
 TODO

\subsection{Multivariate functions}

 Note of book~\cite{bib:calc2}.

 \subsubsection{Partial derivative}

$$ \frac{\partial f}{\partial x} |_{(x_0,y_0)} =
 \lim_{\Delta x \rightarrow 0} \frac{f(x_0+\Delta x,y_0)-f(x_0,y_0)}
 {\Delta x} $$

 A special example of partial derivative. For ideal gas there is
 $$ pV=RT $$, however
 $$ \frac{\partial p}{\partial V} \cdot
    \frac{\partial V}{\partial T} \cdot
    \frac{\partial T}{\partial p} = -1 $$

 If $z=f(u,v)$, then $$ \frac{dz}{dt} = \frac{\partial z}{\partial u}\cdot
 \frac{\partial u}{\partial t} + \frac{\partial z}{\partial v}\cdot
 \frac{\partial v}{\partial t} $$

$$ \mathbf{grad} f = \frac{\partial f}{\partial x} \hat{i} +
 \frac{\partial f}{\partial y} \hat{j} $$

Hamilton Operator
$$ \nabla = \hat{i} \frac{\partial}{\partial x} + \hat{j}\frac{\partial}{\partial y}
		   + \hat{k} \frac{\partial}{\partial z} $$

The gradient of a numerical field $u$ is
$$ \mathbf{grad}~ u = \nabla u $$

 \subsubsection{Lagrange Multiplier Method}

To look for the possible extreme values of function $z=f(x,y)$ under
the condition $\varphi(x,y)=0$, first a Lagrange function should be
defined like this
$$ L(x,y) = f(x,y) + \lambda \varphi(x,y) $$
where $\lambda$ is a parameter. Combine the function with the following
conditions:
$$ f_x(x,y) +\lambda \varphi_x(x,y) = 0 $$
$$ f_y(x,y) +\lambda \varphi_y(x,y) = 0 $$
$$ \varphi(x,y) = 0 $$
Then several points with possible extreme value can be found be solving
the equation set.

E.g. find the maximum volume of a cuboid when its surface area is $a^2$.
$$ L(x,y,z) = xyz + \lambda(2xy + 2yz + 2xz - a^2)$$
$ yz+2\lambda(y+z) = 0$,
$ xz+2\lambda(x+z) = 0$,
$ xy+2\lambda(x+y) = 0$.
We get $x=y=z=\frac{\sqrt{6}}{6} a $.

 \subsubsection{Local Extreme Value}

 reference: operations research / optimization

 $f_{xx} > 0$, $f_{yy} > 0$, $f^2_{xy} < f_{yy}f_{xx}$

 Total differential $df = \frac{\partial f}{\partial x} dx + \frac{\partial f}{\partial y} dy$

 $$\Delta f = f(x) - f(x_0) \approx \frac{1}{2} \sum_i \sum_j \frac{\partial^2 f}{\partial x_i \partial x_j}
   \Delta x_i \Delta x_j$$

 $$\Delta f = \frac{1}{2} \Delta x^T H \Delta x = \frac{1}{2} \sum_r (\lambda_r \cdot a_r^2)$$
 where $H$ is hessian matrix, $\lambda_\ast$ is eigen value of the hessian matrix.

 minimum: $\Delta f = \frac{1}{2} \sum_r (\lambda_r \cdot a_r^2) > 0$
 maximum: $\Delta f = \frac{1}{2} \sum_r (\lambda_r \cdot a_r^2) < 0$

\subsection{Automatic Differentiation (AD)}

% cite 自动微分方法与最优化, 张海斌,高欢,科学出版社

Computation Graphs can represent a group of computations,
{\it e.g.} functions. They are formed of a set of symbols and
elementary (atomic) mathematical operations. Computation graph
can be used to reduce redundant computations to improve
efficiency. Apart from that, computation graph can help
finding the (partial) derivatives. This feature is extremely
helpful when the graph is too large or too complex to find
the derivatives manually. This way, automatic differentiation,
with the help of computation graphs, has been studied and
developed.

\href{https://en.wikipedia.org/wiki/Automatic_differentiation}
{Wikipedia:AD}.

  \begin{figure}[!h]
   \centering
   \includegraphics[scale=0.7]{pic/pic-ad-ex-crop.pdf}
   \caption{Example Diagram for AD}
   \label{fig:pic-ad-ex}
  \end{figure}

Here is an example of automatic differentiation, as shown in figure~\ref{fig:pic-ad-ex},
our target function is
$$y=e^{x_1x_2}(e^{x_1x_2}+x_1x_2\sin x_3)$$

Forward scan in the symbolic graph:
\begin{align*}
	x_4 &= x_1 * x_2 \\
	x_5 &= \sin x_3 \\
	x_6 &= \exp x_4 \\
	x_7 &= x_4 * x_5 \\
	x_8 &= x_6 + x_7 \\
	x_9 &= x_6 * x_8 \\
	y   &= x_9
\end{align*}

Forward Mode
\begin{align*}
	dx_4 &= dx_1 * x_2 + x_1 * dx_2 \\
	dx_5 &= \cos(x_3) dx_3 \\
	dx_6 &= \exp(x_4) dx_4 \\
	dx_7 &= dx_4 * x_5 + x_4 * dx_5 \\
	dx_8 &= dx_6 + dx_7 \\
	dx_9 &= dx_6 * x_8 + x_6 * dx_8 \\
	dy   &= dx_9
\end{align*}

Backward Mode, let $bx_i = \frac{\partial y}{\partial x_i}$, apply
the chain rule then we can obtain:
\begin{align*}
	bx_9 &= by \\
	bx_8 &= bx_9 * x_6 \\
	bx_7 &= bx_8 \\
	bx_6 &= bx_9 * x_8 + bx_8 \\
	bx_5 &= bx_7 * x_4 \\
	bx_4 &= bx_6 * x_6 + bx_7 * x_5 \\
	bx_3 &= bx_5 * \cos x_3
\end{align*}

% section Differential equation
\subsection{Oridnal Differential Equation}

\subsubsection{Separation of Variables}

Differential equations with separable variables generally look like this:
$$ g(y)dy = f(x)dx $$
where function $g(y)$ and $f(x)$ are continuous, and $y=\varphi(x)$.

$$ g[\varphi(x)] dy = f(x)dx = g[\varphi(x)]\varphi'(x)dx $$
$$ \int g(y)dy = \int f(x)dx \Rightarrow G(y) = F(x) + C $$

E.g.

$$ \frac{dy}{dx} = 2xy \Rightarrow \frac{dy}{y} = 2xdx $$
$$ \int \frac{dy}{y} = \int 2xdx = \ln|y| + C_1 = x^2 + C_2 $$
$$ y = \pm e^{x^2+C} = \pm e^C e^{x^2} $$

\subsubsection{Homogenous ODE}

$$ \frac{dy}{dx} = f(x,y) = \phi(\frac{y}{x}) $$
Let $u=\frac{y}{x}, y=ux$, and $\frac{dy}{dx} = u + x\frac{du}{dx} = \varphi(u) $
$$ x\frac{dy}{dx} = \varphi(u) - u \Rightarrow \frac{du}{\varphi(u) -u} = \frac{dx}{x} $$

E.g.

$$ y^2 + x^2 \frac{dy}{dx} = xy\frac{dy}{dx} $$
$$ \frac{dy}{dx} = \frac{y^2}{xy - x^2} =
 \frac{(\frac{y}{x})^2}{\frac{y}{x} - 1} $$
 Let $u=\frac{y}{x}$, then
 $$ u+x\frac{du}{dx}=\frac{u^2}{u-1} \Rightarrow
 (1-\frac{1}{u})du = \frac{dx}{x} $$
 $$ u - \ln|u| + C = \ln|x| \Rightarrow \ln|xu| = u+c $$
 $$ \ln|y| = \frac{y}{x} + c $$

%\subsection{First-order Linear ODE}
%
%$$ \frac{dy}{dx} + P(x)y = Q(x) $$
%
%TODO
%
%\subsubsection{Bernoulli ODE}
%
%
%$$ \frac{dy}{dx} + P(x)y = Q(x)y^n $$
%
%TODO
%
%\subsection{High-order ODE}
%
%\begin{itemize}
%	\item $y^{(n)} = f(x)$ 
%\end{itemize}
%

  \subsection{Series}
\href{https://en.wikipedia.org/wiki/Series_(mathematics)}
{Wikipedia:Series}.

\href{https://en.wikipedia.org/wiki/Convergence_tests}
{Wikipedia:Convergence Tests}.

\subsubsection{Tylor Formula}

 \href{https://en.wikipedia.org/wiki/Taylor_series}
 {Wikipedia:Tylor Series}.

\begin{align*}
   f(x) =& f(x_0) + f'(x_0)(x-x_0) + \frac{f''(x_0)}{2!}(x-x_0)^2 + \ldots \\
   &+\frac{f^{(n)}(x_0)}{n!}(x-x_0)^n + R_n(x)
\end{align*}
where
$$ R_n(x) = \frac{f^{(n+1)}(x_0)}{(n+1)!}(x-x_0)^{n+1} $$


\subsubsection{Geometric Series}

$$ \sum_{n=0}^\infty a_n x^n $$

Tylor Series.

$$ f(x) = f(0) + f'(0)x + \frac{f''(0)}{2!} x^2 + \cdots + \frac{f^{(n)}(0)}{n!} x^n + \cdots ~~ (-R<x<R) $$
$$ e^x = 1 + x + \frac{x^2}{2!} + \cdots + \frac{x^n}{n!} + \cdots ~~ R=+\infty$$
$$ \sin(x) = x - \frac{x^3}{3!} + \frac{x^5}{5!} + \cdots + (-1)^{(n-1)} \frac{x^{2n-1}}{(2n-1)!} + \cdots ~~ R=+\infty$$
$$ \cos(x) = 1 - \frac{x^2}{1}  + \frac{x^4}{4!} + \cdots + (-1)^n \frac{x^{2n}}{(2n)!} + \cdots ~~ R=+\infty$$
$$ \frac{1}{1-x} = 1 + x + x^2 + \cdots + x^n + \cdots ~~ (-1<x<1)$$

Eular Formula.
$$ e^{ix} = \cos x + i\sin x $$

\subsubsection{Fourier Series}

 First recall the following equations:

 $$ \int_{-\pi}^{\pi} 1 dx = 2\pi $$
 $$ \int_{-\pi}^{\pi} \sin^2 nx dx = \pi , n = 1, 2, 3, \ldots $$
 $$ \int_{-\pi}^{\pi} \cos^2 nx dx = \pi , n = 1, 2, 3, \ldots $$
 $$ \cos kx \cos nx = \frac{1}{2}[\cos(k+n)x + \cos(k-n)x] $$
 $$ \begin{array}{rcl} \int_{-\pi}^\pi \cos kx \cos nx dx & = & \frac{1}{2}\int_{-\pi}^\pi [\cos(k+n)x + \cos(k-n)x] \\
	 & = & \frac{1}{2} \big[ \frac{\sin(k+n)x }{k+n} + \frac{\sin(k-n)x }{k-n} \big]_{-\pi}^\pi \\
	 & = & 0 \end{array}$$

 Given a periodic function $f(x)$ with period of $2\pi$, we assume that the
 function can be written as the following form:
 $$ f(x) = \frac{a_0}{2} + \sum\limits_{k=1}^\infty (a_k \cos kx + b_k \sin kx) $$
 By integrating the above equation we can obtain $a_0$, $a_k$ and $b_k$ as follows:
 $$ \int_{-\pi}^\pi f(x) dx = \int_{-\pi}^\pi \frac{a_0}{2} dx + \int_{-\pi}^\pi [\sum_{k=1}^\infty (a_k \cos kx + b_k \sin kx)] dx = \frac{a_0}{2} \cdot 2\pi $$
 $$ \int_{-\pi}^\pi f(x) \cos nx dx = \int_{-\pi}^\pi \frac{a_0}{2} \cos nx dx + \sum_{k=1}^\infty  [ (a_k \int_{-\pi}^\pi\cos kx \cos nx + b_k \int_{-\pi}^\pi \sin kx \cos nx )] dx = a_n \int_{-\pi}^\pi \cos^2 nx dx = a_n \pi $$
 We can obtain $b_n$ in a simillar way by integrating $f(x)\sin nx$. In brief, the result looks like this:
 $$ a_n = \frac{1}{\pi} \int_{-\pi}^\pi f(x)\cos nx dx, (n=0,1,2,3,\ldots) $$
 $$ b_n = \frac{1}{\pi} \int_{-\pi}^\pi f(x)\sin nx dx, (n=1,2,3,\ldots) $$

 \subsection{Reference \& Resource}

1. List of Limits: https://en.wikipedia.org/wiki/List_of_limits

2. List of Integrals: https://en.wikipedia.org/wiki/Lists_of_integrals

3. List of Mathematical Series: https://en.wikipedia.org/wiki/List_of_mathematical_series
