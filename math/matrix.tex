\title{Theory of Matrices}
% Ref: 矩阵论,西北工业大学出版社,张凯院 等。

\subsection{Feynman Technique}
\begin{enumerate}
	\item Choose a concept
	\item Teach it to a toddler
	\item Identify gaps and go back to the source material
	\item Review and simplify (optional)
\end{enumerate}

\section{Linear Space and Linear Transformation}
\url{https://en.wikipedia.org/wiki/Vector_space#Definition}

{\bf Linear space} is a non-empty set where several properties are satisfied.

{\bf Theory 1.1} linear space $V$ has a unique zero element. Any element has
its own negative counterpart. This can be proved by contradiction.

{\bf Linera independence} If there is no solution to $\sum_i c_i x_i = 0$ s.t.
$\sum|c_i|=0$, the vectors $x_*$ are linearly independent.

{\bf Dimensionality of linear space} The maxium size of set consists of
linearly independent vectors in linear space $V$. Denoted as $\text{dim}V=n$.

{\bf Base of linear space} A group of linearly independent vectors in linear
space $V$. Any vector in this space can be represented as a linear combination
of this group of vectors. Note, the size of this group is also the
dimensionality of this linear space.

{\bf Coordinate system} A base $x_1,\ldots,x_n$ of linear space $V$, that
$\forall x\in V^n, x = \sum_i \xi_i x_i$. The coefficients for this linear
combination is called coordinate, denoted as column vector $(\xi_1,\ldots,\xi_n)^T$.

{\bf Theory 1.2} $\forall x\in V^n$ can be uniquely represented by the base
vector as a linear combination. This can be proved by contradiction.

{\bf Base shift matrix} $(y_1,y_2,\ldots,y_n) = (x_1,x_2,\ldots,x_n)C$.
Here matrix $C$ is called the base shift matrix, and $y_*$ is the new base
while $x_*$ is the original base. Given $\forall x\in V^n$, and
$x = \sum_i \xi_i x_i = \sum_i \eta_i y_i$, we have
\begin{align}
(x_1,\ldots,x_n)(\xi_1,\ldots,\xi_n)^T = (y_1,\ldots,y_n)(\eta_1,\ldots,\eta_n)^T \\= (x_1,\ldots,x_n)C(\eta_1,\ldots,\eta_n)^T
\end{align}
That is to say
$$(\xi_1,\ldots,\xi_n)^T = C(\eta_1,\ldots,\eta_n)^T$$
$$(\eta_1,\ldots,\eta_n)^T = C^{-1}(\xi_1,\ldots,\xi_n)^T$$

{\bf Linear subspace} $V_s \subset V$ and $V_s \neq \Phi$, s.t.
$\forall x, y \in V_s, x+y\in V_s$, and $\forall x\in V_s, kx \in V_s$.
Note, the dim of this subspace is less or equal to that of the original
linear space, i.e. $\text{dim} V_s \leq \text{dim} V$.

{\bf Column space} $R(A) = \text{lincomb}(a_1,\ldots,a_n) = \{Ax|x\in R^n\}$
where $A\in R^{m\times n}$ and $a_* \in R^m$ is the columns of matrix $A$.
Similarly, the row space can be defined as $R(A^T) = \{A^Tx|x\in R^m\} \subset R^n$.
Note that $\text{rank} A = \text{dim} R(A) = \text{dim} R(A^T)$.

{\bf Null space} $N(A) = \{x|Ax=0\}$ where $A\in R^{m\times n}$. Note,
$\text{rank} A + n(A) = n$ where $n(A) = \text{dim} N(A)$.

{\bf Theory 1.3} A base of subspace $V_s$ can be extended into a base of $V$
by adding $\text{dim} V - \text{dim} V_s$ vectors to the base.

\appendix

\section{External Reference}

1. Wikipedia
