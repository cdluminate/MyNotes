% CVPR 2022 Paper Template
% based on the CVPR template provided by Ming-Ming Cheng (https://github.com/MCG-NKU/CVPR_Template)
% modified and extended by Stefan Roth (stefan.roth@NOSPAMtu-darmstadt.de)

\documentclass[10pt,twocolumn,letterpaper]{article}

%%%%%%%%% PAPER TYPE  - PLEASE UPDATE FOR FINAL VERSION
\usepackage[review]{cvpr}      % To produce the REVIEW version
%\usepackage{cvpr}              % To produce the CAMERA-READY version
%\usepackage[pagenumbers]{cvpr} % To force page numbers, e.g. for an arXiv version

% Include other packages here, before hyperref.
\usepackage{graphicx}
\usepackage{amsmath}
\usepackage{amssymb}
\usepackage{booktabs}

\usepackage{microtype}
%\usepackage{xcolor}
%\usepackage{graphicx}


% It is strongly recommended to use hyperref, especially for the review version.
% hyperref with option pagebackref eases the reviewers' job.
% Please disable hyperref *only* if you encounter grave issues, e.g. with the
% file validation for the camera-ready version.
%
% If you comment hyperref and then uncomment it, you should delete
% ReviewTempalte.aux before re-running LaTeX.
% (Or just hit 'q' on the first LaTeX run, let it finish, and you
%  should be clear).
\usepackage[pagebackref,breaklinks,colorlinks]{hyperref}


% Support for easy cross-referencing
\usepackage[capitalize]{cleveref}
\crefname{section}{Sec.}{Secs.}
\Crefname{section}{Section}{Sections}
\Crefname{table}{Table}{Tables}
\crefname{table}{Tab.}{Tabs.}


%%%%%%%%% PAPER ID  - PLEASE UPDATE
\def\cvprPaperID{*****} % *** Enter the CVPR Paper ID here
\def\confName{CVPR}
\def\confYear{2022}


\begin{document}

%%%%%%%%% TITLE - PLEASE UPDATE
\title{PST: Path Sequence Transformer for Raster Image Classification}

\author{First Author\\
Institution1\\
Institution1 address\\
{\tt\small firstauthor@i1.org}
% For a paper whose authors are all at the same institution,
% omit the following lines up until the closing ``}''.
% Additional authors and addresses can be added with ``\and'',
% just like the second author.
% To save space, use either the email address or home page, not both
\and
Second Author\\
Institution2\\
First line of institution2 address\\
{\tt\small secondauthor@i2.org}
}
\maketitle

%%%%%%%%% ABSTRACT
\begin{abstract}
    %
    In this paper, we restart image classification from scratch, based on
    vector graphics instead of traditional raster graphics represented in
    multi-dimensional arrays.
    %
    By doing so, image classification can be cast as a \textit{pure} sequence
    classification problem, as vector graphics or traced raster images
    can be represented in a hierarchical sequence of paths.
    %
    In this paper, we present Path Sequence Transformer for this purpose.
    %
    The proposed method is evaluated on MNIST, Fashion-MNIST, CIFAR-10,
    CIFAR-100, Tiny-ImageNet, as well as ImageNet-1k.
\end{abstract}

%%%%%%%%% BODY TEXT
\section{Introduction}
\label{sec:intro}

\textcolor{red}{Nobody has done this. But this is a more natural way to represent images in a sequence. 
And such sequence is approximate to human strokes.}

This will be novel enough as long as it works reasonably for MNIST, CIFAR-10,
CIFAR-100, Tiny-ImageNet, and ImageNet.

Reference: Image Vectorization: LIVE \cite{live}

\section{Preliminary Design}

hierarchical transformer.

path transformer for path representation. input is paths, sequence of points. init vector is color.

image transformer for image represetnation. input is sequence of paths.

\section{Preliminary Experiments}

GRU and HGRU works for both mnist and cifar

\begin{verbatim}
Dataset         Model     Accuracy     Parameters
=================================================
MNIST           LeNet     98.9         431k
-------------------------------------------------
MNIST           RNN       96.43        34k
..              GRU       97.31        84k
..              LSTM      96.89        109k
..              PST       97.30        211k
-------------------------------------------------
MNIST           HRNN      96.66        59k
..              HGRU      98.00        159k
..              HLSTM     97.24        209k
..              HPST      98.17        412k
-------------------------------------------------
FashionMNIST    LeNet     88.9         431k
-------------------------------------------------
FashionMNIST    RNN       65.97        34k
..              GRU       72.88        84k
..              LSTM      72.08        109k
..              PST
----------------------------------
FashionMNIST    HRNN
..              HGRU
..              HLSTM
..              HPST      85.57
----------------------------------
CIFAR10         HRNN
..              HGRU
..              HLSTM
..              HPST
===================================
\end{verbatim}


%%%%%%%%% REFERENCES
{\small
\bibliographystyle{ieee_fullname}
\bibliography{egbib}
}

\end{document}
