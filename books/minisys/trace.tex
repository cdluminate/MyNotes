Let's trace the bootup process of the machine.\newline
And, this is just a brief trace.
\subsection{Overall Process}
\begin{verbatim}
* CPU Power up
- Load BIOS from CS:IP=FFFF:0000 Entry
- Load GRUB to 0x7c00 via int 0x19
- Load vmlinuz
- real mode : arch/x86/boot/header.S : _start
- read mode : arch/x86/boot/main.c
- protected mode (0x100000): arch/x86/boot/compressed/head_64.S
- protected mode : arch/x86/boot/compressed/head64.c
- arch independent : start_kernel ();
- create init rootfs : mnt_init ();
- kernel init : rest_init ();  kernel_init ();
- load initramfs : init/initramfs.c : populate_rootfs ();
+ if cpio initrd
    /init
  else if image initrd
    /linuxrc
  fi
- userspace init : /sbin/init
\end{verbatim}

\subsection{BIOS/EFI}
BIOS/EFI reads the machine code at a fixed location on hard disk, typically sector 0, then execute it.\newline
This piece of machine code belongs to boot loader.
\subsection{GRUB2}
\subsubsection{grub stage 1}
Read then execute the first 512 Bytes, then look for file systems.
\subsubsection{grub stage 2}
Grub loads grub.cfg, then loads \emph{linux} and \emph{initrd.img} (or \emph{initramfs.img}) into memory, finally boot them.
\subsection{linux}
\subsubsection{bzImage}
You can look up kernel doc\cite{bib:linux.doc.boot}.
For example, \texttt{ARCH=x86\_64}, find file \texttt{arch/x86/}:
\begin{framed}\begin{verbatim}
boot/header.S:

    293 _start:
    294         # Explicitly enter this as bytes, or the assembler
    295         # tries to generate a 3-byte jump here, which causes
    296         # everything else to push off to the wrong offset.
    297         .byte   0xeb        # short (2-byte) jump
    298         .byte   start_of_setup-1f

    456 start_of_setup:
    457 +-- 51 lines: # Force %es = %ds---------------------------------------------
    508 # Jump to C code (should not return)
    509     calll   main

boot/main.c:

    135 void main(void)
    136 +-- 48 lines: {-------------------------------------------------------------
    184     go_to_protected_mode();
    185 }   

boot/pm.c:

    104 void go_to_protected_mode(void)
    105 +-- 19 lines: {-------------------------------------------------------------
    124     protected_mode_jump(boot_params.hdr.code32_start,
    125                 (u32)&boot_params + (ds() << 4));
    126 }

boot/pmjump.S:
    
    26 GLOBAL(protected_mode_jump)
    27 +-- 18 lines: movl %edx, %esi  # Pointer to boot_params table---------------
    45 2:  .long   in_pm32         # offset

    51 GLOBAL(in_pm32)
    52 +-- 24 lines: # Set up data segments for flat 32-bit mode-------------------
    76     jmpl    *%eax           # Jump to the 32-bit entrypoint
    77 ENDPROC(in_pm32)
\end{verbatim}\end{framed}
After executing pmjump.S, the Processor is in protected mode.
\begin{framed}\begin{verbatim}
boot/compressed/head_64.S:

    37 ENTRY(startup_32)
    38 +--142 lines: 32bit entry is 0 and it is ABI so immutable!------------------
    180     pushl   $__KERNEL_CS
    181     leal    startup_64(%ebp), %eax

    225 ENTRY(startup_64)
    226 +-- 15 lines: 64bit entry is 0x200 and it is ABI so immutable!--------------
    241     jmp preferred_addr

    293 preferred_addr:
    294 +-- 61 lines: #endif--------------------------------------------------------
    355 /*
    356  * Jump to the relocated address.
    357  */
    358     leaq    relocated(%rbx), %rax
    359     jmp *%rax

    376 relocated:
    377 +-- 24 lines: Clear BSS (stack is currently empty)--------------------------
    401 /*
    402  * Do the decompression, and jump to the new kernel..
    403  */
    404     pushq   %rsi            /* Save the real mode argument */
    405     movq    $z_run_size, %r9    /* size of kernel with .bss and .brk */
    406     pushq   %r9
    407     movq    %rsi, %rdi      /* real mode address */
    408     leaq    boot_heap(%rip), %rsi   /* malloc area for uncompression */
    409     leaq    input_data(%rip), %rdx  /* input_data */
    410     movl    $z_input_len, %ecx  /* input_len */
    411     movq    %rbp, %r8       /* output target address */
    412     movq    $z_output_len, %r9  /* decompressed length, end of relocs */
    413     call    decompress_kernel   /* returns kernel location in %rax */
    414     popq    %r9
    415     popq    %rsi

boot/compressed/misc.c:

    369 asmlinkage __visible void *decompress_kernel(void *rmode, memptr heap,
    370 +-- 53 lines: unsigned char *input_data,------------------------------------
    423     debug_putstr("\nDecompressing Linux... ");
    424     decompress(input_data, input_len, NULL, NULL, output, NULL, error);
    425     parse_elf(output);
    426     /*
    427      * 32-bit always performs relocations. 64-bit relocations are only
    428      * needed if kASLR has chosen a different load address.
    429      */
    430     if (!IS_ENABLED(CONFIG_X86_64) || output != output_orig)
    431         handle_relocations(output, output_len);
    432     debug_putstr("done.\nBooting the kernel.\n");
    433     return output;
    434 }

boot/compressed/head_64.S:

    376 relocated:
    377 +-- 36 lines: Clear BSS (stack is currently empty)--------------------------
    413     call    decompress_kernel   /* returns kernel location in %rax */
    414     popq    %r9
    415     popq    %rsi
    416 
    417 /*
    418  * Jump to the decompressed kernel.
    419  */
    420     jmp *%rax

kernel/head_64.S:

    49 startup_64:
    50 +--111 lines: At this point the CPU runs in 64bit mode CS.L = 1 CS.D = 0,---
    161     jmp 1f

    162 ENTRY(secondary_startup_64)
    163 +--122 lines: At this point the CPU runs in 64bit mode CS.L = 1 CS.D = 0,---
    285     movq    initial_code(%rip),%rax
    286     pushq   $0      # fake return address to stop unwinder
    287     pushq   $__KERNEL_CS    # set correct cs
    288     pushq   %rax        # target address in negative space
    289     lretq

    310     GLOBAL(initial_code)
    311     .quad   x86_64_start_kernel

kernel/head64.c:

    141 asmlinkage __visible void __init x86_64_start_kernel(char * real_mode_data)
    142 +-- 47 lines: {-------------------------------------------------------------
    189     x86_64_start_reservations(real_mode_data);
    190 }

    192 void __init x86_64_start_reservations(char *real_mode_data)
    193 +--  7 lines: {-------------------------------------------------------------
    200     start_kernel();
    201 }

../../init/main.c:

    489 asmlinkage __visible void __init start_kernel(void)
    490 {

\end{verbatim}\end{framed}

\subsubsection{vmlinux}
see \texttt{linux-4.0/init/main.c}:
\begin{framed}\begin{verbatim}
489 asmlinkage __visible void __init start_kernel(void)
490 +---183 lines: {-----------------------------------------------------------
673     /* Do the rest non-__init'ed, we're now alive */
674     rest_init();
675 }
\end{verbatim}\end{framed}
trace \texttt{rest\_init()}:
\begin{framed}\begin{verbatim}
382 static noinline void __init_refok rest_init(void)
383 {
384 +--  8 lines: int pid;-----------------------------------------------------
392     kernel_thread(kernel_init, NULL, CLONE_FS);
\end{verbatim}\end{framed}
trace \texttt{kernel\_init()}:
\begin{framed}\begin{verbatim}
924 static int __ref kernel_init(void *unused)
925 {
926 +-- 20 lines: int ret;-----------------------------------------------------
946     /*
947      * We try each of these until one succeeds.
948      *
949      * The Bourne shell can be used instead of init if we are
950      * trying to recover a really broken machine.
951      */
952     if (execute_command) {
953         ret = run_init_process(execute_command);
954         if (!ret)
955             return 0;
956         panic("Requested init %s failed (error %d).",
957               execute_command, ret);
958     }
959     if (!try_to_run_init_process("/sbin/init") ||
960         !try_to_run_init_process("/etc/init") ||
961         !try_to_run_init_process("/bin/init") ||
962         !try_to_run_init_process("/bin/sh"))
963         return 0;
964 
965     panic("No working init found.  Try passing init= option to kernel. "
966           "See Linux Documentation/init.txt for guidance.");
967 }
\end{verbatim}\end{framed}
It shows that, from here the kernel executes the init program as pid 1, then init program do the Operating System initialization things.

\subsection{Busybox init}
\begin{framed}\begin{verbatim}
busybox-1.23.2/init/init.c:

    1022 int init_main(int argc, char **argv) MAIN_EXTERNALLY_VISIBLE;
    1023 int init_main(int argc UNUSED_PARAM, char **argv)
    1024 +-- 98 lines: {------------------------------------------------------------
    1122         parse_inittab();
    1123     }

    652 static void parse_inittab(void)
    653 {
    654 #if ENABLE_FEATURE_USE_INITTAB
    655     char *token[4];
    656     parser_t *parser = config_open2("/etc/inittab", fopen_for_read);
    657 
    658     if (parser == NULL)
    659 #endif
    660     {
    661         /* No inittab file - set up some default behavior */
    662         /* Sysinit */
    663         new_init_action(SYSINIT, INIT_SCRIPT, "");
    664         /* Askfirst shell on tty1-4 */
    665         new_init_action(ASKFIRST, bb_default_login_shell, "");
    666 //TODO: VC_1 instead of ""? "" is console -> ctty problems -> angry users
    667         new_init_action(ASKFIRST, bb_default_login_shell, VC_2);
    668         new_init_action(ASKFIRST, bb_default_login_shell, VC_3);
    669         new_init_action(ASKFIRST, bb_default_login_shell, VC_4);
    670         /* Reboot on Ctrl-Alt-Del */
    671         new_init_action(CTRLALTDEL, "reboot", "");
    672         /* Umount all filesystems on halt/reboot */
    673         new_init_action(SHUTDOWN, "umount -a -r", "");
    674         /* Swapoff on halt/reboot */
    675         new_init_action(SHUTDOWN, "swapoff -a", "");
    676         /* Restart init when a QUIT is received */
    677         new_init_action(RESTART, "init", "");
    678         return;
    679     }

    145 /* Default sysinit script. */
    146 #ifndef INIT_SCRIPT
    147 # define INIT_SCRIPT  "/etc/init.d/rcS"
    148 #endif

\end{verbatim}\end{framed}
