\documentclass[a4paper,10pt]{article}
\usepackage[margin=1in]{geometry}
\usepackage{parallel}
\usepackage{times}

\begin{document}

\section{Mirror Worlds}

A deluge\footnote{flood} of data is giving rise to a new economy.
Ludwig asks how it will work.

An army of doppelgangers is invading the world.
Digital copies of aircraft engines, wind turbines and other heavy equipments came first.
Now the electronic ghosts of smaller and larger things are joining them in the virtual realm, from toothbrushes and traffic lights to entire shops and factories.
Even human have begun developing these alter egos.
In America the national football league is planning to design an electronic avatar for every player.

These "digital twins", as geeks term them, are far more than replicas of the original.
Think of them more as shadows that are, thanks to a multitude of sensors and wireless connectivity, intimately linked to their physical selves, and every producin oceans of data.
If something happens in the real world, it is rapidly reflected in this shadow realm.
Some digital twins already come with the laws of nature programmed in.
They double as a database of everything that has ever happended to the original.
This makes it possible to look into their future.
Sports coaches, for instance, will be able to run simulations, predict when an athlete might get injured and adjust training routines to avoid problems.

\section{A deadly disease disrupts}


the new coronavirus may have persistent impact to global supply chains.
[
	the new coronavirus could have a lasting impact on global supply chains.
	{
		The new coronavirus could have lasting impact on global supply chains.

% para 1

To grasp how the new coronavirus could have affected the global economics, Apple is the first company we should inspect.
[
	To glimpse the impact of the new coronavirus on global businesses, consider apple.
	{
		To glimpse the impact of the new coronavirus on global businesses, consider apple.

This American techinical giant highly depends on china mainland for producing devices and assembling products, and about 50 managers of this company travel
between california and china by US flights.
[
	Such is the American tech titan's reliance on the Chinese mainland for parts and assembly that united airlines typically shuttles some 50 of its executives
	between california and China each day.
	{
		This american tech giant, highly dependent on china mainland for parts and assembly, relies on united airlines for shuttling
		about 50 executives between california and china every day.

However, things become quite different now.
[
	But not at the moment.
	{
		Things became different.

Airline companies including US Airline have paused all the flights from and to
China.
[
	United and other carriers have suspended flights to and from China.
	{
		United airline and other carriers have suspended flights from and to china.

Lacking of workers, foxcon (which produced most of the iPhones over the world)
cannot continue to manufacture at full speed.
[
	A lack of workers meant that after the end of the lunar new-year holiday
	foxcon, which makes most of apple's iphones in china, could not get
	its assembly plants back to full capacity last week.
	{
		A lack of workers meant that after the end of lunar new-year holiday
		foxconn, which makes most of apple's iphones in china, could not get
		its assembly plants back to full capacity last week.

Analysts estimated that the yield of iPhone would reduce by five to ten percent
in this season, which may eventually make the plan to extend the airpods production
aborted.
[
	Analysts reckon that the virus could lead to apple shipping 5-10\% fewer
	iphones this quarter and could scupper its plans to ramp up production of
	its popular airpods.
	{
		Analysts reckon that the virus could lead to apple shipping 5-1\% fewer
		iphones this quater, and could abort its plans to ramp up production
		of its popular airpods.

% para 2

As the new coronavirus spreads, its impact on businesses increases.
[
	As covid-19 spreads, its effect on business is amplified.

Number of tourists entering and departing China has been significantly reduced.
[
	Tourism into and out of the mainland has plunged.

It is estimated that about 400k chinese tourists would cancel their trip to japan by the end of march.
[
	Some 400,000 chinese tourists are forcast to cancel trips to japan by the end of march.

In asia a large mail ship, rejected to dock by five countries and regions due to tens of infected passengers,
eventually docked at cambodia with approval.
[
	one large cruise ship in asia was turned away by five countries and regions because scroes on board are infected (cambodia at last allowed it to dock).

Singapore spacecraft exhibition earned 0.25bln dollars for this city country in 2018, but this year 70 companies including munition giant lockheed martin have canceled
their participation, hence the drastic decrease of the exhibition income last week.
[
	the singapore air show earned the city-state some 250m in 2018, but far less last week owing to cancellations by 70 companies including lockheed marin, an american defense giant.

mobile world congress, originally planned to host in basselona, has been canceled after the quit of companies including
vodafone, BT, facebook and amazon.
[
	the mobile world congress, a giant telecoms conference due to take place in barcelona this month, has been cancelled after
	companies from vodafone and BT to facebook and amazon pulled out.

the new coronavirus could seriously break the global supply chain and impose huge loss on the global economy, which
gets more and more clear as time goes.
[
	it is increasingly clear that the virus could damage global supply chains, costing the world's economy dearly.

% para 3

Most international companies were completely unprepared for it.
[
	Most multinational firms have been caught by surprise.

It is not the first time their asian supply chain being inpacted.
[
	This is not the first time they have suffered shocks to their asian supply chains

Tsunami in japan and the flood in thailand, both happend in the same year 2011,
have also interrupted the production of many companies.
[
	the tsunami that hit japan in 2011 and devastating floods in thailand the same year
	disrupted production for many big firms.

Recently, the trade war towards china lauched by trump, reveals the risk of
the supply chain being overly dependent on china.
[
	more recently, trump's trade war with china has exposed the risks of supply chains
	that rely too heavily on the mainland china.

However, the managers in these companies took no action to counter such impacts
as the new coronavirus outbreak.
[
	but the bosses of such businesses have done little to prepare for shocks such as
	that inflicted by the outbreak of the new coronavirus.

% para 4

Due to such mistakes, companies are being punished by the investors.
[
	investors are punishing companies for this failure.

Since the first announcement of the plague outbreak in early january, stock performance of american companies highly dependent on china has been
lagging behind the standard-500 by 5\%. (see figure)
[
	the shares of american firms with strong exposure to china have underperformed the S\&P 500 index by 5\% since early jan,
	when news of the outbreak first broke. (see chart)

% para 5

For most companies, the following several months could be even tougher for three reasons.
[
	there are three reasons to think the coming months could prove even more unpleasant for many firms.

First, the strategies of large international companies that aim at lowering cost render
them more susceptible to the risk of supply chain.
[
	first, big multinationals have left themselves dangerously exposed to supply-chain risk owing to
	strategies designed to bring down their costs.

For instance, many companies are too confident to believe in the "in-time" supply
only kept goods for use of several weeks.
[
	for example, many keep only enough stock on hand to last a few weeks, confident
	that they can always replenish their inventories "just in time".

Such confidence turns out to be unwise, said bindiya vakil of the inquiry company resilinc.
[
	that confidence is misplaced, argues bindiya vakil of resilink, a consultancy.

%

the second weakness is that the reliance level of big companies on china is much higher than that in 2003 when the SARS outbroke.
[
	the second vulnerability arises rom the fact that gaint firms are much more reliant on chinese factories today than they were at the
	time of the SARS outbreak in 2003.

Nowadays china has contributed as much as 16\% of the global GDP, while it was merely 4\% back in 2003.
[
	china now accounts for 16\% of global GDP, up from 4\% back then.

Now, the export of textile product and cloths takes 40\% of the global amount, while that of furniture takes 26\% globally.
[
	its share of all exports in textiles and apparel is now 40\% of the global total.
	it generates 26\% of the world's furniture exports.

China is also a notable consumer of raw materials for manufactoring industry.
[
	it is also a voracious consumer of things such as metals, needed in manufacturing.

In 2003, china absorbed 7\% of the gloabl mining product import, while it is quite close to 1/5 of the global imports.
[
	in 2003 china sucked in 7\% of global mining imports. today it claims closer to a fifth.

%

research scientist koray kose at Gartner points out that, it is not only the expansion of the chinese manufacturing industry that is importat.

Since 2003, factories have been expanded from costal area to poorer inland areas, such as wuhan where the new cov outbreaks.

nowadays, workers from less advanced areas work in factories all over the country, and return their homes in holiday.

kose thinks, such interlaced relationship greatly increased the risk of the supply chain.

many companies' relying on each other more and more also lead to higher risks.

suppliers in mainland do not only simply assemble products, but they also manufacture many parts within it.

%

\section{Fighting the Slump}

As the virus rages, governments need to be able to dial financial support up and down for people and firms.

\begin{Parallel}{0.5\textwidth}{0.5\textwidth}
	\ParallelPar
	\ParallelLText{It took only two months to completely change the world economics.}
	\ParallelRText{In just two months the world economy has been turned upside down.}

	\ParallelPar
	\ParallelLText{The stock market had plunged by nearly a third.}
	\ParallelRText{Stockmarkets have collapsed by a third}

	\ParallelPar
	\ParallelLText{In many countries, factories, airports, officies, schools and stores had been closed in order to suppress the spread of virus.}
	\ParallelRText{and in many countries factories, airpots, offices, schools and shops have been closed to try to contain the virus.}

	\ParallelPar
	\ParallelLText{Workers are concerned about their jobs, and the investors are afraid of the debts.}
	\ParallelRText{Workers are worried about their jobs and investors fea companies will default on their debts.}

	\ParallelPar
	\ParallelLText{All of these indicates the world is experiencing the most rapid economics shrinkage throughout the modern history.}
	\ParallelRText{All this points toone of the the sharpest economic contractions in modern times.}

	\ParallelPar
	\ParallelLText{Chinese GDP in Jan and Feb has shrinked for about 10\% to 20\% compared to that of the last year.}
	\ParallelRText{China's GDP probably shrank by 10-20\% in Jan and Feb compared with a year earlier.}

	\ParallelPar
	\ParallelLText{As long as the virus keeps spreading, possibly the America and Europe have to face similar shrinkage, and the Asia economics will possibly continue to slump.}
	\ParallelRText{For as long as the virus rages, similar drops are likely in America and Europe, which could trigger a further downward lurch in Asia.}

	\ParallelPar
	\ParallelLText{The government has to interfere at a large scale in order to prevent this impact from turning into a crisis.}
	\ParallelRText{Massive government intervention is required to ensure that this shock does not spiral into a depression.}

	\ParallelPar
	\ParallelLText{But large scale action is not enough, a new financial measure also has to be enforced quickly.}
	\ParallelRText{But scale alone is not good enough -- new financial tools need to be deployed, and fast.}

	\ParallelPar
	\ParallelLText{}
	\ParallelRText{}
\end{Parallel}

Western authorities have already promised huge sums.
A crude estimate for America, Germany, Britain, France and Italy, including spending pledges, tax cuts, central-bank cash injections and load guarantees, amounts to \$7.4trn, or 23\% of their GDP.
Yet central banks are responsible for over four-fifths of that and many governments are doing too little.
A huge array of policies is on offer, from holidays on mortgage-payments to bail-outs of Paris cafes.
Meanwhile, orthodox stimulus tools may not work well.
Interest rates in the rich world are near zero, depriving central banks of their main lever.
Governments typically try to stimulate demand in a downturn but people trapped at home cannot spend freely.
History is not much of a guide.
The global pandemic of 1918 took place when the economy was wrecked by war.
China has endured a lockdown but its social model is different form the West's.

What to do? An economic plan needs to target two groups: households and companies.
And it needs to be fast, efficient and flexible so that if the virus retreats only to resurge, workers and firms can be confident that governments will dial assistance down and up again as needed.
Start with households, where large government spending is needed.
One aim is to protect vulnerable people, by subsidising sick pay and ensuring those without insurance have health care.
But spending is also needed to discourage lay-offs at firms running far below capacity, by subsidising workers' wages -- an area where Germany has led the way.

Governments also need to jerry-rig digital systems so they are able to distribute cash to households directly, as Hong Kong hopes to.
The aim should be to have the capability to ramp further support up and down quickly.
Many places, including America, rely on sluggish postal services and tax agencies to distribute cash.
If funds can be sent instantly through mobile phones or online bank accounts, people will feel more confident and avoid hoarding cash and slowing the recovery when the virus recedes.

All this spending will cost government dear, but the fiscal stimulus of about 1\% of GDP so far across Europe is clearly too low.
America's plan to spend 5\% is closer to the mark given the risk of a double-digit GDP drop.
As fiscal deficits balloon, governments will have to issue piles of bonds.
Central banks should step in to buy those bonds in order to keep yeilds low and markets orderly.
Inflation is a second-order concern and there is little danger of it taking off.
To prevent a euro-zone crisis, the European central bank plans to buy 750bn of assets.
But it and European governments should also give a clear guarantee of sovereign support for Italy and other peripheral economies.

The second priority is to get cash to millions of companies, whose failure would damage the economy's potential.
They face a cash drought even as bills fall due.
Bond markets are closed to many of them.
Mass defaults would fuel unemployment and bad debts at banks, and make it harder for commercial activity to rebound.
Most governments have intervened, but in flawed ways.
France says nationalisation is an option -- which firms will resist.
America is propping up the commercial-paper market, but this funds only a fraction of all corporate debt and is used by big firms -- not small ones, which employ most people.
Germany and Britain have offered loan-guarantee schemes but it is unclear who will process millions of loan applications.
The best approach is to use the banking system -- almost all firms have accounts, and banks know how to issue loans.
Governments should offer banks cheap funding to lend to their clients while guaranteeing that it will bear most of the losses.
Borrowers could be offered bonuses for repaying loads early.

There are huge drawbacks to all of this.
Public and corporate debt will soar.
Handouts will be given to rich people and loads extended to firms that are badly run.
But even with this fearful list of side-effects, the advantages are overwhelming.
Cash will be distributed fast. Vulnerable people will be able to get by.
Households will be confident enough to spend when conditions improve.
And firms will keep their workforces and plants intact, ready to get back to action when this dark episode has passed.

\end{document}
