\documentclass[a4paper,twocolumn,10pt]{article}

\usepackage{CJKutf8}
\usepackage{times}
\usepackage{framed}

\begin{document}
\begin{CJK}{UTF8}{gkai}

\tableofcontents

\section{Tech Write}

In this section, an amount of good sentences from a book
for teaching technology English writing~\cite{bib:techwrite.xidian} are extracted.

	\subsection{Different Types of Words}

	\subsubsection{A/An/The}

	1. An emf is applied across the ends of a copper wire.

	2. A transmitter consists commonly of several parts. 

	3. This is an n-valued function.

	4. A magnet has an S pole and an N pole.

	5. Sudden changes of voltage in an RLC circuit may produce ringing.

	6. This is an 8-volt battery.

	7. A UFO appeared in the sky last night.

	8. When an electric current flows through a wire, it meets (will meet)
	some opposition. The opposition is referred to as resistance.

	9. The resistance of a given section of an electric circuit is equal
	to the ratio of its voltage to the current through this section of the 
	circuit.

	10. The use of trigonometry to describe the electrical signal has proved
	very valuable for engineers.

	11. The ratio of inductive reactance to resistance is called the Q of the
	circuit.

	12. The control of systems is an interdisciplinary subject.
	
	13. In Chap.1 the analysis of diode was discussed in terms of three general methods.

	14. The design of control systems depends on the application of complex-variable
	theory.

	15. Consider the evaluation of the following integral around a closed contour C.

	16. The reader should be aware that the binary data code is converted to base 10
	for human consumption.

	17. Electricity is widely used in industry and agriculture.

	18. Electrical energy can be changed by electric motors into mechanical energy.

	19. Machines are run by electricity.

	20. Titles, name of books, can have ''A/An/The'' omitted. \emph{e.g.} Study of
	phase-locked loops, Introduction to Computers.

	21. Specific Nouns should not follow ''A/An/The''. \emph{e.g.} China, Xidian
	University.

	22. Captions for figures can also have ''A/An/The'' omitted. \emph{e.g.}
	Fig 1 Generation of sine wave by vertical component of rotating vector.
	Fig 2 Effect of ammeter resistance on current in circuit.

	23. Ohm first discovered the relationship between current, voltage, and resistance.

	24. A transistor consists of three parts: emitter, base and collector.

	25. This is called a hertz in honor of Heinrich Hertz, discoverer of radio waves.

	26. The unit of power if a joule per second, which is called a watt (W), in honor
	of James Watt, developer of the steam engine.

	27. This equation is known as Ohm's law.

	28. The voltage induced in the primary winding is proportional to the primary
	inductance according to Faraday's law.

	29. This equation can also be obtained from the Karnaugh map shown in Fig.~1.

	30. We first determine the Thevenin equivalent circuit looking into the base at $AA'$.

	31. The Wien bridge is also useful as a frequency-selective network.

	32. The unit of potential difference is the volt.

	33. The unit of capacitance is the farad.

	34. The unit of resistance is labeled the ohm, after George Simon Ohm, who first
	discovered the relationship between current, voltage, and resistance.

	35. Tehis book is designed for managers who wish to learn about the technology,
	applications and scope of CAD/CAM.

	36. Capacitance depends on the size, shape, and separation between any two conductors.

	37. The useful power is ac circuits also depends on the current and voltage in the circuit.

	38. Microwave engineering is the branch of electrical engineering that deals with the
	transmission, control, detection, and generation of radio waves whose wavelength is short
	compared to the physical dimensions of the system.

	39. It is convenient to include the functions of a voltmeter, ammeter, and ohmmeter within
	one instrument.

	40. The prerequisite is a good knowledge of electric circuit fundamentals.

	41. The design of control systems of this kind requires a knowledge of the Z-transform and
	some aspects of information theory.

	42. The scpoe of this book does not premit ad detailed discussion of all of these 
	mathematical devices.

	43. A brief quilitative discussion of some basic concepts is presented in this paper.

	44. The manner in which the Wheatstone bridge is used may be understood from an analysis
	of the circuit.

	45. A quantitative analysis of this circuit is rather involved.

	46. A short calculation will convince you that this is indeed true.

	47. All circuit designs should include a calculation of thermal conditions.

	48. A more detailed descriptions of the operation of a transistor in saturation is given
	in Example.~2.

	49. A general knowledge of the characteristics of electrical transmission is essential if 
	the reader is to gain an understanding of data communications.
	
	50. It is often useful to be able to make a quick estimate of the quiescent current
	in a transistor amplifier.

	51. An examination of the two experiments shows that a definite relationship exists
	between current, voltage, and resistance.

	52. There is a growing awareness that this technique is of value.

	53. All the instruments in our laboratory are home-made.

	54. Both the devices here are very good in quality.

	55. It is necessary to determine how large a force is required to move this body.

	56. This manipylator can lift as heavy a weight as 450 kilograms.

	57. In practical applications, the power rating of a resistor is often as important
	a characteristic as its resistance value.

	58. The usual moving-coil galvanometer has too large a moment of inertia to follow
	the instantaneous values of an alternating current.

	\subsubsection{Numeral}

	1. Three fifths, one half.

	2. The voltage across the resistor is a few tenths of a volt.

	3. The resistance of the wire is a few hundredths of an ohm.

	4. By varying $V_{BE}$ only a few hundredths of a volt, the base current can be changed significantly.

	5. This is only a few thousandths of the heat of vaporization.

	6. 3 part per million, seven parts in a thousand.

	7. a third part in $10^6$, a seventh part in $10^3$.

	8. This voltage is 80 times the signal applied to the amplifier.

	9. Its speed is three tenths that of light.

	10. This wire is five times longer than that one.

	11. This value is nearly 4.5 times greater than the ideal one.

	\subsubsection{Preposition}

	{\bf of}

	1. Engineers may find the book of value as a reference on basic problems.

	2. What is described in this section is of great importance.

	3. Of all the computers in this laboratory, this one works best.

	4. Of the four parameters, three can be disposed of rather quickly.

	5. Of these, the first treats pulse and data communications, the next convers radar,
	and the last one introduces television.

	6. ECL III has the smaller propagation delay of the two ECL series.

	7. The lighter machine part has a mass of 7 kg.

	8. This device can supply 4 mA of output drive current.

	9. Fig.~2 shows the variation of the output with the input.

	10. Ellipses are used to describe the motions of the planets around the sun.

	11. The resolution of a force into $x-$ and $y-$component is possible.

	12. Exposure of the body to potentially toxic substances should be avoided.

	{\bf with}

	13. These experiments should be done with care.

	14. This parameter can be measured with accuracy.

	15. The conductivity of a semiconductor varies with temperature.

	16. With friction present, a part of power has been lost as heat.

	17. The equation to the circle with its center at the origin and of radius $a$
	is $x^2 + y^2 = a^2$.

	18. With the alternating current, things are different.

	19. With radar, we can see distant objects.

	{\bf by}

	20. In this case v and i differ in phase by $90^\circ$.

	21. Typical noise margins are usually better than the guaranteed value
	by about 75 mV.

	22. By Eq.~(3), we can obtain the following expression.

	23. By an examination of the performance of the device, we can understand its features.

	24. By analyzing this model, one can learn about the structure of an atom.

	{\bf for}

	25. The output may stay high for a long time.

	26. For $x>1$, this equation does not hold.

	27. For $x=1$, this factor equals unity.

	28. This book is too difficult for a beginner.

	29. It is necessary for us to solve this equation for $x$.

	30. There are several methods for antenna design.

	{\bf on/upon}

	31. The flip-flop will change the stored information only upon application of proper
	control signals.

	32. On being compressed, the volume of a substance will be reduced.

	33. On simplifying, the result becomes as follows.

	34. Upon rearranging the above equations, we get to the following set of equations.

	{\bf in}

	35. Here $t$ is measured in seconds.

	36. We measure resistance in ohms.

	37. Frequency is measured in hertz.

	38. Inductance is measured in henries.

	39. Radio waves travel in all directions.

	40. Direct current flows only in one direction.

	41. This computer is good in quality.

	42. These divices differ greatly in size.

	43. In our discussion of differential equations, we shall
	restrict our attention to equations of the first degree.

	44. Because of the difficulty in producing uniform films, it is bot possible to control resistance
	values precisely.

	45. The inclusion of $R_e$ causes a decrease in amplification.

	46. We can measure the slight change in pressure.

	\subsubsection{Verb}

	1. This graph looks (appears) puzzling.

	2. Its result proves correct.

	3. When an electric current flows through a wire, the wire will get hot.

	4. In this case, the output stays high.

	5. This problem remains to be solved.

	6. These experiments seem (appear) to indicate that there are only two kinds of
	electronic charge in the universe.

	7. Copper conducts electric current better than aluminium does (than does aluminium).

	8. These radio waves behave as light waves do.

	\subsubsection{Adverb}

	{\bf above, below, here, there, around, nearby, up, down}

	1. The table below lists resistivities of some substances.

	2. Molecules at the hot end of a rod vibrate faster and faster as the temprature there
	increases.

	3. The capacitance of a capacitor depends on the size of the plates and their distance apart.

	4. The problem now is to determine the magnitude of the current.

	5. One end of the rope is tied to a tree 10 ft away.

	6. The scientists 50 years ago could not do that.

	7. Usually some of these parameters are known.

	8. Simultaneously the base current in $T_1$ rises.

	9. Conventionally current flowing toward a device is designated 
	as positive.

	10. Thus, many combinations of $R_c$, $R_b$, and $h_{ie}$ will satisfy this requirement.

	11. The device consists mainly of five parts.

	12. The percentage of emitter current depends almost solely on the construction of
	the transistor.

	13. This leads directly to the circuit of Fig.~6.

	14. $h_{oe}$ varies exponentially with the collector-emitter voltage.

	15. The emf decreases only slightly with the collector-emitter voltage.

	16. At temperatures below the critical temperature, the electrons move freely throughout
	the lattice.

	17. This variation of $I_{CQ}$ with temperature is due primarily to variations in $V_{BE}$.

	18. These parameters are easily measured.

	19. Computers have been widely used.

	20. Other frequency dividers can readily be implemented using the configuration shown
	in Fig.~1-4.

	21. This parameter can also be obtained from the $vi$ characteristic.

	22. The reverse voltage gain can usually be neglected.

	23. It can similarly be shown that ECL gates have a lower noise margin than TTL gates.

	24. The minimum gate input voltage will reliably be recognized as logic 1.

	25. Noise may be internally generated.

	26. The noise voltage can be greatly attenuated.

	27. The ac and dc components can be treated separately.

	28. This technique is used extensively in Chap.~7.

	29. This complicated relationship must be represented graphically.

	30. This switching is accomplished rapidly.

	31. The advantage of this scheme may be seen qualitatively by refering to Fig.~12-3.

	32. This is written symbolically as $L=S_1\cdot S_2$.

	33. The longer vertical line always corresponds to the positive terminal.

	34. We now introduce the usual symbols used in electric-circuit diagrams.

	35. In the following discussion, we usually assume the emf of a source to be constant.

	36. The delay in $G_1$ far exceeds the delay in $G_3$.

	37. This figure clearly shows the test points.

	38. These two signals completely override all other inputs.

	39. As we go around the loop we easily find that $\bar{Q} = 1$ and $Q = 0$.
	
	40. The subscript $i$ thus (therefore) stands for input.

	41. Electronic switches do not readily perform the OR and the AND logic operations.

	42. You must use these rules correctly.

	43. We can solve the equations simultaneously.

	44. We should mark the direction of current clearly on the diagram.

	45. This may damage the gates permanently.

	46. Rows 5, 7, and 8 lead to the terms $X\bar{YZ}$, $X\bar{Y}Z$ and $XYZ$ respectively.

	47. We consider first some special cases.

	48. Fig.~1 represents schematically a few free paths of an electron in an electric field.

	49. We assume only that a current I is present in the circuit.

	50. We show below that the h parameters are each a function of the Q point.

	51. Suppose next that the circuit consists of a capacitor.

	52. Consider first that the base-emitter junction of $T_0$ will be forward-biased.

	53. To find these CB parameters, simply divide the corresponding CE parameters by 
	$(1+h_{fe})$.

	54. Now consider that all inputs are HIGH.

	55. The reverse voltage gain is usually negligible.

	56. $h_i$ is dimensionally an impedance.

	57. The final values of the resistors are then as follows.

	58. Our logic equation for L will thus be a sum of products containing eight terms.

	59. The Pr and Cl inputs cannot simutaneously be in the 1 state.

	60. A better solution is to effectively reduce $R_c$ by replacing it by an active resistance.

	61. These equations can be used to define completely the terminal behavior of the two-port network
	shown in Fig.~6-1.

	62. It is necessary to decrease propagation delay time dramatically.

	63. In this section we discuss the inverter primarily to illustrate the important characteristics
	common to all families of gates.

	\subsubsection{Adjective}

	1. This charge interacts with other charges present.

	2. In this case no rotation whatsoever will result.

	3. These are the smallest particles obtainable.

	4. This formula can be found in the physics books available.

	5. This measure is the key for the extremely low dc power available.

	6. Even in this case, there are two directions possible.

	7. The quantity of information we can send down the channel is approximately
	proportional to its range of frequencies usable.

	8. These series of positive terms corresponding will be as follows.

	9. The instruments necessary are conventional.

	10. The video bandwidth 18 MHz is equivalent to 1200 voice channels total.

	{\bf some,every,any,no + thing, body, one}

	11. Now there is nothing mysterious about computers.

	12. This book contains something new.

	13. Everything electronic will be done digitally.

	14. There is something peculiar that should be noted about the left side
	of the graph.

	15. This book is a htlp to circuit designers both new and old.

	16. Every object, large or small, possess gravitation.

	17. The power rule can be used for all rational exponents, positive and negative.

	18. The earth is surrounded by a magnetic field that affects the orientation of lodestones
	and other magnets, both natural and artificial.

	19. There are many problems, both technological and financial, that remain to be solved.

	20. Neutron has no charge, neither positive nor negative.

	21. The human body is made up of countless structures both large and small.

	22. The moving point $\rho$ describes a line, straight or curved, in the plane.

	23. No part of this book may be reproduced or utilized in any form or by any
	means, electronic or mechanical.

	24. The current equals the source emf divided by the total circuit resistance, external plus internal.

	25. Let us consider a current-carrying wire 10 meters long.

	26. The efficiency of the amplifier would be 50 percent, the maximum theoretically possible.

	27. Their electrical and magnetic properties, often paramount, are discussed in this chapter.

	28. Fig.~16 shows a situation a bit more complex.

	29. For belts 8 in. wide and over, use the second figure of the column.

	30. With the help of a computer, this machine can work unattended.

	31. Even if a student can follow every line of every example in this book, that doesn't mean
	that he or she can solve problems unaided.

	32. When a picture is taken of an object at a great distance, the rays from any one point on the
	object come into the lens almost parallel.

	33. The electrons able to move freely within a wire play an extremely inportant role
	in the formation of electric current.

	34. Infinity is a quantity greater than any number.

	35. An example of this is motion parallel to the earth's surface.

	36. The more the spring is stretched, the greater is the force necessary to stretch it.

	37. In the measurement of time intervals much shorter than a second, the decimal system is used.

	38. Something as small as a worm may be composed of millions of cells.

	39. A base of logarighms must be a positive number, not equal to one.

	40. Einstein put up the principle of relativity, valid not only in mechanics but
	in all physics.

	41. Simple in structure and low in price, this device is in gread demand.

	42. Accurate in operation and high in speed, computers can save man a lot of time and labor.

	43. Large or small, all the circuits will contain the same kinds of components.

	44. Analogous to the eyelid, the camera shutter opens for a predetermined length of time
	to allow light to enter through the lens and expose the film.

	45. Contrary to common belief, forces are not transmitted only by "direct contact".

	46. Free from the attack of moisture, a piece of iron will not rust very fast.

	47. The output voltages go through a minimum at $w=w_0$, very analogous to the situation 
	in series resonance.

	48. It must be noted that the current increases proportional to every decreast of resistance.

	49. The force acts perpendicular to the $x$-axis.

	50. The conductor is moving parallel to the magnetic field.

	\subsubsection{Pronoun}

	1. Physics is the most quantitative of the science, and we must become accustomed
	to its insistance upon accurate measurements and precise relationships if we are to
	appreciate its results.

	2. An ideal ammeter has a very low equivalent resistance so that its presence in any circuit
	alters the properties of the circuit as little as possible.

	3. Aiken's machine was limited in speed by its use of relays rather than electronic devices.

	4. The questions of convergence are not simple, and their study forms an important chapter
	in modern analysis.

	5. Many circuits texts seem to neglect nonsinusoidal signals and waveforms by postponing
	their introduction or avoiding much of the subject.

	6. Many scientists have worked at the theory of magnetism since its discovery.

	7. The limitation defined by Eq.~41 restricts its use in importanct situations.

	8. Chapter~8 is devoted to applications of the phase-locked loop and Chapter~9 to its analysis.

	{\bf one}

	9. By the "most efficient" algorithm one normally means the fastest.

	10. No one can see radio waves.

	11. One often thinks that friction is undesirable.

	12. Before one studies a system, it is necessary to define and discuss some important terms.

	13. If one wishes to detect objects, the radar transmitter may be used.

	14. A bridge several times stronger than needed to carry its heaviest possible load serves
	no one better.

	{\bf it, its, they, their}

	15. Before it can work, a computer must be told what to do.

	16. When they get hot, all metals melt.

	17. Because of its capacity to handle large volumes of data in a very short time,
	a computer may be the only means of resolving problems when time is limited.

	18. In their study of electricity, physics defined the electric field intensity $E$
	at a point in space as $E=F/q$.

	\subsubsection{Noun}

	1. "Basic Electronics" is a book much the same size as this one.

	2. Here $m$ is any number not zero.

	3. The fovea centralis is a small spot a few tenths of a millimeter in diameter.

	4. Oxygen has a mass about 16 times the mass of a hydrogen atom.

	5. They made hollow steel boxes the width of the bridge.

	6. They installed a $250W$ generator the size of a teapot.

	7. The cubic centimeter represents the volume of a cube one centimeter on an edge.

	8. The nucleus has a diameter only $0.01\%$ of that of the atom itself.

	9. Small transistorized equipment is often battery powered.

	10. The local oscillator is phase locked to the counter.

	11. This device can be computer controlled.

	12. Air cooling the equipment is necessary.

	13. The material in this book has been class tested.

	14. This control unit is instruction dependent.

	15. The container is water resistant.

	16. Transistors are temperature sensitive.

	17. Waves are able to bend around the edge of an obstacle in their path, a property
	called diffraction.

	18. In every normal atom, the number of protons equals the number of electrons,
	a fact which is directly related to the electrical properties of the proton and the electron.

	19. Upon making use of Eq.~1, the exact frequencies found in the hydrogen spectrum are
	obtained - a remarkable achievement.

	20. The car noses up when it accelerates, a familiar effect.

	21. If the base relation $I_C = \beta I_B$ were to hold there, we should have $I_C = 100 mA$
	and $U_{CE} = -90V$, an impossible situation.

	22. One of the principle applications of the diode is in the production of a dc voltage
	from an ac supply, a process called rectification.

	23. When we use this multiplication method it is not necessary to figure out
	all possibilities, a procedure which is often very lengthy, or even impossible
	from a practical point of view.

	24. One of these low-energy "thermal" neutrons will soon enter the uranium and
	cause the fission of a $U^{235}$ nucleus, a job at which thermal neutrons are particularly effective.

	25. When the applied force is less than $F_f$, the frictional force always equals the applied
	force. Otherwise, since $F_f$ acts in the opposite direction to an applied force, objects would
	move backward when pushed weakly, somthing that does not, of course, occur.

	26. The careful study of spectral lines shows that many of them actually consist of two or
	more separate lines that are close together, something that the Bohr theory cannot account for.

	27. An instrument for measuring electric resistance, the ohmmeter is widely used in electrical engineering.

	28. An electroacoustic transducer, the loud speaker converts audio-frequency power into acoustic power.

	29. This section deals with the advantages of transistors over electron tubes.

	30. Speed is defined as the ratio of distance to time.

	31. The distance of the sun from the earth is great.

	32. The superiority of radar to ordinary vision lies in the greater distances at which seeing
	is possible with radar.

	33. This chapter deals with the effect of temperature on transistors.

	34. The curve shows the variation of the current in the circuit with the applied voltage.

	35. The rate of decrease of power with increased case temperature is $\theta_{jc}$.

	36. The dependence of y upon (on) $x$ is expressed by $y = f(x)$.

	37. Ellipses are used to describe the motions of the planets around (round) the sun.

	38. We have discussed the passage of an electric current through liquid solutions of acids, bases and salts.

	39. The response of a body to a net force $F$ is an acceleration $\alpha$ proportional to $F$.

	40. The resolution of a force into $x$- and $y$-components is possible.

	41. These special problems arise from the use of atomic energy as a source of power.

	42. A comparison of Eq.~4 with Eq.~6 leads to the following relations.

	43. The energy radiated by the sun is due to the continuous transformation of hydrogen into
	helium in its interior.

	{\bf some special cases}

	44. These two emf's (or EMF's or EMFs) are equal.

	45. This section deals with analog-to-digital converters (A/D's).

	46. Manufacturers provide a large variety of IC's designed to effect proper interfacing.

	47. Special resonant components can have Q's as high as several thousand.

	48. All the R's and C's are made equal.

	49. Note that the g's have canceled out in the calculation.

	50. The computer can interpret the same binary configuration of 0's and 1's
	as data or as instruction.

	51. This is the origin of the $4\pi$'s in the mks values of $t_0$ and $\mu_0$.

	52. The effect of $h_{re}$ is therefore negligible in this (and most) examples.

	53. In this case both the resistor and the source absorb 0 watts.

	54. The transition time given in this data sheet represents the rise and fall
	times of the output wacvform.

	55. The control signal can arrive at different times without affecting the state of the output.

	56. This distortion is shown in dotted lines in Fig.~5 d to f.

	57. The amplitude varies 0.44 units above and below the zero axis.

	58. Usually two or three trys are sufficient to attain satisfactory agreements.

	\subsection{Misc}

	\subsubsection{Parenthesis}

	1. However, the following factors must be taken into account.

	2. Digital information, however, consists of discrete numerical values.

	3. The ripple is very much reduced by the double filtering action, however.

	4. For example, consider the map in Fig.~11.

	5. The wing design of a supersonic aircraft, for example, depends upon many factors.

	6. Consider, for example, the sum-of-products expression.

	7. Consider Theorem 3B, for example.

	8. We can take advantage, for example, of the high speed of Schottky TTL.

	9. The twin-T filter is equivalent in response characteristic to the Wien bridge.

	10. This circuit is similar in operation to the circuit of Fig.1.

	11. This increase is in turn limited by the increased collector resistance.

	12. More important, we see how to use the Fourier transform to describe and analyze systems
	and circuits. (more important $\approx$ what is more important)

	13. Potential energy, it is true, might not affect the sense of touch in the same way as kinectic
	energy would.

	14. This book is, we hope, a concise introduction to communications systems.

	15. Magnetism, you remember, is produced when electrons flow through a wire.

	16. What, you may wonder, does 54/74 stand for?

	17. Nuclear fission, it has been claimed, will be a cheap, chean and almost inexhaustible
	source of power.

	18. This, it will be noticed, is a real function due to the fact that the time function
	is an even one.

	19. The special revision units will, it is hoped, constitute a valuable aid in the task
	of consolidation.

	20. The other part of the solution is \[A e^{-at}\]
	which we recognize has the form of the natural response.

	21. Ohm's law has a few forms which it will be found are very useful.

	22. We could apply these two vectors, which it will be noticed are in polar form,
	to any network characteristic.

	23. In this case $I_C$ varies little with $U_{CE}$, which we think is very important.

	24. Determining the indefinite integral is called integration, which we can see is
	essentially the same as finding an antiderivative.

	25. Figure~3 pictures an addition of velocities that our common sense tells us must be right.

	26. Electrical inventors who followed Edison did not have to experiment with the substances
	which, he had found, would not work.

	27. Well-grounded students can absorb chapters 4 and 5 in say three weeks.

	28. A Boolean variable, say A, can take on only two values.

	29. If, say, $U_1 = 0$, the output can never become negative.

	30. All this is happening, remember, on a time scale measured in microseconds or picoseconds.

	31. If we dissolve, say, a spoon of sugar in water, the dissolved sugar molecules remain
	electrically neutral.

	32. The pump creates pressure of, let us say, forty pounds per square inch.

	33. We try to make x drop below say 5.

	\subsubsection{Denial}

	{\bf no, none, neither} for complete denial.

	1. None of these problems can be solved at once.

	2. No book is perfect.

	3. Neither of the devices is good in quality.

	{\bf every, both, all, not}

	4. All these values are not correct.

	5. Both books are not suitable.

	6. Students are not familiar with all these problems.

	\subsubsection{Tense}

	1. By 1980 the corporation had produced 100 radars.

	2. By the time students graduate, they will have taken up 30 courses in all.

	3. In recent years, they have been developing a new type of aircraft.

	4. These days he has been conducting an important test, but he has not finished it yet.

	5. In the 1940's, they designed the first generation of electronic computers.

	6. China launched another communications satellite three years ago.

	7. This paper deals with the features of modulated signals.

	\subsubsection{Morphology}

	1. As early as the 1820's it had been realized that this metal could be used in industry.

	2. At that time it was found that an electric current can produce the heating effect.

	3. If one wishes to send out radio waves, it is necessary to generate high-frequency oscillations.

	4. Actually, this result could have been arrived at intuitively.

	5. The name of the circuit is accounted for by the features of the amplifier.

	6. No mention has been made of this phenomenon in the book. (This phenomenon has been
	made no mention of in the book.)

	7. This point must be paid full attention to. (Full attention  must be paid to this point.)

	\subsubsection{Comparative}

	1. The sun is much brighter than other stars.

	2. This problem is much more complicated than that one.

	3. The distance of the terminal from the computing center is as great as three kilometers.

	4. The efficiency of as much as $99\%$ can be obtained in some transformers.

	5. The applications of computers are wider and wider. (The applications of computers are ever wider.)

	6. Electronic devices have become more and more complicated. (Electronic devices have become increasingly complicated.)

	7. The higher the frequency (is), the smaller the capacitive reactance (will be).

	8. The faster the body moves, the greater kinectic energy it will possess.

	9. Of all these machines here, this one works most satisfactorily.

	10. Of all the stars in the sky, the sun looks the biggest.

	\subsubsection{Verb as Predicate}

	1. It is necessary (for us) to have a good command of Kirchhoff's laws.

	2. It takes (some) dozens of seconds for a computer to solve this problem.

	3. It remains now (for us) to determine this coefficient.

	4. If it is desired to find out the voltage across this resistor, we must
	first know the current flowing through it.

	5. $R$ is the resistance to be mearsured.

	6. There are many problems for us to consider.

	7. We have deinfed length and time units with which to measure $\alpha$.

	8. A horizontal line can be taken as the axis along which to represent $n$.

	9. In this case, the best choice for the axis around which to calculate torques is the base of the ladder.

	10. We shall use such a field on which to base our discussion of magnetic properties.

	11. Energy is defined as the ability of a body to do work.

	12. A few factors affect the ability of a capacitor to store a charge.

	13. The interference of light waves puts limit on the ability of any telescope to resolve
	the details of an object.

	14. Gravity is the tendency of all objects to attract, and be attracted by, each other.

	15. The greater the tendency of an object to resist a change of velocity, the greater its inertia.

	16. Elasticity may be defined as the tendency of a body to return to its original state after being deformed.

	17. The capacity of air to absorb water vapor increases as its temperature rises.

	18. The deviations from the expected periodicity in Mendeev's list were due to
	failure of contemporary chemistry to have discovered some of the elements existing in nature.

	19. This computer is easy to operate.

	20. We find it very easy to solve this problem.

	21. We find this quantity difficult to measure.

	22. We have to find how large to make $x$ so as for this inequality to hold.

	23. For a transistor to function normally, it is necessary to apply proper voltages to its electrodes.
	(Usually not ``To make a transistor function normally'')

	24. The velocity of light is too great for us to measure in simple units.

	\subsubsection{Participle}

	1. These moving electrons form the current.

	2. The greater the resistivity, the greater the electric field
	needed to establish a given current density.

	3. The free electrons in a conductor carrying charge also play
	an important role in the conduction of heat.

	4. A capacitor is a device consisting of two conductors,
	separated by a non-conductor.

	5. In this case, the current flowing through a wire is directly
	proportional to the potential difference maintained.

	6. Distance is equal to speed multiplied by time.

	7. Current equals voltage divided by resistance.

	8. A lamp connnected across a dry cell is an example of a
	simple electric circuit.

	9. Its resistance decreases with the increased (increasing)
	temperature.

	10. Fig.~8 shows a cell connected across a load.

	11. Expressed in a formula, the resationship between voltage,
	current and resistance can be written as $U=IR$.

	12. Being a good conductor of electricity, copper is widely used
	in electrical engineering.

	13. Flowing through a circuit, the current will lose part of
	its energy.

	14. Given (Knowing) current and resistance, we can find out voltage.

	15. Lacking knowledge of just what these radiations were,
	the experiments named them simply alpha, beta, and gamma radiation,
	from the first three letters of the Greek alphabet.

	16. Having obtained the initial conditions, we go on to solve the
	network differential equations.

	17. Designed primarily for protons, this accelerator has achieved
	energies of 300 GeV.

	18. Known as a ``man-amplifier'', a machine under development consists
	of a framework that the operator wears.

	19. In these graphs the horizontal axis measures time, increasing toward
	the right away from the vertical axis.

	20. $U_{BE}$ varies between 0.3 and 0.7, depending on the base current.

	21. Silver is the best conductor, followed by copper.

	22. In this case, an emf is induced in the direction coinciding with that of current,
	(thus) opposing the decrease of current.

	23. A valence electron can move away from its atom, leaving it a hole.

	24. This reduction of Boolean expressions eliminates unnecessary gates, thereby
	saving cost, space and weight.

	25. Some of these electrons on the rod will leave it and transfer to
	the ball, giving the ball a negative charge.

	26. These points are plotted and joined, thus forming the curve which
	represents the function.

	27. In one of the trigonometric functions is known, we can determine the other functions
	of the angle, using the Pythagorean theorem and the definitions of the functions.

	28. Four node equations may be written based on an energy balance.

	29. All state changes take place following the occurrence of a pulse input.

	30. If you walk along a horizontal floor carrying a weight, no work will be done.

	31. Equation (5) may be solved using the Laplace transformation.

	32. This current changing, the magnetic field will change as well.

	33. An electron is about as large as a necleus, its diameter being about $10^{-12}$ cm.

	34. Almost all metals are good conductors, silver being the best.

	35. The sign of the integral depends on the direction of the path taken,
	a counter-clockwise direction begin taken as positive.

	36. There are several basic laws governing these interactions, all 
	of them discovered early in the nineteenth centry.

	{\bf with}

	37. With its base grounded, Q4 is very high impedance.

	38. With its base voltage 0, transistor Q1 will be cut off.

	39. With the key open, the voltage on the grid is equal to the bias voltage.

	40. With friction present, a part of power has been lost as heat.

	41. With Q2 on, the voltage fed to the base of Q3 rise.

	42. Each planet revolves around the sun in an elliptical orbit, with the
	sun at one focus of the ellipse.

	43. Both practical design techniques and thoretical problems are covered with emphasis
	on general concepts.

	44. Its phase angle will be $45^\circ$ with the current leading.

	45. All present stats are stable with no inputs present.

	46. The conditoin of resonance can be achieved with L and C
	either in series or parallel.

	47. The current gain $\beta_n$ shall be measured with $E_p$ grounded.

	48. The second term is evaluated similarly, with Z a function of y alone.

	49. The device with buttons on it is a keyboard.

	50. This is an inequality with zero on the right.

	\subsubsection{Verb+ing}

	1. Selecting a desired signal is only one of three important functions performed by the tuning circuit.

	2. Potential energy is capable of being changed into kineic energy.

	3. By analyzing the performance of the device, one can appreciate it better.

	4. Upon subsituing the actual magnitudes, $v$ turned out to be the velocity of light.

	\subsection{Clauses}

	\subsubsection{Adverbial clause}

	1. A rotating body possesses kinetic energy because its constituent particles are in motion.

	2. Since k and m are both constants, the ratio k/m is constant.

	3. Obviously, no current can flow since there is no circuit.

	4. Sunstroke differs from heat exhaustion in that one of the heat regulators is affected;
	namely, the sweat glands.

	5. As air has weight, it exerts force on any object immersed in it.

	6. Now that we have discussed that meaning of a graphical solution of a system
	of simultaneous euqations and the method of plotting a line, we are in a position
	to find graphical solutions of systems of linear equations.

	7. In previous chapters we did not use the trigonometric,
	inverse trigonometric, exponential, or logarithmic function, 
	for the dirivative of each of these is a special form.

	8. When a rocket is launched form the earth's surface, the thrust of its engines must
	exceed its weight for it to rise from the ground.

	9. It is possible for a body to remain in one place while it is rotating.

	10. While the transistor is off, the ac load line is horizontal.

	11. As radio waves travel along the surface of the earth, part of its energy
	will be lost.

	12. The foregoing provides a basis for this theorem, although it cannot be considered
	as a proof.

	13. Important though this law is, it is seldom used in practice.

	14. Small as electrons are, they play an important role in the formation of electric current.

	15. While x can only lie between -1 and +1, there are an infinite number of values of y for every value of x.

	16. Men are more susceptible to heart disease, while women are more prone to develop diabetes.

	17. At this time the kinetic energy approaches infinity, while the potential energy
	approaches the minimum.

	18. Input A goes low while input B remains high.

	19. While energy is the capacity to do work, power is the quantity of work in unit time.

	20. There is a growing awareness that these techniques are also of value in 
	some other areas.

	21. One of the main achievements is the recognition that properties of
	a material should be included in the analytic model.

	22. The question whether there is water on that planet will be discussed.

	23. The users have no guarantee how long this kind of device will be operating.

	Page 56.

	\subsection{Exercises}

	\begin{itemize}
\item
	A comparison of M and N shows that \ldots

	A comparison of M with N shows that \ldots

	A comparison between N and N shows that \ldots
\item
	Thus. Therefore. As a result. It follows that. so that. such that.
\item
	available. xxx we have (possess). The existing xxx. The current xxx.
\item
	The resultant current \ldots

	The resulting current \ldots

	The current which results \ldots
\item
	There is evidence that \ldots

	There is evidence that to show (indicate, suggest) that \ldots
\item
	The theory holds (maintains, claims, implies) that \ldots
\item
	There is (are) \ldots

	\ldots is (are) available.

	We have (possess, are in possession of) \ldots

	There is (are) available \ldots
\item
	In what follows, \ldots

	In what is to follow, \ldots

	In the following, \ldots
\item
	the graph (plot) of A as a function of B

	the graph (plot) of A versus (against) B

	the graph (plot) of the dependence of A on (upon) B

	the graph (plot) of the variation of A with B
\item
	The theory comes (stems, emerges, originates) from \ldots

	The theory is obtained (provided, furnished) from \ldots
\item
	This chapter deals with (treats, covers, describes, discusses, involves, bears on) \ldots

	This chapter is devoted to (confined to, concerned with) \ldots
\item
	These data lead us to a conclusion that \ldots

	These data lead us to conclude that \ldots

	These data enable us to conclude that \ldots

	(On the basis of these data, From these data),(one (we) can conclude that\ldots, it can be concluded that \ldots)
\item
	\ldots remain(s).
	
	\ldots stay(s).

	\ldots is (are) kept (held, maintained, left) unaltered (constant, unchanged, fixed, the same).
\item
	be inversely proportional to

	vary inversely with

	depend inversely as
\item
	The result obtained agress with (is in agreement with, is in line with, is consistent with, fits into) the computer simulation.
\item
	\ldots small in size (good in quality; light in weight; low in price)
\item
	The input can be represented (expressed, given, denoted) by $r(t) = sin(t)$.
\item
	There is no doubt that \ldots

	There is no question that \ldots

	Without doubt (question), \ldots

	Beyond doubt (question), \ldots

	No (out of) doubt, \ldots
\item
	One of the great advantages of this method is its great simplicity. (\ldots is that it is very simple).

	Among the great advantages of this method is its great simplicity.
\item
	This method has many advantages over those available. (over the existing ones)

	This method is advantageous in many respects (as) compared with those available. (with the existing ones)
\item
	\ldots is widely used. (is in wide use, is widely in use)
	
	\ldots find(s) wide application. (use)
\item
	G is an indication of ...

	G gives an indications of ...

	G indicates ...

	G is indicative of ...
\item
	substituing M in (into) N, we obtain (have, get) ...

	substituing M (substitution of M) in (into) N gives (yields, produces, results in)
\item
	This experiment failed to show (demonstrate) ...

	This experiment has not shown (demonstrate) ...
\item
	We have performed (done, made, conducted) a number of experiments to test (verify, prove, check)
	the theory.
\item
	This euqation holds for (holds true for, is true for, is valid for, applies to)...
\item
	We have (succeeded in doing, been successful in doing, successfully done)
\item
	This has nothing (little, somthing, much) in common with ...
\item
	It is (most) desirable (to (do), that)

	It would be best if ...

	\end{itemize}

\section{Tech Trans}

\subsection{Overview}

1. Today it can be said that wheels run America. The four 
rubber tires of the automobile move America through work
and play. Wheels spin, and people drive off to their jobs.
Tires run, and people shop for the week's food at the big 
supermarket down the highway. Hubcaps whirl, and the whole
family spend a day at the lake. Each year more wheels crowd
the highways as 10 million new cars roll out of the factories.
其中“转动,运转”之意 run, move, spin, turn, whirl, roll.

2. Contraction during cooling of heated solids is made use of
in riveting.

3. Attempts were made to regenerate the catelyst by passing oxygen
at the rate of $8 cc/sec$ for 4 hours at $300^\circ C$.

4. Electrical energy can be stored in two metal plates separated by
an insulating medium. Such a device is called a capacitor, or a
condenser, and its ability to store electrical energy (is called)
capacitance. It is measured in farads.

5. The signals are grouped into packages and a number of packages
(are) combined into a composite signal.

6. The speed against the wind is 20 miles per hour and 
(the speed) with the wind (is) 80 miles (per hour).

7. The first program would require a minimun of 108 weeks,
(while) the second (program would require) only 72 weeks.

8. T will represent temprature alone in this book unless
(it is) otherwise stated.

9. Most elements as (they are) found in nature are mixtures
of isotopes.

10. The more mixing that occurs in the estuary (is),
the greater is the landward flow in the subsurface layer.

11. (Since it is) Composed of several different pure
substances, air is a mixture.

12. The improvement of tensile properties depends indeed on
the procedures of heat treatment, as (it is shown) in Fig.~5.

\begin{framed}
固定表达:

as described above

indicated in Fig.X

as noted later

as previously mentioned

as shown in Table X

if any/anything

if convenient

if necessary

if not

if possible

if required

if so

when in use

when necessary

when needed

where feasible

where possible
\end{framed}

13. The wind (that is) directed inward around the center makes
angles of $20^\circ$ to $40^\circ$ with the isobars.

14. A scalar quantity is one (which is) defined by its magnitude
alone.

15. \emph{Thus} presented (=In the form in which they have been presented),
the figures give no useful information.

16. These small units have an electrically compressive pump,
hence (they have) the name "electric" refrigerator.

\begin{framed}
带形式主语的惯用陈述句型

It appears that

It can be seen that

It follows (from this) that

It has been proved that

It has been shown that

It is believed that

It is evident that

It is necessary to point out that

It is not hard to imagine that

It is possible that

It is well known that

It may be remarked that

It must be noted that

It should be mentioned that

It was reported that

It will be found that

带表语或表语从句的惯用陈述句型

Of importance is/are

Of recent concern is/are

Particularly noteworthy is that

The case is that

The chance are that

The conclusion is that

The fact is that

The purpose of this paper is

The question is that

含宾语或宾语从句的惯用句型

Calculations indicated that

Experience has shown that

Fig.~X illustrates

Practice has shown that

Results demonstrate that

Tests have proved that

This implies that
\end{framed}

17. If lines, called isanomals, are drawn on a world map, joining places
of equal thermal anomaly, an isanomalous map is the result.

18. More research needs to be done, if this problem is to be solved.

19. If (they are) put together and heated to $25^\circ C$ at constant
pressure, these substances will result a new compound.

一般现在时:

20. Electronics \emph{is} the basis of all telecommunication systems.

21. Many years ago, scientists were convinced that malaria \emph{is} caused
by a germ carried by a certain species of mosquito.

22. Common salt \emph{dissolves} in water.

23. Figure 2 shows the principal layout of an oil refinery.

24. A: Tin resists corrosion by air or water. 
    B: Tin is resistant to corrosion by air or water.
    C: Tin has resistance to corrosion by air or water. 

现在完成时:

25. The reaction has already come to the end.

26. One of the most striking characteristics of modern science has been
the increasing trend towards closer cooperation between scientists and
scientific institutions all over the world.

27. Pascal carried a mercury tube to the top of a mountain and observed
that the column fell steadily as the height increased.

28. The highest dry weight is shown for culture A, which recieved the
greatest amount of the ammonium salt. This may mean that the amount of 
nitrogen added was the determining condition for the experiments.

29. Rice grew better, under the other conditions of these tests,
when ammonium sulphate was added to the soil.

30. The jet stream was not fully recognized as a meteorological
entity until 1949.

过去完成时:

31. It was reported that scientists had worked at the problem of 
storing the sun's heat for many years.

32. The data had no sooner been charted than analysis was started.

一般将来时:

33. The scientists and technicians will carry out a very important
test next month.

34. Water will boil at $100^\circ C$.

35. Liquids will expand and contract like gases.

36. A gas under compression will become hotter.

大量长句,科技英语特色:

37. In laser hardening, the surface of the workpiece, coated with
a substance (\emph{e.g.} graphite or manganese phosphate) which 
increases its absorptivity, is heated to austenitizing temperature
by impingement of the beam at high power densities; on removal of
the beam, cooling, sufficiently fast to promote martensitic
transformation, takes place by heat transfer to the body of the
component, noexternal cooling being applied.

38. Depending on the application, data base handling in a multimini
system may be assigned to a special-purpose back-end data base
management processor that usually can be accessed by all the minis
in the system, or the data bases may be distributed throughout
the system in such a way that transfer of raw data is to site of the 
data base and computing udpates are performed in a manner that minimizes
the load on the communications facility.

大量动名词:

39. Oxygen is also injected upwards into the molten bath through
the refractory walls providing close control of thermal and chemical
conditions, superior process flexibility, high reaction rates and
minimization of refractory, dust, and other problems encountered with
conventional tuyeres or top-blowing methods.

40. Before cutting metal you should carefully clean the machine.

41. The screw is also used in calipers for measureing very accurately
the diameters of small objects.

42. To liberate oxygen from this compound is rather difficult.

43. When oxygen is blown into molten pig iron, silicon begins
to oxidize first.

44. A: They suggest that my students take part in this project.
B: They suggest my students taking part in this project.

45. A: The sun gives the solar cell its constant supply of energy.
B: The sun gives the solar cell the energy it constantly supplies.

46. A: I like to surf the internet this evening.
B: I like surfing the internet in the evening.

47. The reaction is helped by using a catalyst.

48. A: Television can transmit and receive images of moving objects
by radio waves, which is its basic function.
B: The transmissino and reception of images of moving objects by radio
waves is the basic function of television.

49. For this new type of car, the attainment of the very high speeds
is rather quick and easy.

50. The experimental demonstration of the existence of two kinds of
lead, identical in chemical properties and X-ray spectra but different in
properties involving atomic mass, was the first proof of the existence
of isotopes.

大量采用非谓语动词:

51. Whenever occasion arises intensive observations are to be done to 
get the information on the fine structure of the atmospheric event.

52. During periods of light winds or reduced river flow coastal
currents often become so weak as to be barely discernible.

53. The cutting tool must be harder than the material to be cut.

54. The heat causes the reaction to be accelerated.

55. In communications, the problem of electronics is how to convey
information from one place to another.

56. In order for the tide waves to move fast enough at the equator
to keep up with the moon, the ocean would have to be about 22 km deep.

57. Radiating form the earth, heat causes air currents to rise.

58. Combined with extremely high waves generated by the storm,
hurricanes surges can be highly destructive.

59. Pure iron is a silver-white metal, melting at $1535^\circ C$.

60. Built in ancient times, the imperial graves have now become one
of the most exciting archaeological sites.

61. Two fronts will exist if we have three air masses, a cold, a warm,
and a cold, moving from west to east one after another.

62. The result of the experiment is rather disappointing, so it has to
be done all over again.

63. Other things being equal, the evapotranspiration rate decreases
with increasing humidity.

64. One of the fluids flows through the outside pipe, the other 
through the inside pipe, with heat exchange taking place through the wall.

65. One body never exerts a force upon another without the second
reacting against the first.

66. On being heated, air trapped in the gas bag expanded, so the ballon rose.

67. Series circuits have the advantage of increasing resistance to reduce
current when this is desirable.

68. Cutting metal is no simple operation.

69. This process is called directional cooling.

70. There is something wrong with the heating coil in the device.

71. This accounts for melting ice and freezing water having the same
temprature.

词性转换:

72. The oil tanker carries crude oil to a refinery, where to oil
is processed.

\subsection{Introduction}

73. To be sure, ambition, the sheer thing unalloyed by some lager purpose than merely
clambering up, is never a pretty prospect to ponder.

74. There are always crossed transverse steady and longitudinal alternating fields.

75. This possibility was supported to a limited extent in the tests.

76. The development of an economical artificial heart is only a few transient failures away.

77. Rubber is not hard; it gives way to pressure.

78. Each product must be produced to rigid quality standard.

79. The petrol engine has been used to drive almost every kind of thing
that runs on wheels; it is also used for aeroplane and small boats.

80. Angina pectoris mean pain in the chest, a symptom which accompanies
any interference with blood supply or oxygenation of the heart muscles.

81. The engineering ceramics consists mainly of the so-called refractory
oxides and the borides, carbides, nitrides, and silicides of many of
the transition metals.

82. Metallic iron contents were determined by electrochemical solution of the
iron with copper nitrate solution.

83. And I look forward to seeing the brilliance of the new China
when all your people are connected and China is a major force in the networked
world.

84. The ability of the crystals to form the cubes as they grow, with plane
faces at right angles to on another, and their cubic cleavage, the ability to
be cleaved along cube faces when struck a blow, are both determined by the
cubic symmetry of the ionic arrangement.

85. The rate of dissociation was followd by placing the unit on a scale and 
noting the weight loss due to chlorine evolution.

86. Charles is now with his children in Los Angeles; it is already
ten years since he was the CEO of a famous company.

87. The principles of aluminum foil and strip manufacture have been
the subject of a series of investigations recently.

88. For example, the artificial modification of rain at New York City
may suggest a seeding activity just a few miles upwind of the metropolis.

89. Since earlist time, men have watched the sky, the wind, and
the atmosphere, as well as a variety of other weather signs, in 
order to make weather forecast.

90. The idea here is to locate in the weather records conditions
that are as analogous to the current conditions as possible.

91. A gas occupies all of any container in which it is placed.

92. The most effective method of removing this acid cotaminant
is to cool and then neutralize the exhaust gasses.

93. The flow sheet shown in Fig.~2 is intended to illustrate the
SL/RN process.

94. It's my belief that this confluence of a networked economy,
China's reforms, embrace of new technologies and a rapidly growning
economy puts China in a unique postion.

95. It is not only possible but altogether probable that by
diminishing nature we diminish ourselves, and vise versa.

96. But I hated Sakamono, and I had a feeling ht'd surely
lead us both to our ancestors.

97. Nothing could be done. (我们无能为力)

98. Do you see any green in my eye?

99. The officer asked him not to let the cat out of the bag.

100. The patient chair is adjustable in three dimensions and precisely
controlled by a joystick mounted right in front of the microscope where
doctors can easily reach and accurately adjust prior to operation.

101. Of all elements, hydrogen is the lightest one.

102. By far the largest number of the start that we see through
a powerful telescope are in the Milky Way System.

103. Friction always oppose the motion whatever its direction may be.

104. Cutting tools should be inspected carefully before they
are used.

105. The hydrogen produced by electrolysis is nearly pure,
though rather more expensive than that obtained by the thermal cracking process.

106. Powder coal injection was apparently accomplished as early as 1948 in Russia.
However, the first commercial scale trial reported by a western country
was carried out by National Steel and Bethlehem Steel in 1962.

107A. The thicker is the wire, the smaller is the resistance.

107B. The air can absorb more water, the higher its temperature is.

108A. Where there is little train, the reservoir would provide us water.

108B. A signal will be shown wherever anything wrong occurs with the control system.

PASS

\appendix
\begin{thebibliography}{9}
	\bibitem{bib:techwrite.xidian}
		秦狄辉,科技英语写作教程,西安电子科技大学出版社
	\bibitem{bib:techtrans.science}
		闫文培,实用科技英语翻译要义,科学出版社

\end{thebibliography}

\end{CJK}
\end{document}
