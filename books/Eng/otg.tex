%% LyX 2.3.3 created this file.  For more info, see http://www.lyx.org/.
%% Do not edit unless you really know what you are doing.
\documentclass[10pt,twocolumn]{article}
\usepackage{mathptmx}
\usepackage[T1]{fontenc}
\usepackage[utf8]{inputenc}
\usepackage{CJKutf8}
\usepackage[margin=1in]{geometry}
\usepackage{indentfirst}

\begin{document}
\begin{CJK}{UTF8}{gkai}

\section*{Memo: Official TOEFL Guide}

\verb|Dec 2019 M. Zhou|

\section{Reading Section}

阅读部分由3~4篇文章构成,每篇文章字数约为700。每篇文章有12~14道题。考试时间60分钟。
托福阅读涉及许多不同学科,考生不一定需要了解所涉及的主题,因为回答问题所需要的所有
信息都会在文章中出现。托福阅读文章一般分为三种类型:(1)解释说明型;(2)立论型;(3)历史题材型。
阅读过程中,考生需要特别注意理解全文的结构。常见的文章结构大致有以下几种:分类,比较/对比,原因/结果,问题/解决方法。

阅读部分考察学生寻找基本信息的能力,推理能力以及理解文章内容的能力。题目类型分为10种。

\subsection{Basic Information and Inferencing Questions}

\begin{enumerate}
	\item Factual Information Questions. 识别文章明确阐述的事实信息。通常相关信息在文章中只是一句或两句话。
		常见的提问方式比如:"Accurding to paragraph, which of the following is true of X?"。
		回答此类题目可能需要重新阅读文章以考证文章中对应的阐述,因为考生第一次阅读可能会记不住这些信息。
		其次,考生可以将与文中信息矛盾的选项直接排除。需要注意的是,不能因为某个选项在文中被
		提及就将其视作正确选项,也不能选择看似正确但是完全没有提及的选项。
	\item Negative Factual Information Questions. 明确文章陈述信息,确定选项中的真实选项,不真实选项以及
		未提及选项。回答此类问题时,首先定位相关信息在原文中的位置,再作出判断。此类问题非常容易辨别,
		因为题目中会出现大写的"EXCEPT"或者"NO"。否定事实信息问题需要回顾原文内容较多,三个曾被提及的选项
		可能分布在一个或者几个段落中。这类问题的正确答案要么是与原文产生矛盾,要么就是原文并未提及。
	\item Inference Questions. 考察对文章观点或者立场的理解程度。这种观点或者立场在文中会强烈暗示,但是不会
		直接阐明。处理此类问题时,考生需要注意相关句子之间的逻辑暗示。此类问题特征也相对明显,比如会出现
		"infer", "suggest", "imply" 等词语。具体解答时,首先确定所选答案与文章主要观点不矛盾。正确答案一定
		可以从文章推断得出,而看似正确但是文章中不能直接推断得到的选项不正确。作出选择后,考生应该能明确指出
		其在原文中对应的位置。
	\item Rhetorical Purpose Questions. 修辞类题目可能会要求考生确定一个段落与另外一个段落之间的联系,或者
		提问作者提到某一条特别信息的原因。此类题目通常不会考察整篇文章的结构。求解时需要特别注意句子或
		段落之间的逻辑联系。
	\item Vocabulary Questions. 确定个别单词或者短语的具体意思。一个单词可能有不同意思,但是在选项中只会有一个
		与文章中的意思相符。词汇类问题非常容易辨认。这类问题不仅仅考察单词基本意思,更重要的是考察单词在上下文
		中的意思。回答时可以将选项代入回原文,确认意思仍然一致。
	\item Reference Questions. 考察单词之间指代关系。这类问题与词汇类相似。考生要确保所选答案与被标记的代词具有
		数(单复数)和格(第几人称)的一致性,也可以将选项替换如原文检查是否违反语法规则,是否能使句子通顺。
	\item Sentence Simplification Questions. 选出一个与原文某个句子意思相同的选项。与高亮句子矛盾,或者遗漏重要
		信息,信息产生偏差的选项都是错误的。
	\item Insert Text Questions. 选择一个新句子在文中最恰当的插入位置。重点在于文章的结构和逻辑关系,尤其要注意逻辑
		连接词。常见的连接词比如: on the other hand, for example, on the ontrary, as a result, further, furthermore,
		therefore, in other words, similarly, in contrast, finally。另外,需要确保被插入句子与前后句子逻辑上的
		衔接,以及代词与被指代对象的一致性。
\end{enumerate}

\subsection{Reading to Learn Questions}

这类问题涉及以下方面:确认文章结构和写作目的,将
脑海中的信息整理成框架,区分主要观点/信息和次要观点/信息,理解各种修辞手法,
例如因果,对照,论点论据等。

\begin{enumerate}
	\item Prose Summary Questions. 首先排除错误信息,次要信息和未提及信息,再作出后一步选择。
	\item Fill in a Table Questions. 首先排除干扰选项。错误答案文中可能未提及,或者与表格内容不相关。需要注意的是,
		错误选项中同样可能会含有与文章原来词句像匹配或类似的单词短语。
\end{enumerate}

\subsection{Strategies for Preparing for the Reading Section}

提高阅读技巧最好的方法是大量阅读各种类型的文章(自然科学,社会科学,艺术,商业等)。
最好要定期阅读学术类文章。

针对托福阅读三种命题目的的建议:
\begin{enumerate}
	\item Reading to find information. 浏览文章,标记重要事实(时间,数量,术语)和信息。不断练习以提高阅读效率和流畅性。
	\item Reading for basic comprehension. 扩充词汇量。练习迅速通读全文把握主旨,不在个别单词上卡顿。通读一段文字后,
		再次仔细阅读并确定主要观点信息和重要事实。找出生词并通过上下文猜测意思。挑出所有代词并确定它们的指代对象。练习
		通过文章整体进行推断总结。
	\item Reading to learn. 确定文章类型(因果,对照,问题求解,描述,叙述等)。拟定文章大纲并区别主要和次要信息。
		利用表格或者大纲对文章进行总结。转述文中的句子,继而转述整个段落。转述对于综合写作和口语也会有帮助。
\end{enumerate}

\section{Listening Section}

托福听力部分有4~6个讲座和2~3个对话。每个讲座后有6个问题,而每个对话后有5个问题。
听力考试时间为60~90分钟。每段音频长度为3~6分钟。对话中可能会出现包括口误在内的常见
口语现象。听的过程中应该要做笔记。听力考察的不是记忆能力,而是理解能力。

对话类。对话场景主要为办公室和学生服务,以大学校园为背景。

讲座类。讲座内容设计大学各学科的入门级课程知识,范围广泛。考生无需对话题相关
学科背景有预先了解,因为回答问题所需要的所有信息都在讲座中给出。听力考试中
讲座的主要话题类别为Arts, Life Science, Physical Science, Social Science几类。

\begin{itemize}
	\item \textbf{Arts.} Architecture, industrial design/art, city planning,
		crafts (weaving, knitting, fabrics, furniture, carving, mosaics, ceramics, folk and tribal art),
		cave/rock art, music and music history, photography, literature and authors,
		books, newspapers, magazines, journals.
	\item \textbf{Life Science.} Extinction of or conservation efforts for animals and plants,
		fish and other aquatic organisms, bacteria and other one-celled organisms,
		viruses, medical techniques, public health, physiology of sensory organs,
		biochemistry, animal behavior (migration, food foraging, defenses),
		habitats and the adaptation of animals and plants to them,
		nutrition and its impact on the body, animal communication.
	\item \textbf{Physical Science.} Weather and atmosphere, oceanography,
		glaciers, glacial landforms, ice ages, deserts and other extreme environments,
		pollution, alternative energy, environmental policy,
		other planets' atmosphere, astronomy and cosmology, properties of light, optics,
		properties of sound, electromagnetic radiation,
		particle physics, technology of TV, radio, radar, chemistry of inorganic things,
		computer science, seismology (plate structure, earthquakes, tectonics,
		continental drift, structure of volcanoes).
	\item \textbf{Social Science.} Anthropology of nonindustrialized civilizations,
		early writing systems, historical linguistics, business, management,
		marketing, accounting, TV/radio as mass communication,
		social behavior of groups, community dynamics, communal behavior,
		child development, education, modern history (including the history of urbanization and
		industrialization and their economic and social effects.
\end{itemize}

\subsection{Basic Comprehension Questions}

内容主旨,目的主旨,细节三种考察角度。

\begin{enumerate}
	\item Gist-Content Questions. 内容主旨。提问的特征词是What。对听到的内容归纳总结,考察的是总体内容,所以可以
		排除只涉及局部听力材料的选项。练习时,可以试着用一个短语或一句话来概括音频内容的主题。
	\item Gist-Purpose Questions. 目的主旨。提问的特征词是Why。考生应该注意每段对话的最初目的是什么。
		另外需要注意,对话并不总是与主题相关。
	\item Detail Questions. 听懂并记住音频中明晰的细节和事实。这些细节可能是直接或间接地为文章主题提供例子,解释或者其他支持内容。
		有些情况下,听力材料也会出现与主题不太相关的内容,这些细节也可能会考。考生回答问题时需要参考笔记,
		不过注意题目不会考察一些小的细节。笔记内容应该倾向于重要细节。作出选择时,需要注意包含听力文本
		原词不意味着对应选项是否正确。如果无法确定正确选项,则选择与内容最符合的选项。
\end{enumerate}

\subsection{Pragmatic Understanding Questions}

情景理解,需要考生听出隐含意思。在大多数情况下,情景理解考察的内容并没有直接阐述,
说话人的字面意思和实际意思之间存在差异。情景理解题通常让考生重新听一小部分内容,
并分析句子功能或者说话人态度。

\begin{enumerate}
	\item Understanding the function of what is said. 记住,说话人的字面意思可能和实际意思不符。
	\item Understanding the speaker's attitude. 除了上下文外,说话人的语气也可以帮助回答此类问题。
\end{enumerate}

\subsection{Connecting Information Questions}

信息整合,推断,结论,预测。考生需要能够识别并解释各种观点细节之间的联系。

\begin{enumerate}
	\item Understanding organization questions. 某一句话在上下文中的功能。这些功能包括:
		转换话题,连接主题和分论点,提供引言或结论,距离,离题,甚至玩笑。此类问题更有可能
		考察整篇文章的组织结构。教授不会一开始就按照非常规则的顺序来进行讲座。此外,注意
		教授作出的比较,以及用熟悉事物解释新概念的方法。
	\item Connecting Content Questions. 观点之间的联系。题目可能会要求考生按照与
		原文不同的方式来组织信息。填表和排序题可以归为此类型。考生练习过程中,
		注意训练自己做笔记的方式。听清楚术语和定义,听清楚事件发展中的各个步骤,
		这些有助于回答此类问题。
	\item Making Inferences Questions. 根据听到的内容得出结论。有时候教授可能会暗示这些结论,而不是直接说明。
		大多数情况下,正确答案往往使用了原文中没出现的词汇。
\end{enumerate}

\subsection{Strategies for Preparing for Listening Section}

考试四个部分中三个都考察了听力。听力对无论考试还是学习生活都非常重要。提高听力的
最佳方法是经常听不同领域的各类材料。

Listening for basic comprehension. 扩充词汇量。听的时候注重内容和大意,而不要受
说话人风格和方式干扰。预测将要说的话,并在听到新的内容后及时调整预测。主动向自己
提问比如教授的主要观点是什么这样的问题。将main idea, major points, important details
这些在纸上列出,并提醒自己注重这些方面的笔记。反复听材料,直到记下所有重要观点和细节。
边听材料一边写出主要观点或者总结。

Listening for pragmatic understanding. 思考说话人的目的,注意说话人的说话方式,确定程度,
语调和重音,以及在表达反对或者提建议时采用的间接方式。

Listening for connect ideas. 思考讲稿中的布局,注意材料中的标志词。它们会标示引言,
主要步骤或观点,例子,结论,总结。弄清楚材料之间观点的关系,注意可能表示关系的词汇。
听录音时,预测下文将出现的信息和观点。听完后,列一个内容的提纲。注意偶尔会发生的话题转换。

Tips for the day of the test. 记笔记。考试只考察重点,不要记录所有细节。
听讲座时,注意教授的新名词或者新概念,讲座的内容结构以及意义连接。
考试过程中,听力问题必须按顺序回答。一旦考生选择确定,则不能返回前边的题目。

\end{CJK}
\end{document}
