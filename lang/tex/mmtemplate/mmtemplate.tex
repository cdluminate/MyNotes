\documentclass[10pt,a4paper]{article}
% use Cantarell (Gnome) as default English font
\usepackage[default]{cantarell}
% CJK utf8
\usepackage{CJKutf8}
% CJKulem : such as uline
\usepackage{CJKulem}
\usepackage{geometry}
\usepackage{listings}
\usepackage{booktabs}
\usepackage{amsmath} 
\usepackage{amssymb} 
\usepackage{fancyhdr}
\usepackage{indentfirst}
\usepackage{graphicx} 
%\usepackage{natbib}
\usepackage{float}

\usepackage[colorlinks,linkcolor=blue,anchorcolor=blue,citecolor=blue]{hyperref}

\begin{titlepage}
\title{数学建模论文模板:这是论文题目}
\author{cdluminate@gmail.com}
\date{\today}
\end{titlepage}

% 论文用白色A4纸打印;上下左右各留出至少2.5厘米的页边距
\geometry{left=3cm,right=3cm,top=3cm,bottom=3cm}

% ----start document----
\begin{document}
\begin{CJK}{UTF8}{gbsn}
\thispagestyle{empty}
% generate the title page
\maketitle
% this cancels page number display.
% \thispagestyle{empty}

\section*{关键词}
这里填写关键词们

\section*{Abstract/摘要}
\subsection*{部分论文格式要求}
论文题目、摘要和关键词写在论文第三页上(无需译成英文),并从此页开始编写页码;
页码必须位于每页页脚中部,用阿拉伯数字从“1”开始连续编号。
注意:摘要应该是一份简明扼要的详细摘要,请认真书写(但篇幅不能超过一页)。

\newpage

\section{正文长度应当控制在20页以内}
我们需要正确引用参考文献,比如此模板是根据全国大学生数学建模竞赛论文格式规范(2012年8月修订版)\cite{bib:plate}创建的。
	\begin{verbatim}
	参考文献按正文中的引用次序列出,其中书籍的表述方式为:
	[编号] 作者,书名,出版地:出版社,出版年。
	参考文献中期刊杂志论文的表述方式为:
	[编号] 作者,论文名,杂志名,卷期号:起止页码,出版年。
	参考文献中网上资源的表述方式为:
	[编号] 作者,资源标题,网址,访问时间(年月日)。
	\end{verbatim}
这三条要求用\LaTeX的\underline{thebibliography}很容易达成。

\section{正文节2}
我在本模板中忽略了规范\cite{bib:plate} 所要求的第一页和第二页,因为用户完全可以自行打印那两页然后连同本文档一起装订。
这样也可以免去在本tex模板中对那两页进行白白浪费时间的排版。

\section{数学建模的一般步骤}
此章节参考了一本数学建模教材\cite{bib:mm},旨在填充本模板。
\subsection{模型准备}
\subsection{模型假设}
\subsection{模型构成}
\subsection{模型求解}
\subsection{模型分析}
\subsection{模型检验}
\subsection{模型应用}

\section{公式示例}
比如这是有名的 Fourier Transform :
\begin{equation}
	F(j\omega) = \int_{-\infty}^{\infty} f(t) e^{-j\omega t} dt
\end{equation}
及其逆变换:
\begin{equation}
	f(t) = \frac{1}{2\pi} \int_{-\infty}^{\infty} F(j\omega) e^{j\omega t} d\omega
\end{equation}

\newpage
\appendix
\begin{thebibliography}{10}
	\bibitem{bib:mm} 姜启源,谢金星,叶俊,数学建模(第三版),北京:高等教育出版社,2003
	\bibitem{bib:plate} 全国大学生数学建模竞赛组委会,\\
		全国大学生数学建模竞赛论文格式规范(2012年8月修订版),\\
		http://www.mcm.edu.cn/html\_cn/node/896257972ddd6961d0250d2431522904.html\\
		\today
\end{thebibliography}

\section{其他}
在论文纸质版附录中,应给出参赛者实际使用的软件名称、命令和编写的全部计算机源程序(若有的话)。同时,所有源程序文件必须放入论文电子版中备查。论文及源程序电子版压缩在一个文件中,一般不要超过20MB,且应与纸质版同时提交。(如果发现程序不能运行,或者运行结果与论文中报告的不一致,该论文可能会被认定为弄虚作假而被取消评奖资格。)
\subsection{实际使用的软件}
\begin{enumerate}
	\item GNU C Compiler
	\item Python 3.4
	\item Matlab R2013a UNIX
	\item LINGO xxx
	\item your programs
\end{enumerate}
\subsection{源代码}
\subsection{xxx}
xxx

\end{CJK}
\end{document}
% ----end document----
