% This version of CVPR template is provided by Ming-Ming Cheng.
% Please leave an issue if you found a bug:
% https://github.com/MCG-NKU/CVPR_Template.

\documentclass[review]{cvpr}
%\documentclass[final]{cvpr}

\usepackage{times}
\usepackage{epsfig}
\usepackage{graphicx}
\usepackage{amsmath}
\usepackage{amssymb}
\input{math_commands.tex}

% Include other packages here, before hyperref.

% If you comment hyperref and then uncomment it, you should delete
% egpaper.aux before re-running latex.  (Or just hit 'q' on the first latex
% run, let it finish, and you should be clear).
\usepackage[pagebackref=true,breaklinks=true,colorlinks,bookmarks=false]{hyperref}


\def\cvprPaperID{****} % *** Enter the CVPR Paper ID here
\def\confYear{CVPR 2021}
%\setcounter{page}{4321} % For final version only


\begin{document}

%%%%%%%%% TITLE
\title{\LaTeX\ Author Guidelines for CVPR Proceedings}

\author{First Author\\
Institution1\\
Institution1 address\\
{\tt\small firstauthor@i1.org}
% For a paper whose authors are all at the same institution,
% omit the following lines up until the closing ``}''.
% Additional authors and addresses can be added with ``\and'',
% just like the second author.
% To save space, use either the email address or home page, not both
\and
Second Author\\
Institution2\\
First line of institution2 address\\
{\tt\small secondauthor@i2.org}
}

\maketitle


%%%%%%%%% ABSTRACT
\begin{abstract}
   The ABSTRACT is to be in fully-justified italicized text, at the top
   of the left-hand column, below the author and affiliation
   information. Use the word ``Abstract'' as the title, in 12-point
   Times, boldface type, centered relative to the column, initially
   capitalized. The abstract is to be in 10-point, single-spaced type.
   Leave two blank lines after the Abstract, then begin the main text.
   Look at previous CVPR abstracts to get a feel for style and length.
\end{abstract}

%%%%%%%%% BODY TEXT
\section{Perturbation Decomposition}

original

\[ \tilde{x} = x + r, \|r\|_p < \varepsilon, x+r \in [0,1]^N \]

assuming l-infty bound.
decomposition with global carrier + class-wise payload.

\begin{align}
	\tilde{x} &= x + r_c + r_p \\
	&= x + \varepsilon \Big(\frac{\xi}{2}\sin \theta + \frac{1-\xi}{2} \sin \rho \Big)
\end{align}
where $\rho = P\hat{y}$ is a column of the class-wise payload matrix $P$ , $0\leq \xi \leq 1$, $x+r\in [0,1]^N$.

decompostion with class-wise carrier + class-target-wise payloads.
\begin{align}
	\tilde{x} &= x + r_c + r_p \\
	&= x + \varepsilon \Big(\frac{\xi}{2}\sin \mC \vy + \frac{1-\xi}{2} \sin \tP_{y,\tilde{y},:} \Big)
\end{align}

next?

1. use these factorized components as features? classification? adversarial resetting?

2. accelerate black-box attack? digging out the intrinsics of the target black-box model?

3. the carrier is not necessarily a universal adversarial perturbation (UAP).

difference?

$p$, $c+p$, UAP; $Py$ cls wise UAP;
$c+Py$ coarse decomp (adv ptb more related to model parameters ... being source-class agnostic);
$Cy+Pyy$ fine decomp (adv ptb more related to source image ... being source-class sensitive ... chance for adv resetting);
$Cyy+Pyy$ per-source-target UAP;

%%%%%%%%%%%%%%%%%%%%%%%%%%%%%%%%%%%%%%%%%%%%%%%%%%%%%%%%%%%%%%%%%%%%%%%%%%%%%%%

{\small
\bibliographystyle{ieee_fullname}
\bibliography{egbib}
}

\end{document}
